\section{Proofs of Corollaries~\ref{corollary:kt-hilbert-space-olo-regret} and~\ref{corollary:kt-experts-regret}}
\label{section:corollaries_reductions}

First we state some technical lemmas that will be used in the following proofs.

We first prove a lower bound on the potential function of the \ac{KT} estimator. This is a generalization to real numbers of the lower bound proved for integers in \citet{WillemsST95}.
\begin{lemma}
\label{lemma:approx_gamma_real}
If $c \ge 1$ and $a,b$ are non-negative reals such that $a + b = c$ then
$$
\ln \left(\frac{\Gamma(a + 1/2) \cdot \Gamma(b + 1/2)}{\pi \cdot \Gamma(c+1)} \right)
\ge - \ln(e \pi \sqrt{2}) -\frac{1}{2} \ln(c) +\ln \left(\left( \frac{a}{c} \right)^a \left( \frac{b}{c} \right)^b\right) \; .
$$
\end{lemma}
%
\begin{proof}
Using \cite{WhittakerW27}[p. 263 Ex. 45], implies that
\[
\frac{\Gamma(a+1/2)\Gamma(b+1/2)}{\Gamma(a+b+1)} > \sqrt{2\pi} \frac{(a+1/2)^a (b+1/2)^b}{(a+b+1)^{a+b+1/2}} \ \forall a,b>0~.
\]
It remains to show that
$$
\sqrt{2\pi} \frac{(a+1/2)^a (b+1/2)^b}{(a+b+1)^{a+b+1/2}} > \frac{\sqrt{\pi}}{e} \frac{1}{\sqrt{a+b}} \left( \frac{a}{a+b} \right)^a \left( \frac{b}{a+b} \right)^b \; ,
$$
which is equavivalent to
$$
\frac{(1+\frac{1}{2a})^a (1+\frac{1}{2b})^b}{(1+\frac{1}{a+b})^{a+b+1/2}} > \frac{1}{e\sqrt{2}} \; .
$$
From the inequality $1 \le (1+1/x)^x < e$ valid for any $x \ge 0$, it follows
that $1 \le (1+\frac{1}{2a})^a < \sqrt{e}$ and $1 \le (1+\frac{1}{2b})^b < \sqrt{e}$
and $1 \le (1+1/(a+b))^{a+b} < e$. Hence,
$$
\frac{(1+\frac{1}{2a})^a (1+\frac{1}{2b})^b}{(1+\frac{1}{a+b})^{a+b+1/2}}
> \frac{1}{e \sqrt{1 + \frac{1}{a+b}}}
\ge \frac{1}{e \sqrt{2}} \; .
$$
\end{proof}


\begin{lemma}
\label{lemma:ratio_gamma}
Let $\delta \geq 0$. Then
\[
\frac{\Gamma(\delta+1)}{2^\delta \Gamma(\frac{\delta+1}{2})^2}
\geq \frac{\sqrt{\delta+1}}{\pi}~.
\]
\end{lemma}
%
\begin{proof}
We will prove the equivalent statement that
\[
\ln \frac{\Gamma(\delta+1) \pi}{2^\delta \Gamma(\frac{\delta+1}{2})^2 \sqrt{\delta+1}} \geq 0~.
\]
The inequality holds with equality in $\delta=0$, so it is enough to prove that
the derivative of the right-hand side is positive for $\delta > 0$. The
derivative of the right-hand side is equal to
$$
\Psi(\delta+1) - \frac{1}{2(\delta+1)} -\ln(2) - \Psi\left(\frac{\delta+1}{2} \right) \; ,
$$
where $\Psi(x)$ is the digamma function.

We will use the following upper~\citep{Chen05} and lower bound~\citep{Batir08} to the digamma function, for any $x>0$
\begin{align*}
\Psi(x) &< \ln(x) -\frac{1}{2x} -\frac{1}{12 x^2} +\frac{1}{120 x^4}\\
\Psi(x+1) &> \ln \left(x+\frac{1}{2} \right) \; .
\end{align*}
Hence, we obtain
\begin{align*}
&\Psi(\delta+1) - \frac{1}{2(\delta+1)} - \ln(2) - \Psi \left(\frac{\delta+1}{2} \right) \\
&\quad \ge \ln \left(\delta+\frac{1}{2} \right) - \frac{1}{2(\delta+1)} -\ln(2) -\ln \left(\frac{\delta+1}{2}\right) + \frac{1}{\delta+1} +\frac{1}{3 (\delta+1)^2} -\frac{2}{15 (\delta+1)^4}\\
&\quad = \ln \left(1+\frac{\delta}{\delta+1} \right)+ \frac{1}{2(\delta+1)} -\ln(2)+\frac{1}{3 (\delta+1)^2} -\frac{2}{15 (\delta+1)^4}\\
&\quad \ge \frac{\delta}{\delta+1} - (1-\ln(2))\frac{\delta^2}{(\delta+1)^2}+ \frac{1}{2(\delta+1)} -\ln(2)+\frac{1}{3 (\delta+1)^2}-\frac{2}{15 (\delta+1)^4}\\
&\quad = \frac{\delta^2 \ln(2)+\frac{3}{2}\delta +\frac{5}{6}}{(\delta+1)^2} - \ln(2)-\frac{2}{15 (\delta+1)^4}\\
&\quad = \frac{(\frac{3}{2}-\ln(4)) \delta +\frac{5}{6}-\ln(2)}{(\delta+1)^2} -\frac{2}{15 (\delta+1)^4} \\
&\quad = \frac{15 (\delta+1)^2 \left((\frac{3}{2}-\ln(4)) \delta +\frac{5}{6}-\ln(2)\right)-2}{15(\delta+1)^4} \\
&\quad = \frac{15 \left(\frac{5}{6}-\ln(2)\right)-2}{15(\delta+1)^4} >0,
\end{align*}
where in the second inequality we used the elementary inequality $\ln(x+1) \leq x - (1-\ln(2))x^2$ valid for $x \in [0,1]$.
\end{proof}

\begin{lemma}
\label{lemma:lower_bound_gamma}
Let $T \ge 1$, $\delta \ge 0$, and $x \in [-T,T]$. Then
\begin{align*}
\frac{2^T \cdot \Gamma(\delta+1) \Gamma \left(\frac{T+\delta+1}{2} + \frac{x}{2} \right) \cdot \Gamma \left(\frac{T+\delta+1}{2} - \frac{x}{2} \right)}{ \Gamma(\frac{\delta+1}{2})^2 \Gamma(T+\delta+1)}
&\geq\exp\left(\frac{x^2}{2(T+\delta)} +\frac{1}{2} \ln \left(\frac{1+\delta}{T+\delta}\right) - 3\right) \;.
\end{align*}
\end{lemma}
\begin{proof}
Using Lemma~\ref{lemma:approx_gamma_real}, we have
\begin{align*}
&\ln \frac{2^T \cdot \Gamma(\delta+1) \Gamma \left(\frac{T+\delta+1}{2} + \frac{x}{2} \right) \cdot \Gamma \left(\frac{T+\delta+1}{2} - \frac{x}{2} \right)}{ \Gamma(\frac{\delta+1}{2})^2 \Gamma(T+\delta+1)} \\
&\quad \geq \ln \frac{2^{T+\delta} \sqrt{\delta+1} \cdot \Gamma \left(\frac{T+\delta+1}{2} + \frac{x}{2} \right) \cdot \Gamma \left(\frac{T+\delta+1}{2} - \frac{x}{2} \right)}{ \pi \Gamma(T+\delta+1)} \\
&\quad \geq -\ln(e \pi \sqrt{2}) +\frac{1}{2} \ln\left(\frac{1+\delta}{T+\delta}\right) +\ln \left(\left( 1+\frac{x}{T+\delta} \right)^\frac{T+\delta+x}{2} \left( 1+\frac{x}{T+\delta} \right)^\frac{T+\delta-x}{2}\right) \\
&\quad = -\ln(e \pi \sqrt{2}) +\frac{1}{2} \ln\left(\frac{1+\delta}{T+\delta}\right)+ (T+\delta) \, \KL{\frac{1}{2}+\frac{x}{2(T+\delta)}}{\frac{1}{2}} \\
&\quad \geq -\ln(e \pi \sqrt{2}) +\frac{1}{2} \ln\left(\frac{1+\delta}{T+\delta}\right)+ \frac{x^2}{2(T+\delta)} ,
\end{align*}
where in the first inequality we used Lemma~\ref{lemma:ratio_gamma}, in the second one Lemma~\ref{lemma:approx_gamma_real}, and in third one the known lower bound to the divergence $\KL{\frac{1}{2}+\frac{x}{2}}{\frac{1}{2}} \geq \frac{x^2}{2}$. Exponentiating and overapproximating, we get the stated bound.
\end{proof}


\subsection{Proof of Corollary~\ref{corollary:kt-hilbert-space-olo-regret}}

Define the Lambert function $W(x):\R\rightarrow\R$ as the one that satisfies
the equality\footnote{For $x<0$ the Lambert function is multivalued. Hence, to
avoid complication and because we only need non-negative arguments, we will
define it only for positive values of $x$.}
\begin{equation}
\label{eq:lambert}
x=W(x) \exp \left(W(x)\right), \ \forall x\geq0.
\end{equation}
It satisfies the following properties.
%
\begin{lemma}
\label{lemma:lambert}
The Lambert function satisfies $0.6321 \log(x+1) \leq W(x) \leq \log(x+1), \forall x\geq0$.
\end{lemma}
%
\begin{proof}
The inequalities are satisfied for $x=0$, hence we in the following we assume $x>0$.
We first prove the lower bound. From \eqref{eq:lambert} we have
\begin{equation}
W(x) = \log\left(\frac{x}{W(x)}\right)~. \label{eq:lm_lambert_1}
%&= \log\left(\frac{x}{\log(x/W(x))}\right). \label{eq:lm_lambert_1b}
\end{equation}
From the first equality, using the elementary inequality $\ln (x) \leq \frac{a}{e} x^\frac{1}{a}$ for any $a>0$, we get
\[
W(x) \leq \frac{1}{a\, e}\left(\frac{x}{W(x)}\right)^a  \ \ \forall a>0,
\]
that is
\begin{equation}
\label{eq:lm_lambert_2}
W(x) \leq \left(\frac{1}{a\, e}\right)^\frac{1}{1+a} x^\frac{a}{1+a} \ \ \forall a>0.
\end{equation}
Using \eqref{eq:lm_lambert_2} in \eqref{eq:lm_lambert_1}, we have
\begin{align*}
W(x)
\geq \log\left(\frac{x}{\left(\frac{1}{a\, e}\right)^\frac{1}{1+a} x^\frac{a}{1+a}}\right)
= \frac{1}{1+a}\log\left(a \, e\, x\right) \ \ \forall a>0~.
\end{align*}
Consider now the function $g(x)=\frac{x}{x+1} - \frac{b}{\log(1+b) (b+1)}
\log(x+1), x\geq b$. This function has a maximum in $x^*=(1+\frac{1}{b})
\log(1+b)-1$, the derivative is positive in $[0,x^*]$ and negative in
$[x^*,b]$. Hence the minimum is in $x=0$ and in $x=b$, where it is equal to
$0$.  Using the property just proved on $g$, setting $a=\frac{1}{x}$, we have
\begin{align*}
W(x)
\geq \frac{x}{x+1} \geq \frac{b}{\log(1+b) (b+1)} \log(x+1) \ \  \forall x\leq b~.
\end{align*}
For $x>b$, setting $a=\frac{x+1}{e x}$, we have
\begin{align}
W(x)
&\geq \frac{e\,x}{(e+1) x + 1} \log(x+1) \geq \frac{e\,b}{(e+1) b + 1} \log(x+1)
\end{align}
Hence, we set $b$ such that
\[
\frac{e\, b}{(e+1)b + 1} = \frac{b}{\log(1+b) (b+1)}
\]
Numerically, $b=1.71825...$, so
\[
W(x) \geq 0.6321 \log(x+1)~.
\]

For the upper bound, we use Theorem~2.3 in \cite{hoorfar2008inequalities}, that says that
\[
W(x) \leq \log\frac{x+C}{1+\log(C)}, \quad \forall x> -\frac{1}{e}, \ C>\frac{1}{e}.
\]
Setting $C=1$, we obtain the stated bound.
\end{proof}

\begin{lemma}
Define $f(\theta)= \beta \exp\frac{x^2}{2 \alpha}$, for $\alpha,\beta>0$, $x\geq0$. Then
\[
f^*(y)=y \sqrt{\alpha W\left(\frac{\alpha y^2}{\beta^2}\right)} - \beta \exp\left(\frac{W\left(\frac{\alpha y^2}{\beta^2}\right)}{2}\right).
\]
Moreover
\[
f^*(y) \leq y \sqrt{\alpha \log \left(\frac{\alpha y^2}{\beta^2} +1 \right)} - \beta.
\]
\end{lemma}
\begin{proof}
From the definition of Fenchel dual, we have
\begin{align*}
f^*(y)= \max_{x} \  x\, y - f(x) = \max_{x} \  x\, y - \beta \exp\frac{x^2}{2 \alpha} \leq x^*\,y -\beta
\end{align*}
where $x^*= \argmax_{x} x\, y - f(x)$. We now use the fact that $x^*$ satisfies $y = f'(x^*)$, to have
\begin{align*}
x^*=\sqrt{\alpha W\left(\frac{\alpha y^2}{\beta^2}\right)},
\end{align*}
where the $W(\cdot)$ is the Lambert function.
Using Lemma~\ref{lemma:lambert}, we obtain the stated bound.
\end{proof}

\begin{corollary}
\label{cor:dual_exp_square}
Define $f(\theta)= \beta \exp\frac{\norm{\theta}^2}{2 \alpha}$, for $\alpha,\beta>0$. Then
\[
f^*(w) \leq  \norm{w} \sqrt{\alpha \log \left(\frac{\alpha \norm{w}^2}{\beta^2} +1 \right)} - \beta.
\]
\end{corollary}


\begin{proof}[Proof of Corollary~\ref{corollary:kt-hilbert-space-olo-regret}]
Define $G_t(x)=\exp\left(\frac{x^2}{2(t+\delta)} +\frac{1}{2} \ln \frac{1+\delta}{t+\delta} -3\right)$ and $g_t(x)=\ln(G_t(x))$.
Using Lemma~\ref{lemma:lower_bound_gamma}, we have
\[
G_t(x) \leq \frac{2^T \cdot \Gamma(\delta+1) \Gamma \left(\frac{T+\delta+1}{2} + \frac{x}{2} \right) \cdot \Gamma \left(\frac{T+\delta+1}{2} - \frac{x}{2} \right)}{ \Gamma(\frac{\delta+1}{2})^2 \Gamma(T+\delta+1)},
\]
that implies $F^*_t(x) \leq G_t^*(x)$.
Using Corollary~\ref{cor:dual_exp_square}, we have
\[
\forall u \in \H \qquad \qquad
F^*_t\left(\left\| \sum_{t=1}^T g_t \right\|\right)
\leq \sqrt{T \log \left(\frac{4 T^2 \norm{u}^2}{\epsilon^2} +1 \right)}+\epsilon\left(1-\frac{1}{2\sqrt{T}}\right).
\]
An application of Theorem~\ref{theorem:hilbert-space-olo-regret-bound} completes the proof.
\end{proof}

\subsection{Proof of Corollary~\ref{corollary:kt-experts-regret}}

\begin{proof}
Define $G_t(x)=\exp\left(\frac{x^2}{2(t+\delta)} +\frac{1}{2} \ln \frac{1+\delta}{t+\delta} -3\right)$ and $g_t(x)=\ln(G_t(x))$.
Using Lemma~\ref{lemma:lower_bound_gamma}, we have
\[
G_t(x) \leq \frac{2^T \cdot \Gamma(\delta+1) \Gamma \left(\frac{T+\delta+1}{2} + \frac{x}{2} \right) \cdot \Gamma \left(\frac{T+\delta+1}{2} - \frac{x}{2} \right)}{ \Gamma(\frac{\delta+1}{2})^2 \Gamma(T+\delta+1)} \;.
\]
From the above, it is immediate to have that $f^{-1}_t(x) \leq g^{-1}_t(x)$.
Setting $\delta=\frac{T}{2}$ and overapproximating we get the stated bound.
\end{proof}
