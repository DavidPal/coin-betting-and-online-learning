\section{From Coin Betting to OLO over Hilbert Space}

We show how to use a sequence of coin-betting potentials $\{F_t\}_{0=1}^\infty$
to construct an algorithm for online linear optimization over a Hilbert space
and how to prove regret bound for it. The basic idea is to realize that
continuous coin-betting corresponds to certain type algorithms
for online linear optimization over one-dimensional Hilbert space $\R$.
This idea can be then generalized to arbitrary Hilbert spaces.

\subsection{One-Dimensional Hilbert Space}

As a warm-up, let us consider an algorithm for OLO over one-dimensional Hilbert
space $\R$. Let $\{w_t\}_{t=1}^\infty$ be its sequence of predictions on a
sequence of rewards $\{g_t\}_{t=1}^\infty$, $g_t \in [-1,1]$. The total reward
of the algorithm is
$$
\Reward_t = \sum_{i=1}^t g_i w_i \; .
$$
Let us define its ``wealth'' of the OLO algorithm as $\Wealth_t = \epsilon +
\Reward_t$, i.e., according to the equation \eqref{equation:reward-wealth}. Now,
suppose we want to satisfy the recurrence \eqref{equation:wealth-recurrence}.
Clearly, the recurrence is not necessarily satisfied for an arbitrary
OLO algorithm. However, if we assume that its predictions are of the form
\begin{equation}
\label{equation:one-dimensional-olo}
w_t = \beta_t \Wealth_{t-1}
\end{equation}
where $\beta_t \in (-1,1)$, we see that th recurrence
\eqref{equation:wealth-recurrence} holds. Indeed,
$$
\Wealth_t
= g_t w_t + \Wealth_{t-1}
= \beta_t \Wealth_{t-1} + \Wealth_{t-1}
= (1+\beta_t) \Wealth_{t-1} \; .
$$
This of course works in reverse: If we have a coin-betting algorithm that on a
sequence of coin flips $\{g_t\}_{t=1}^\infty$, $g_t \in [-1,1]$, bets fractions
$\beta_t \in (-1,1)$ we can use it to construct an OLO algorithm in a
one-dimensional Hilbert space $\R$ according to equation
\eqref{equation:one-dimensional-olo}.

If the betting algorithm is based on a sequence of coin-betting potentials
$\{F_t\}_{t=1}^\infty$ then according to \eqref{equation:wealth-lower-bound-generic},
\begin{equation}
\label{equation:one-dimensional-olo-reward-lower-bound}
\Reward_T
= \sum_{t=1}^T g_t, w_t
= \Wealth_T \ - \ \epsilon \ge F_T\left(\sum_{t=1}^T g_t \right) \ - \ \epsilon \; .
\end{equation}

It is straightforward to convert a lower bound on the reward to an upper bound
on the regret. This can be done using the following lemma, as observed
by~\cite{McMahanO14}.

\begin{lemma}[Reward-Regret relationship]
\label{lemma:reward-regret}
Let $V,V^*$ be a pair dual vector spaces. Let $F:V \to \R \cup \{+\infty\}$ be a
proper convex lower semi-continuous function and let $F^*:V^* \to \R \cup
\{+\infty\}$ be its Fenchel conjugate. Let $w_1, w_2, \dots, w_T \in V$ and
$g_1, g_2, \dots, g_T \in V^*$ be two sequences of vectors.  Then, the condition
$$
\underbrace{\sum_{t=1}^T \langle g_t, w_t \rangle}_{=\Reward_T} \ge F\left( \sum_{t=1}^T g_t \right)
$$
is equivalent to
$$
\forall u \in V^* \qquad \qquad
\underbrace{\sum_{t=1}^T \langle g_t, u - w_t\rangle}_{=\Regret_T(u)} \le F^*(u) \; .
$$
\end{lemma}

Applying the lemma to the function $F(x) = F_T(x) - \epsilon$, we get a regret
upper bound
$$
\forall u \in \H \qquad \qquad
\Regret_T(u) \le F_T^*\left(\sum_{t=1}^T g_t \right) \ + \ \epsilon \; .
$$

\subsection{Arbitrary Hilbert Space}

The one-dimensional construction for OLO can be generalized to an arbitrary
Hilbert space $\H$. Reward and wealth are defined analogously
to the one-dimensional case:
\begin{align*}
\Reward_t &= \sum_{i=1}^t \langle g_i, w_i \rangle &
& \text{and} &
\Wealth_t &= \epsilon + \Reward_t \; .
\end{align*}
Given a sequence of coin-betting potentials $\{F_t\}_{t=0}^\infty$,
we define fraction
\begin{equation}
\label{equation:potential-based-strategy-hilbert-space}
\beta_t = \frac{F_t \left(\norm{\sum_{i=1}^{t-1} g_i} + 1\right) - F_t\left(\norm{\sum_{i=1}^{t-1} g_i} - 1 \right)}{F_t\left(\norm{\sum_{i=1}^{t-1} g_i} + 1 \right) + F_t\left(\norm{\sum_{i=1}^{t-1} g_i} - 1 \right)} \; .
\end{equation}
This definition of $\beta_t$ is a generalization of equation
\eqref{equation:potential-based-strategy}.  Analogously to equation
\eqref{equation:one-dimensional-olo}, the prediction of the OLO algorithm
defined by this potentials is
\begin{equation}
\label{equation:hilbert-space-olo}
w_t = \beta_t \Wealth_{t-1} \frac{\sum_{i=1}^{t-1} g_i}{\norm{\sum_{i=1}^{t-1} g_i}}  \; .
\end{equation}
The only difference between \eqref{equation:hilbert-space-olo} and
\eqref{equation:one-dimensional-olo} is the multiplication by the unit vector
$\frac{\sum_{i=1}^{t-1} g_i}{\norm{\sum_{i=1}^{t-1} g_i}}$. If $\sum_{i=1}^{t-1}
g_i$ is the zero vector, we define $w_t$ be the zero vector as well. For
arbitrary Hilbert space, the hard part is to show an analogue of
inequality \eqref{equation:wealth-lower-bound-generic},
\begin{equation}
\label{equation:wealth-lower-bound-hilbert-space}
\Wealth_t \ge F_t\left(\norm{\sum_{t=1}^T g_t} \right) \; .
\end{equation}
To prove \eqref{equation:wealth-lower-bound-hilbert-space}, we imitate the induction
proof from Section~\ref{section:coin-betting-potentials}. The base case
$t=0$ is trivial, since both sides of the inequality equal $\epsilon$.
If $t \ge 1$, if we let $x = \sum_{i=1}^{t-1} g_i$, we have
\begin{multline*}
\Wealth_t
= \langle g_t, w_t \rangle + \Wealth_{t-1}
= \left(1 + \beta_t \frac{\langle g_t, x \rangle}{\|x\|} \right) \Wealth_{t-1} \\
\ge \left(1 + \beta_t \frac{\langle g_t, x \rangle}{\|x\|} \right) F_{t-1}(\norm{x})
\stackrel{\text{\textbf{(?)}}}{\ge} F_t(\norm{x + g_t})
= F_t\left(\norm{\sum_{i=1}^t g_i} \right) \; .
\end{multline*}
The only non-trivial inequality is marked with a question mark. The inequality
is the content of the Lemma~\ref{lemma:recursion_hilbert}. Its proof can be
found in Appendix~\ref{section:hilbert-space-reduction}. The proof relies
mainly on property (4) of Definition~\ref{definition:potential}.

\begin{lemma}
\label{lemma:recursion_hilbert}
Let $\{F_t\}_{t=0}^\infty$ be a sequence of excellent coin-betting potentials.
Let $g_1, g_2, \dots, g_t$ be vectors in a Hilbert space $\H$. Let $\beta_t$
be defined by \eqref{equation:potential-based-strategy-hilbert-space}
and let $x = \sum_{i=1}^{t-1} g_i$. Then,
$$
\left(1 + \beta_t \frac{\langle g_t, x \rangle}{\|x\|} \right) F_{t-1}(\norm{x})
\ge F_t(\norm{x + g_t}) \; .
$$
\end{lemma}

Once the inequality \eqref{equation:wealth-lower-bound-hilbert-space} is
established, same as in the one-dimensional case, we have a reward lower bound
\begin{equation}
\label{equation:hilbert-space-olo-reward-lower-bound}
\Reward_T
= \sum_{t=1}^T \langle g_t, w_t \rangle
= \Wealth_T \ - \ \epsilon
\ge F_T\left(\norm{\sum_{t=1}^T g_t} \right) \ - \ \epsilon \; .
\end{equation}
We apply Lemma~\ref{lemma:reward-regret} to the function $F(x) = F_T(\norm{x}) -
\epsilon$ and we obtain a regret bound. We state the result as a theorem.

\begin{theorem}[Regret Bound for OLO in Hilbert Spaces]
\label{theorem:hilbert-space-olo-regret-bound}
Let $\{F_t\}_{t=0}^\infty$ be a sequence of excellent potentials. Let
$\{g_t\}_{t=1}^\infty$ be any sequence of reward vectors in a Hilbert space $\H$
such that $\|g_t\| \le 1$ for all $t$. Then, the algorithm that makes prediction
$w_t$ defined by \eqref{equation:hilbert-space-olo} and
\eqref{equation:potential-based-strategy-hilbert-space} has regret bounded as
$$
\forall u \in \H \qquad \qquad
\Regret_T(u) \le F_T^*\left(\norm{\sum_{t=1}^T g_t} \right) \ + \ \epsilon \; .
$$
\end{theorem}

For the proof of the theorem (besides Lemma~\ref{lemma:reward-regret}) the only
property we need is that if $F$ is an even function then Fenchel conjugate of
$F(\norm{\cdot})$ is $F^*(\|\cdot\|)$; see \citet[Example 13.7]{BauschkeC2011}.
