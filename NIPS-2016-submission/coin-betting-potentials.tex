\section{Coin-Betting Potentials}
\label{section:coin-betting-potentials}

\citet{Kelly-1956} proposed a general strategy for sequential bets. For coin
betting, the strategy assumes that the coin flips $\{g_t\}_{t=1}^\infty$, $g_t
\in \{+1,-1\}$, are generated i.i.d. with known probability of heads. If $p \in
[0,1]$ is the probability of heads, the Kelly bet is
$\beta_t = 2p - 1$.
He showed that, in the long run, this strategy will provide more wealth than
betting any other fixed fraction~\cite{Kelly-1956}.

For adversarial coins, Kelly betting does not make sense. However, we can
replace $p$ with an estimate. This problem was studied by \citet{Krichevsky-Trofimov-1981},
who proposed that after seeing coin flips $g_1, g_2, \dots, g_{t-1}$ the
empirical estimate $k_t = \frac{1/2 + \sum_{i=1}^{t-1} \indicator[g_i = +1]}{t}$
should be used instead of $p$. Their estimate is commonly called \emph{KT
estimator}.\footnote{Compared to the maximum likelihood estimate
$\frac{\sum_{i=1}^{t-1} \indicator[g_i = +1]}{t-1}$, KT estimator shrinks
slightly towards $\nicefrac{1}{2}$.} The KT estimator results in the betting strategy
\begin{equation}
\label{equation:kt-estimator-betting-strategy}
\beta_t = 2k_t - 1 = \tfrac{\sum_{i=1}^{t-1} g_i}{t}
\end{equation}
which we call \emph{adaptive Kelly betting based on KT estimator}.
\citeauthor{Krichevsky-Trofimov-1981} showed that
\[
\ln(\Wealth_t) \ge \max_{\beta \in [-1,1]} \  \ln(\Wealth_t(\beta)) \ - \ \tfrac{1}{2} \ln t \ - \ \ln(2) \; ,
\]
where $\Wealth_t(\beta)$ is the wealth of a strategy that bets the same
fraction $\beta$ in every round. This guarantee is optimal up to constant
additive factors~\citep{Cesa-Bianchi-Lugosi-2006} and mirrors the guarantee of the Kelly bet.

Here, we propose a new set of definitions that allows to generalize the
strategy of adaptive Kelly betting based on KT estimator. For these strategies it will be
possible to prove that, for $g_1, g_2, \dots, g_t \in [-1,1]$,
\begin{equation}
\label{equation:wealth-lower-bound-generic}
\Wealth_t \ge F_t \left( \sum_{i=1}^t g_i \right) \; ,
\end{equation}
where $F_t(x)$ is a certain function. We call such functions \emph{potentials}.
The betting strategy will be determined uniquely by the potential (see (c) in the Definition~\ref{definition:potential}),
and we restrict our
attention to potentials for which \eqref{equation:wealth-lower-bound-generic}
holds. These constraints are specified in the definition below.
\begin{definition}[Coin-Betting Potential]
\label{definition:potential}
Let $\epsilon > 0$. Let $\{F_t\}_{t=0}^\infty$ be a sequence of functions
$F_t:(-a_t, a_t)  \to \R_+$, where $a_t > t$.  The sequence
$\{F_t\}_{t=0}^\infty$ is called a \textbf{sequence of coin-betting potentials
for initial endowment $\epsilon$}, if it satisfies the following three
conditions:
\begin{enumerate}[(a)]
\item $F_0(0) = \epsilon$.

\item For every $t \ge 0$, $F_t(x)$ is even, logarithmically convex, strictly
increasing on $[0,a_t)$, and
$\lim_{x \to a_t} F_t(x) = +\infty$.
\item Given $t \ge 1$ and $x \in [-(t-1), (t-1)]$, let 
\begin{equation}
\label{equation:potential-based-strategy}
\beta_t=\tfrac{F_t(x + 1) - F_t(x - 1)}{F_t(x + 1) + F_t(x - 1)} \;.
\end{equation}
For every $t \ge 1$, every $x \in [-(t-1), (t-1)]$ and every $g \in [-1,1]$, $\left(1 + g \beta_t \right) F_{t-1}(x) \ge F_t(x+g)$.
\end{enumerate}
The sequence $\{F_t\}_{t=0}^\infty$ is called a
\textbf{sequence of excellent coin-betting potentials for initial
endowment $\epsilon$} if it satisfies conditions (1)--(3) and the condition (4)
below.
\begin{enumerate}[(a)]
\setcounter{enumi}{3}
\item For every $t \ge 0$, $F_t$ is twice-differentiable and
satisfies $x \cdot F_t''(x) \ge F_t'(x)$ for every $x \in [0,a_t)$.
\end{enumerate}
\end{definition}

Let's give some intuition on this definition.  First, let's show by induction
on $t$ that (b) and (c) of the definition together with
\eqref{equation:wealth-recurrence} give a betting strategy that satisfies
\eqref{equation:wealth-lower-bound-generic}. The base case $t=0$ is trivial. At
time $t \ge 1$, let $x = \sum_{i=1}^{t-1} g_i$ and bet $\beta_t \Wealth_{t-1}$
where $\beta_t$ is defined in \eqref{equation:potential-based-strategy}, then
\vspace{-.2cm}
\[
\Wealth_t
= (1+g_t \beta_t) \Wealth_{t-1}
\ge (1 + g_t \beta_t) F_{t-1}(x)
\ge F_t(x + g_t)
= F_t \left( \sum_{i=1}^t g_i \right) \; .
\]

With perfect knowledge of the future, the gambler could always bet everything
on the right outcome. Hence, after $T$ rounds and with initial endowment
$\epsilon$, the maximum wealth he could get is $\epsilon 2^T$.  Instead,
assuming the knowledge of the value of $\sum_{t=1}^T g_t$ beforehand and betting the
fixed fraction $(\sum_{t=1}^T g_t)/T$ at each round, gives the wealth $\epsilon
\exp\left(T \KL{\tfrac{1}{2}+\tfrac{\sum_{t=1}^T g_t}{2
T}}{\tfrac{1}{2}}\right) \geq \epsilon \exp\left(\tfrac{(\sum_{t=1}^T g_t)^2}{2
T}\right)$, as observed in \cite{McMahan-Abernethy-2013}.  In this sense,
\emph{good} betting strategies should guarantee an exponential growth of the
wealth, when the outcomes of the coin are predictable.

The condition (d) is a technical condition that will allow us to
seamlessly reduce the infinite dimensional case to the one dimensional one.

As the last point, the formula for the potential-based
strategy~\eqref{equation:potential-based-strategy} might seem strange. However,
it is derived---see Theorem~\ref{theorem:optimal-betting-fraction} in
Appendix~\ref{section:optimal-betting-fraction}---by minimizing the worst-case
value of the right-hand side of the inequality used w.r.t. to $g_t$ in the
induction proof above: $F_{t-1}(x) \ge \tfrac{F_{t}(x + g_t)}{1+g_t\beta_t}$.

As an example, it possible to show that the potentials of form $F_t(x)=\epsilon
\exp \left(x^2/(2t)\right)/\sqrt{t}$, essentially the ones used in the
parameter-free algorithms in \cite{McMahan-Orabona-2014, Orabona-2014} for
\ac{OLO} and in \cite{Chaudhuri-Freund-Hsu-2009, Luo-Schapire-2014,
Luo-Schapire-2015} for \ac{LEA}, is an excellent coin betting potential.  We
will show new examples of sequences of coin-betting potentials in
Section~\ref{section:kt-estimator}.  In the next two sections, we reduce
\ac{OLO} and \ac{LEA} to coin-betting.
