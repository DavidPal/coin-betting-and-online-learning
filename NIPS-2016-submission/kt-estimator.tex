\section{Applications of Krichevsky-Trofimov Estimator}
\label{section:kt-estimator}

In the previous two sections we have shown that a coin-betting potential with a
guaranteed rapid growth of the wealth will give good regret guarantees for
\ac{OLO} and \ac{LEA}. In this section we show that the optimal
Krichevsky-Trofimov (KT) estimator has associated a sequence of excellent
coin-betting potentials, which we call \emph{KT potentials}. We then prove
corollaries of the regret bounds for \ac{OLO} over Hilbert space and \ac{LEA}
that we have proved in previous sections, obtaining optimal regret bounds.

The potential corresponding to adaptive Kelly betting strategy
$\beta_t$ defined by \eqref{equation:kt-estimator-betting-strategy}
based on the KT estimator is
\begin{equation}
\label{equation:kt-estimator-potential}
F_t(x) = \epsilon \tfrac{2^t \cdot \Gamma \left( \tfrac{t+1}{2} + \frac{x}{2} \right) \cdot \Gamma \left( \tfrac{t+1}{2} - \frac{x}{2} \right)}{\pi \cdot t!}
\qquad \qquad \text{$t \ge 0$, \quad $x \in \left(-t-1, t+1\right)$,}
\end{equation}
where $\Gamma(x) = \int_0^\infty t^{-x} e^{-t} dt$ is Euler's gamma function.
The potential was introduced by~\citet{KrichevskyT81} who used it for proving
regret bound for online prediction with log-loss; see also \cite[Section
9.7]{Cesa-BianchiL06}.
Theorem~\ref{theorem:kt-potential} stated in
Appendix~\ref{section:properties-kt-potential} shows that
\eqref{equation:kt-estimator-potential} is a sequence of excellent coin-betting
potentials for initial endowment $\epsilon$. Theorem~\ref{theorem:kt-potential}
also shows that KT betting strategy $\beta_t$ as defined by
\eqref{equation:kt-estimator-betting-strategy}, is a potential-based strategy,
i.e., it satisfies \eqref{equation:potential-based-strategy}.

\subsection{OLO in Hilbert Space}
\label{sec:kt-olo}

\begin{algorithm}[t]
\caption{Algorithm for OLO over Hilbert space $\H$ based on KT potential
\label{algorithm:kt-hilbert-space-olo}}
\begin{algorithmic}[1]
{
\REQUIRE{Initial endowment $\epsilon > 0$}
%\STATE{Initialize $\Wealth_0 \leftarrow \epsilon$}
\FOR{$t=1,2,\dots$}
\STATE{Predict with $w_t \leftarrow \tfrac{1}{t} \left(\epsilon + \sum_{i=1}^{t-1} \langle g_i, w_i \rangle \right) \sum_{i=1}^{t-1} g_i$}
%\STATE{Predict $w_t$}
\STATE{Receive reward vector $g_t \in \H$ such that $\norm{g_t} \le 1$}
%\STATE{Update $\Wealth_t \leftarrow \Wealth_{t-1} \ + \ \langle g_t, w_t \rangle$}
\ENDFOR
}
\end{algorithmic}
\end{algorithm}

We apply KT potential for construction of an OLO algorithm over a Hilbert
space $\H$. According to \eqref{equation:hilbert-space-olo}, the resulting algorithm predicts
in round $t$,
\[
w_t = \beta_t \Wealth_{t-1} \tfrac{\sum_{i=1}^{t-1} g_i}{\norm{\sum_{i=1}^{t-1} g_i}}
\]
where $\beta_t$ is defined by
\eqref{equation:potential-based-strategy-hilbert-space}. According to
Theorem~\ref{theorem:kt-potential} in Appendix
\ref{section:properties-kt-potential}, the formula for $\beta_t$ simplifies to
$\beta_t = \frac{\norm{\sum_{i=1}^{t-1} g_i}}{t}$. Hence, the prediction can be
written as
\[
w_t
= \tfrac{1}{t} \Wealth_{t-1} \sum_{i=1}^{t-1} g_i
= \tfrac{1}{t} \left(\epsilon + \sum_{i=1}^{t-1} \langle g_i, w_i \rangle \right) \sum_{i=1}^{t-1} g_i \; .
\]
The algorithm is stated as Algorithm~\ref{algorithm:kt-hilbert-space-olo}.
We derive a regret bound for it by applying Theorem~\ref{theorem:hilbert-space-olo-regret-bound} to the KT
potential \eqref{equation:kt-estimator-potential}. The regret bound is stated as
Corollary~\ref{corollary:kt-hilbert-space-olo-regret} below. Its proof can be
found in the Appendix~\ref{section:corollaries_reductions}. The only technical
part of the proof is an upper bound on Fenchel conjugate $F_t^*$ since it cannot
be expressed as an elementary function.

\begin{corollary}[Regret Bound for Algorithm~\ref{algorithm:kt-hilbert-space-olo}]
\label{corollary:kt-hilbert-space-olo-regret}
Let $\epsilon > 0$. Let $\{g_t\}_{t=1}^\infty$ be any sequence of reward vectors
in a Hilbert space $\H$ such that $\norm{g_t} \le 1$.
Algorithm~\ref{algorithm:kt-hilbert-space-olo} satisfies
\[
\forall \, T \ge 0 \quad
\forall u \in \H \qquad \qquad
\Regret_T(u) \le \norm{u} \sqrt{T \ln\left(1 + \tfrac{4T^2 \norm{u}^2}{\epsilon^2} \right)} + \epsilon \left(1 - \tfrac{1}{2\sqrt{T}} \right) \;.
\]
\end{corollary}

\subsection{Learning with Expert Advice}
\label{sec:kt-lea}

\begin{algorithm}[t]
\begin{algorithmic}[1]
\caption{Algorithm for Learning with Expert Advice based on shifted KT potential
\label{algorithm:kt-experts}}
{
\REQUIRE{Number of experts $N$, number of rounds $T$, prior distribution $\pi \in \Delta_N$}
\FOR{$t=1,2,\dots,T$}
\STATE{For each $i \in [N]$, set $w_{t,i} \leftarrow \tfrac{\sum_{j=1}^{t-1} g_{j,i}}{t+T/2} \left(1 + \sum_{j=1}^{t-1} g_{j,i} w_{j,i} \right)$}
\STATE{For each $i \in [N]$, set $\widehat{p}_{t,i} \leftarrow \pi_i [w_{t,i}]_+$}
\STATE{Predict with $p_t \leftarrow
\begin{cases}
\widehat{p}_t/\norm{\widehat{p_t}}_1 & \text{if $\norm{\widehat p_t}_1 > 0$} \\
\tfrac{1}{N}(1,1,\dots,1) & \text{if $\norm{\widehat p_t}_1 = 0$}
\end{cases}$}
%\STATE{Predict $p_t$}
\STATE{Receive loss vector $\ell_t \in [0,1]^N$}
\STATE{For each $i \in [N]$, set $g_{t,i} \leftarrow \begin{cases}
\langle \ell_t, p_t \rangle - \ell_{t,i} & \text{if $w_{t,i} > 0$} \\
[\langle \ell_t, p_t \rangle - \ell_{t,i}]_+ & \text{if $w_{t,i} \le 0$}
\end{cases}$}
\ENDFOR
}
\end{algorithmic}
\end{algorithm}

We will construct an algorithm for Learning with Expert Advice based on
\emph{shifted KT potential}. The shifted potential and the resulting algorithm
requires to know the number of rounds $T$ in advance. The shifted KT
potential is defined as
\[
F_t(x) = \tfrac{2^t \cdot \Gamma\left(T/2 + 1 \right) \cdot \Gamma\left(\tfrac{t+T/2+1}{2} + \frac{x}{2} \right) \cdot \Gamma\left(\tfrac{t+T/2+1}{2} - \frac{x}{2} \right)}{\Gamma\left(\tfrac{T/2+1}{2} \right)^2 \cdot \Gamma \left(t+T/2+1\right)} \; .
\]
The reason for its name is that, up to a multiplicative constant, $F_t$ is equal
to the KT potential shifted in time by $T/2$, i.e., $t$ is replaced by $T/2+t$.
According to Theorem~\ref{theorem:kt-potential} in Appendix
\ref{section:properties-kt-potential}, the shifted KT potentials form
a sequence of coin-betting potentials for initial endowment $1$. Furthermore, the
corresponding betting fraction is
\[
\beta_t = \tfrac{\sum_{j=1}^{t-1} g_j}{T/2+t} \; .
\]
Recall that for construction of the final algorithm, we need, as an intermediate
step, an OLO algorithm for one-dimensional Hilbert space $\R$. This algorithm
predicts for any sequence $\{g_t\}_{t=1}^\infty$ of reward vectors,
\[
w_t
= \beta_t \Wealth_{t-1}
= \beta_t \left(1 + \sum_{j=1}^{t-1} g_j w_j \right)
= \frac{\sum_{i=1}^{t-1} g_i}{T/2+t} \left(1 + \sum_{j=1}^{t-1} g_j w_j \right) \; .
\]
Following the construction in Section~\ref{section:reduction-experts}, we arrive
at the final algorithm, Algorithm~\ref{algorithm:kt-experts}.

We can derive a regret bound for Algorithm~\ref{algorithm:kt-experts} by
applying Theorem~\ref{theorem:regret-bound-experts} to the shifted KT potential.
The result is stated as Corollary~\ref{corollary:kt-experts-regret} below. The
proof of the corollary is in the Appendix~\ref{section:corollaries_reductions}.
The technical part of the proof is an upper bound on $f_t^{-1}(x)$, which we
conveniently do by lower bounding $F_t(x)$. The reason for using the shifted
potential comes from the analysis of $f_t^{-1}(x)$. The unshifted algorithm would
have $O(\sqrt{T (\log T + \KL{u}{\pi}})$ regret bound; shifting improves the
bound to $O(\sqrt{T (1 + \KL{u}{\pi}})$.

\begin{corollary}[Regret Bound for Algorithm~\ref{algorithm:kt-experts}]
\label{corollary:kt-experts-regret}
Let $N \ge 2$ and $T \ge 0$ be integers. Let $\pi \in \Delta_N$ be a prior.
For any sequence $\ell_1, \ell_2, \dots, \ell_T \in
[0,1]^N$ of loss vectors, Algorithm~\ref{algorithm:kt-experts}
with input $N,T,\pi$ satisfies
\[
\forall u \in \Delta_N \qquad \qquad \Regret_T(u) \le \sqrt{3T (4 + \KL{u}{\pi})} \; .
\]
\end{corollary}
By changing $T/2$ in Algorithm~\ref{algorithm:kt-experts} to another constant
fraction of $T$, it is possible to trade-off between the two constants $3$ and
$4$ present in the square root.

The requirement of knowing the number of rounds $T$ in advance can be lifted by
the standard doubling trick~\citep[Section 2.3.1]{Shalev-Shwartz12}. We obtain
an anytime algorithm at the expense of slightly worse regret bound,
\[
\forall \, T \ge 0 \quad \forall u \in \Delta_N \qquad \qquad
\Regret_T(u) \le \tfrac{\sqrt{2}}{\sqrt{2} - 1} \sqrt{3T (4 + \KL{u}{\pi})} \; .
\]

Also, as observed by \citet{ChernovV10}, bounds in terms of the KL
divergence are superior to the $\epsilon$-quantile bounds.
