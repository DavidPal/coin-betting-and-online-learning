\section{Applications of Krichevsky-Trofimov Estimator}
\label{section:kt-estimator}

In the previous section we have shown that a coin-betting potential with a
guaranteed rapid growth of the wealth will give good regret guarantees for
\ac{OLO} and \ac{LEA}. Here, we show that the KT estimator has associated an
excellent coin-betting potential, which we call \emph{KT potential}.  Then, the
optimal wealth guarantee of the KT potentials will translate to optimal
parameter-free regret bounds.

The excellent coin-betting potential for initial endowment $\epsilon$
corresponding to the adaptive Kelly betting strategy
$\beta_t$ defined by \eqref{equation:kt-estimator-betting-strategy}
based on the KT estimator is
\begin{equation}
\label{equation:kt-estimator-potential}
F_t(x) = \epsilon \tfrac{2^t \cdot \Gamma \left( \tfrac{t+1}{2} + \frac{x}{2} \right) \cdot \Gamma \left( \tfrac{t+1}{2} - \frac{x}{2} \right)}{\pi \cdot t!}
\qquad \qquad \text{$t \ge 0$, \quad $x \in \left(-t-1, t+1\right)$,}
\end{equation}
where $\Gamma(x) = \int_0^\infty t^{-x} e^{-t} dt$ is Euler's gamma
function---see Theorem~\ref{theorem:kt-potential} in
Appendix~\ref{section:properties-kt-potential}.  This potential was used to
prove regret bounds for online prediction with the logarithmic
loss~\cite{Krichevsky-Trofimov-1981}\cite[Section
9.7]{Cesa-Bianchi-Lugosi-2006}.  Theorem~\ref{theorem:kt-potential} also shows
that the KT betting strategy $\beta_t$ as defined by
\eqref{equation:kt-estimator-betting-strategy} satisfies
\eqref{equation:potential-based-strategy}.

This potential has the peculiar property to satisfy the inequality in (c) of
Definition~\ref{definition:potential} with equality when $g_t\in \{-1,1\}$,
i.e. $F_t(x+g_t)=(1+g_t \beta_t) \, F_{t-1}(x)$.

We also generalize the KT potential to a \emph{$\delta$-shifted KT
potential}, where $\delta\geq0$, defined as
\[
F_t(x) = \tfrac{2^t \cdot \Gamma\left(\delta + 1 \right) \cdot \Gamma\left(\tfrac{t+\delta+1}{2} + \frac{x}{2} \right) \cdot \Gamma\left(\tfrac{t+\delta+1}{2} - \frac{x}{2} \right)}{\Gamma\left(\tfrac{\delta+1}{2} \right)^2 \cdot \Gamma \left(t+\delta+1\right)} \; .
\]
The reason for its name is that, up to a multiplicative constant, $F_t$ is
equal to the KT potential shifted in time by $\delta$.
Theorem~\ref{theorem:kt-potential} also proves that the $\delta$-shifted KT
potentials are excellent coin-betting potentials for initial endowment $1$, and
the corresponding betting fraction is $\beta_t = \tfrac{\sum_{j=1}^{t-1}
g_j}{\delta+t}$.

\vspace{-0.1cm}

\subsection{OLO in Hilbert Space}
\label{section:kt-olo}

We apply the KT potential for the construction of an OLO algorithm over a
Hilbert space $\H$. We will use \eqref{equation:hilbert-space-olo}, and we just
need to calculate $\beta_t$. According to Theorem~\ref{theorem:kt-potential} in
Appendix \ref{section:properties-kt-potential}, the formula for $\beta_t$
simplifies to $\beta_t = \frac{\norm{\sum_{i=1}^{t-1} g_i}}{t}$ so that $w_t =
\tfrac{1}{t} \left(\epsilon + \sum_{i=1}^{t-1} \langle g_i, w_i \rangle \right)
\sum_{i=1}^{t-1} g_i$.

\begin{algorithm}[t]
\caption{Algorithm for OLO over Hilbert space $\H$ based on KT potential
\label{algorithm:kt-hilbert-space-olo}}
\begin{algorithmic}[1]
{
\REQUIRE{Initial endowment $\epsilon > 0$}
\FOR{$t=1,2,\dots$}
\STATE{Predict with $w_t \leftarrow \tfrac{1}{t} \left(\epsilon + \sum_{i=1}^{t-1} \langle g_i, w_i \rangle \right) \sum_{i=1}^{t-1} g_i$}
\STATE{Receive reward vector $g_t \in \H$ such that $\norm{g_t} \le 1$}
\ENDFOR
}
\end{algorithmic}
\end{algorithm}

The resulting algorithm is stated as
Algorithm~\ref{algorithm:kt-hilbert-space-olo}.  We derive a regret bound for
it as a very simple corollary of
Theorem~\ref{theorem:hilbert-space-olo-regret-bound} to the KT potential
\eqref{equation:kt-estimator-potential}. The only technical part of the proof,
in Appendix~\ref{section:corollaries_reductions}, is an upper bound on $F_t^*$
since it cannot be expressed as an elementary function.
%
\begin{corollary}[Regret Bound for Algorithm~\ref{algorithm:kt-hilbert-space-olo}]
\label{corollary:kt-hilbert-space-olo-regret} Let $\epsilon > 0$. Let
$\{g_t\}_{t=1}^\infty$ be any sequence of reward vectors in a Hilbert space
$\H$ such that $\norm{g_t} \le 1$.
Algorithm~\ref{algorithm:kt-hilbert-space-olo} satisfies
\[
\forall \, T \ge 0 \quad
\forall u \in \H \qquad \qquad
\Regret_T(u) \le \norm{u} \sqrt{T \ln\left(1 + \tfrac{4T^2 \norm{u}^2}{\epsilon^2} \right)} + \epsilon \left(1 - \tfrac{1}{2\sqrt{T}} \right) \;.
\]
\end{corollary}
%
It is worth noting the extreme simplicity of
Algorithm~\ref{algorithm:kt-hilbert-space-olo} and contrast it with the
algorithms in \cite{Streeter-McMahan-2012, McMahan-Orabona-2014, Orabona-2013,
Orabona-2014}.  Also, the regret bound is
optimal~\cite{Streeter-McMahan-2012,Orabona-2013}.  The parameter $\epsilon$
can be safely set to any constant, e.g. $1$. Its role is equivalent to the
initial guess used in doubling tricks.

\subsection{Learning with Expert Advice}
\label{section:kt-lea}

We will now construct an algorithm for \ac{LEA} based on the $\delta$-shifted KT
potential. We set $\delta$ to $T/2$, requiring to know
the number of rounds $T$ in advance.

\begin{algorithm}[t]
\begin{algorithmic}[1]
\caption{Algorithm for Learning with Expert Advice based on $\delta$-shifted KT potential
\label{algorithm:kt-experts}}
{
\REQUIRE{Number of experts $N$, prior distribution $\pi \in \Delta_N$, number of rounds $T$}
\FOR{$t=1,2,\dots,T$}
\STATE{For each $i \in [N]$, set $w_{t,i} \leftarrow \tfrac{\sum_{j=1}^{t-1} \widetilde g_{j,i}}{t+T/2} \left(1 + \sum_{j=1}^{t-1} \widetilde g_{j,i} w_{j,i} \right)$}
\STATE{For each $i \in [N]$, set $\widehat{p}_{t,i} \leftarrow \pi_i [w_{t,i}]_+$}
\STATE{Predict with $p_t \leftarrow
\begin{cases}
\widehat{p}_t/\norm{\widehat{p_t}}_1 & \text{if $\norm{\widehat p_t}_1 > 0$} \\
\pi & \text{if $\norm{\widehat p_t}_1 = 0$}
\end{cases}$}
\STATE{Receive reward vector $g_t \in [0,1]^N$}
\STATE{For each $i \in [N]$, set $\widetilde g_{t,i} \leftarrow \begin{cases}
g_{t,i} - \langle g_t, p_t \rangle & \text{if $w_{t,i} > 0$} \\
[g_{t,i} - \langle g_t, p_t \rangle]_+ & \text{if $w_{t,i} \le 0$}
\end{cases}$}
\ENDFOR
}
\end{algorithmic}
\end{algorithm}

To use the construction in Section~\ref{section:reduction-experts}, we need an
OLO algorithm for the 1-d Hilbert space $\R$.  Using the $\delta$-shited KT
potentials, the algorithm predicts for any sequence $\{\widetilde
g_t\}_{t=1}^\infty$ of reward
\[
w_t
= \beta_t \Wealth_{t-1}
= \beta_t \left(1 + \sum_{j=1}^{t-1} \widetilde g_j w_j \right)
= \frac{\sum_{i=1}^{t-1} \widetilde g_i}{T/2+t} \left(1 + \sum_{j=1}^{t-1} \widetilde g_j w_j \right) \; .
\]
Then, following the construction in Section~\ref{section:reduction-experts}, we
arrive at the final algorithm, Algorithm~\ref{algorithm:kt-experts}.
We can derive a regret bound for Algorithm~\ref{algorithm:kt-experts} by
applying Theorem~\ref{theorem:regret-bound-experts} to the $\delta$-shifted KT
potential.
%
\begin{corollary}[Regret Bound for
Algorithm~\ref{algorithm:kt-experts}] \label{corollary:kt-experts-regret} Let
$N \ge 2$ and $T \ge 0$ be integers. Let $\pi \in \Delta_N$ be a prior.  For
any $g_1, g_2, \dots, g_T \in [0,1]^N$ of reward vectors,
Algorithm~\ref{algorithm:kt-experts} with input $N,\pi,T$ satisfies
\[
\forall u \in \Delta_N \qquad \qquad \Regret_T(u) \le \sqrt{3T (4 + \KL{u}{\pi})} \; .
\]
\end{corollary}
%
Hence, the Algorithm~\ref{algorithm:kt-experts} has \emph{both} the best known
guarantee on worst-case regret and per-round time complexity, see
Table~\ref{table:bounds}. Also, it has the advantage of being very simple.
%As observed by \citet{Chernov-Vovk-2010}, bounds in terms of the KL divergence
%are superior to the $\epsilon$-quantile bounds.

The proof of the corollary is in the
Appendix~\ref{section:corollaries_reductions}.  The technical part of the proof
is an upper bound on $f_t^{-1}(x)$, which we conveniently do by lower bounding
$F_t(x)$.

The reason for using the shifted potential comes from the analysis of
$f_t^{-1}(x)$. The unshifted algorithm would have a $O(\sqrt{T (\log T +
\KL{u}{\pi}})$ regret bound; the shifting improves the bound to $O(\sqrt{T (1 +
\KL{u}{\pi}})$.  By changing $T/2$ in Algorithm~\ref{algorithm:kt-experts} to
another constant fraction of $T$, it is possible to trade-off between the two
constants $3$ and $4$ present in the square root.

The requirement of knowing the number of rounds $T$ in advance can be lifted by
the standard doubling trick~\cite[Section 2.3.1]{Shalev-Shwartz-2011},
obtaining an anytime guarantee with a bigger leading constant,
\[
\forall \, T \ge 0 \quad \forall u \in \Delta_N \qquad \qquad
\Regret_T(u) \le \tfrac{\sqrt{2}}{\sqrt{2} - 1} \sqrt{3T (4 + \KL{u}{\pi})} \; .
\]
