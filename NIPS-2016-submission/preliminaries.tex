\section{Preliminaries}
\label{sec:prel}

The Kullback-Leibler divergence between two discrete distributions $p$ and $q$
is $\KL{p}{q} = \sum_{i} p_i \ln\left( p_i/q_i \right)$. Abusing
notation, if $p,q$ are real numbers in $[0,1]$, we denote by $\KL{p}{q} = p \ln
\left(p/q \right) + (1-p) \ln \left((1-p)/(1-q) \right)$ the
Kullback-Leibler divergence between two Bernoulli distributions with parameters
$p$ and $q$.  We denote by $\H$ a Hilbert space with inner product $\langle
\cdot, \cdot\rangle$ and by $\norm{\cdot}$ the induced norm.  We denote by
$\norm{\cdot}_1$ the $1$-norm in $\R^N$.  A function $F:I \to \R_+$ is called
\emph{logarithmically convex} iff $f(x) = \ln(F(x))$ is convex.
%It follows that
%if $F(x)$ is logarithmically convex, then it is also convex (in the usual
%sense).
Let $f:V \to \R \cup \{\pm\infty\}$ be a function defined on a real
vector space $V$. Fenchel conjugate of $f$ is $f^*:V^* \to \R \cup \{\pm
\infty\}$ defined on the dual vector space $V^*$ by $f^*(\theta) = \sup_{x \in
V} \ \langle \theta, x \rangle - f(x)$.  A function $f:V \to \R \cup \{+
\infty\}$ is said to be proper if  it attains finite
value at at least one point. If $f$ is a proper lower semi-continuous convex
function then $f^*$ is also is proper lower semi-continuous convex and
$f^{**}=f$.

\textbf{Binary and Continuous Coin Betting.} We consider a gambler making
repeated bets on the outcomes of adversarial coin flips. The gambler starts with an
initial endowment $\epsilon > 0$. In each round $t$, he bets on an outcome of a
coin flip $g_t \in \{-1,1\}$ where $+1$ denotes heads and $-1$ denotes tails.
We do not make any assumption on how $g_t$ is generated, that is, it can be
chosen by an adversary.

The gambler can bet any amount on either heads or tails. However, he is not
allowed to borrow any additional money. If he loses (i.e. he bets on the
incorrect outcome), he loses the betted amount. If he wins (i.e. he bets on the
correct outcome), he gets the betted amount back and, in addition to that, he
gets the same amount as a reward.  We encode gambler's bet in round $t$ by a
single number $\beta_t \in [-1,1]$. The sign of $\beta_t$ encodes whether he is
betting on heads or tails. The absolute value encodes the betted amount as the
fraction of his current wealth.

Let $\Wealth_t$ be gambler's wealth at the end of round $t$. It satisfies the
recurrence
\begin{align}
\label{equation:wealth-recurrence}
\Wealth_0 & = \epsilon &
& \text{and} &
\Wealth_t & = (1 + g_t \beta_t) \Wealth_{t-1} \qquad \text{for $t \ge 1$} \; .
\end{align}
Note that since $\beta_t \in [-1,1]$, gambler's wealth stays always non-negative.
Gambler's net reward (difference of wealth and initial endowment) after $t$
rounds is
\begin{align}
\label{equation:reward-wealth}
\Reward_t = \Wealth_t - \ \epsilon \; .
\end{align}
We generalize the problem slightly by allowing the outcome of the coin flip
$g_t$ to be any real number in the interval $[-1,1]$.
%In this case we talk
%about \emph{betting on a continuous coin}.
Equations
\eqref{equation:wealth-recurrence} and \eqref{equation:reward-wealth} defining
wealth and reward remain exactly the same.
