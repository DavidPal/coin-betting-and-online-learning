\section{Proof of Theorem~\ref{theorem:regret-bound-experts}}
\label{sec:appendix-erpert-reduction}

\begin{proof}
We first prove that $\sum_{i=1}^N \pi_i g_{t,i} w_{t,i} \le 0$. Indeed,
\begin{align*}
\sum_{i=1}^N \pi_i g_{t,i} w_{t,i}
& = \sum_{i \, : \, \pi_i w_{t,i} > 0} \pi_i [w_{t,i}]_+ (\langle \ell_t, p_t \rangle - \ell_{t,i})  \ + \ \sum_{i \, : \, \pi_i w_{t,i} \le 0} \pi_i w_{t,i} [\langle p_t, \ell_t\rangle - \ell_{t,i}]_+ \\
& = \norm{\widehat p_t}_1 \sum_{i=1}^N p_{t,i} (\langle \ell_t, p_t \rangle - \ell_{t,i})  \ + \ \sum_{i \, : \, \pi_i w_{t,i} \le 0} \pi_i w_{t,i} [\langle p_t, \ell_t\rangle - \ell_{t,i}]_+ \\
& = 0 \ + \ \sum_{i \, : \, \pi_i w_{t,i} \le 0} \pi_i w_{t,i} [\langle p_t, \ell_t\rangle - \ell_{t,i}]_+
\ \le 0 \; .
\end{align*}
The first equality follows from definition of $g_{t,i}$. To see the second equality,
consider two cases: If $\pi_i w_{t,i} \le 0$ for all $i$ then $\norm{\widehat p_t}_1 = 0$ and $p_t$ is the uniform distribution;
therefore $\norm{\widehat p_t}_1 \sum_{i=1}^N p_{t,i} (\langle \ell_t, p_t \rangle - \ell_{t,i}) = 0$ and
$\sum_{i \, : \, \pi_i w_{t,i} > 0} \pi_i [w_{t,i}]_+ (\langle \ell_t, p_t \rangle - \ell_{t,i}) = 0$.
If $\norm{\widehat p_t}_1 > 0$ then $\pi_i [w_{t,i}]_+ = \widehat p_{t,i} = \norm{\widehat p_t}_1 p_{t,i}$ for all $i$.

Inequality $\sum_{i=1}^N \pi_i g_{t,i} w_{t,i} \le 0$ and the assumption
\eqref{equation:experts-one-dimensional-assumption} imply
\begin{equation}
\label{equation:bounded-potential}
\sum_{i=1}^N  \pi_i F_t \left(\sum_{t=1}^T g_{t,i} \right) \le 1 + \sum_{i=1}^N \pi_i \sum_{t=1}^T  g_{t,i} w_{t,i} \le 1 \; .
\end{equation}
Now, let $G_{t,i} =
\sum_{t=1}^T g_{t,i}$. For any competitor $u \in \Delta_N$,
\begingroup
\allowdisplaybreaks
\begin{align*}
\allowdisplaybreaks
&\Regret_T(u)
= \sum_{t=1}^T \langle \ell_t, p_t - u \rangle
= \sum_{t=1}^T \sum_{i=1}^N u_i \left( \langle \ell_t, p_t \rangle - \ell_{t,i} \right) \\
& \le \sum_{t=1}^T \sum_{i=1}^N u_i g_{t,i} \qquad \text{(by definition of $g_{t,i}$)} \\
& \le \sum_{i=1}^N u_i |G_{T,i}| \qquad \text{(since $u \ge 0$)}  \\
& = \sum_{i=1}^N u_i f_T^{-1}\left(\ln [F_T(G_{T,i})] \right)  \qquad \text{(since $F_T(x) = \exp(f_T(x))$ is even)} \\
& \le f_T^{-1}\left(\sum_{i=1}^N u_i \ln \left[ F_T(G_{T,i}) \right]\right) \qquad \text{(by concavity of $f_T^{-1}$)} \\
& = f_T^{-1}\left(\sum_{i=1}^N u_i \left\{\ln \left[\frac{u_i}{\pi_i}\right] +\ln \left[ \frac{\pi_i}{u_i} F_T(G_{T,i}) \right] \right\} \right)
= f_T^{-1}\left(\KL{u}{\pi}+\sum_{i=1}^N u_i\ln \left[\frac{\pi_i}{u_i} F_T(G_{T,i}) \right]\right) \\
& \le f_T^{-1}\left(\KL{u}{\pi}+\ln \left(\sum_{i=1}^N \pi_i F_T(G_{T,i}) \right)\right) \qquad \text{(by concavity of $\ln(\cdot)$)} \\
& \le f_T^{-1}\left(\KL{u}{\pi}\right) \qquad \text{(by \eqref{equation:bounded-potential})}.
\end{align*}
\endgroup
The chain of inequalities above is based on a simple modification of the change
of measure lemma used in the PAC-Bayes literature; see for
example~\citet{McAllester13}.
\end{proof}
