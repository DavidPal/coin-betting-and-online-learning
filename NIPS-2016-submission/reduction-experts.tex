\section{From Coin Betting to Learning with Expert Advice}
\label{section:reduction-experts}

In this section, we show how to use the algorithm for OLO over one-dimensional
Hilbert space $\R$ from
Section~\ref{section:one-dimensional-hilbert-space-olo}--which is itself based
on a coin-betting strategy---to construct an algorithm for \ac{LEA}.

Let $N \ge 2$ be the number of experts and $\Delta_N$ be the $N$-dimensional
probability simplex. Let $\pi = (\pi_1, \pi_2, \dots, \pi_N) \in \Delta_N$ be
any \emph{prior} distribution. Let $A$ be an algorithm for OLO over
one-dimensional Hilbert space $\R$, based on coin-betting potential 
$\{F_t\}_{t=0}^\infty$ with initial endowment\footnote{Any initial endowment $\epsilon > 0$ can be
rescaled to $1$. Instead of $F_t(x)$ we would use $F_t(x)/\epsilon$. The
prediction $w_t$ would become $w_t/\epsilon$. The key observation is that $p_t$
is invariant to scaling of $w_t$, hence the resulting LEA algorithm would be
the same regardless of $\epsilon$.} $1$. We instantiate $N$ copies of $A$.

Consider any round $t$. Let $w_{t,i} \in \R$ be the prediction of $i$-th copy of
$A$. The LEA algorithm computes $\widehat p_t = (\widehat p_{t,1}, \widehat
p_{t,2}, \dots, \widehat p_{t,N}) \in \R_{0,+}^N$,
\begin{equation}
\label{eq:phat}
\widehat p_{t,i} = \pi_i \cdot [w_{t,i}]_+
\end{equation}
where $[x]_+ = \max\{0,x\}$ is the positive part of $x$. Then, the LEA
algorithm predicts $p_t = (p_{t,1}, p_{t,2}, \dots, p_{t,N}) \in \Delta^N$,
\begin{equation}
\label{eq:preds_experts}
p_t = \frac{\widehat p_t}{\norm{\widehat p_t}_1} \; .
\end{equation}
If $\norm{\widehat p_t}_1 = 0$, the algorithm predicts the prior $\pi$.
Then, the algorithm receives the reward vector
$g_t = (g_{t,1}, g_{t,2}, \dots, g_{t,N}) \in [0,1]^N$. Finally, it
feeds the reward to each copy of $A$. The reward for $i$-th copy is $\widetilde g_{t,i} \in
[-1,1]$ defined as
\begin{align}
\label{eq:gradients_experts_reduction}
\widetilde g_{t,i} =
\begin{cases}
g_{t,i} - \langle g_t, p_t \rangle & \text{if } w_{t,i} > 0 \; , \\
\left[g_{t,i} - \langle g_t, p_t \rangle \right]_+ & \text{if } w_{t,i} \le 0 \; .
\end{cases}
\end{align}

The construction above defines a \ac{LEA} algorithm defined by the predictions $p_t$, based on the algorithm $A$.
We can prove the following regret bound for it.
%
\begin{theorem}[Regret Bound for Experts]
\label{theorem:regret-bound-experts}
Let $A$ be an algorithm for \ac{OLO} over
one-dimensional Hilbert space $\R$, based on coin-betting potential 
$\{F_t\}_{t=0}^\infty$ for initial
endowment $1$. Let $f_t^{-1}$ be in the inverse of $f_t(x) = \ln(F_t(x))$
restricted to $[0,\infty)$.  Then, the regret of the \ac{LEA}
algorithm with prior $\pi \in \Delta_N$ that predicts at each round with $p_t$ in \eqref{eq:preds_experts} satisfies
\[
\forall T \ge 0 \quad \forall u \in \Delta_N \qquad \qquad
\Regret_T(u) \le f_T^{-1}\left( \KL{u}{\pi} \right) \; .
\]
\end{theorem}
The proof, in Appendix~\ref{section:appendix-expert-reduction}, 
is based on the fact that \eqref{eq:phat}--\eqref{eq:gradients_experts_reduction} guarantee that $\sum_{i=1}^N \pi_i \widetilde g_{t,i} w_{t,i} \le 0$ and on a modification of the change
of measure lemma used in the PAC-Bayes literature, e.g.~\cite{McAllester-2013}.

