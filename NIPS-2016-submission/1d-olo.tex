\section{Warm-Up: One-Dimensional Hilbert Space}
\label{section:one-dimensional-hilbert-space-olo}

There are many frameworks for the design and analysis of online learning
algorithms, e.g. the potential function view~\cite{Cesa-Bianchi-Lugosi-2006},
the regularizer view~\cite{Shalev-Shwartz-2011}, relax and
randomize~\cite{Rakhlin-Shamir-Sridharan-2012}. However, all of them do not
provide much help on how to craft the potential, regularizer, or relaxation.
Here, we show how the betting view gives a very natural intuition on online
learning problems and it also provides a clear path to the tools needed to
solve them.

As a warm-up, let us consider an algorithm for OLO over one-dimensional Hilbert
space $\R$.  Let $\{w_t\}_{t=1}^\infty$ be its sequence of predictions on a
sequence of rewards $\{g_t\}_{t=1}^\infty$, $g_t \in [-1,1]$. The total reward
of the algorithm after $t$ rounds is
\[
\Reward_t = \sum_{i=1}^t g_i w_i \; .
\]
Let us define ``wealth'' of the OLO algorithm as $\Wealth_t = \epsilon +
\Reward_t$, in accordance with \eqref{equation:reward-wealth}.

Intuitively, we want the reward to be big, on any sequence of $g_t$, so that
the regret will be small. We now restrict our attention to algorithms whose
predictions are of the form of a bet:
\begin{equation}
\label{equation:one-dimensional-olo}
w_t = \beta_t \Wealth_{t-1}
= \beta_t \left(\epsilon+ \sum_{i=1}^{t-1} g_i w_i \right),
\end{equation}
where $\beta_t \in [-1,1]$. In this way the recurrence
\eqref{equation:wealth-recurrence} holds. We will see that this restriction
does not prevent us to obtain parameter-free algorithms with optimal bounds.
If we have a coin-betting algorithm that, on a sequence of coin flips
$\{g_t\}_{t=1}^\infty$, $g_t \in [-1,1]$, bets fractions $\beta_t \in [-1,1]$,
we can use it to construct an OLO algorithm in a one-dimensional Hilbert space
$\R$ according to \eqref{equation:one-dimensional-olo}.

Assume now that the betting algorithm at hand guarantees that its wealth is at
least $F(\sum_{t=1}^T g_t)$ starting from an endowment $\epsilon$, then
\begin{equation}
\label{equation:one-dimensional-olo-reward-lower-bound}
\Reward_T
= \sum_{t=1}^T g_t w_t
= \Wealth_T \ - \ \epsilon \ge F\left(\sum_{t=1}^T g_t \right) \ - \ \epsilon \; .
\end{equation}

We are almost done, we just need to convert a lower bound on the reward to an
upper bound on the regret. This can be done using the following lemma
from~\cite{McMahan-Orabona-2014}.
\begin{lemma}[Reward-Regret relationship~\cite{McMahan-Orabona-2014}]
\label{lemma:reward-regret}
Let $V,V^*$ be a pair of dual vector spaces. Let $F:V \to \R \cup \{+\infty\}$
be a proper convex lower semi-continuous function and let $F^*:V^* \to \R \cup
\{+\infty\}$ be its Fenchel conjugate. Let $w_1, w_2, \dots, w_T \in V$ and
$g_1, g_2, \dots, g_T \in V^*$.  Then,
\[
\underbrace{\sum_{t=1}^T \langle g_t, w_t \rangle}_{\Reward_T} \ge F\left( \sum_{t=1}^T g_t \right) -\epsilon
\qquad \text{is equivalent to} \qquad
\forall u \in V^*, \quad
\underbrace{\sum_{t=1}^T \langle g_t, u - w_t\rangle}_{\Regret_T(u)} \le F^*(u) + \epsilon\; .
\]
\end{lemma}

Applying the lemma, we get a regret upper bound
\[
\forall u \in \H \qquad \qquad
\Regret_T(u) \le F^*(u) \ + \ \epsilon \; .
\]

Hence, we have showed that if we have a betting algorithm that guarantees a
minimum wealth, $F(\sum_{t=1}^T g_t)$, this can be used to design and analyze
the one-dimensional \ac{OLO} case. Moreover, the faster is the growth of the wealth, the
smaller the regret will be.  
Hence, we can re-use optimal methods from the betting/prediction with the logarithmic loss literature that maximize the wealth, e.g. \cite[Section 9]{Cesa-Bianchi-Lugosi-2006}, to get optimal regret minimization algorithms.
In the next sections, we will provide the tools and the reductions to extend the 1-d case to the generic \ac{OLO} case and
to \ac{LEA} as well.
