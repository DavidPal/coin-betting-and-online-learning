\section{From Coin Betting to Learning with Expert Advice}
\label{section:reduction-experts}

We show how to use an algorithm for OLO over one-dimensional Hilbert space $\R$
to construct an algorithm for learning with expert advice.

Let $N \ge 2$ be the number of experts and $\Delta_N$ be the $N$-dimensional
probability simplex. Let $\pi = (\pi_1, \pi_2, \dots, \pi_N) \in \Delta_N$ be
any \emph{prior} distribution. Let $A$ be an algorithm for OLO over
one-dimensional Hilbert space $\R$. We instantiate $N$ copies of $A$.

Consider any round $t$. Let $w_{t,i} \in \R$ be the prediction of $i$-th copy of
$A$. The LEA algorithm computes $\widehat p_t = (\widehat p_{t,1}, \widehat
p_{t,2}, \dots, \widehat p_{t,N}) \in \R_{0,+}^N$,
$$
\widehat p_{t,i} = \pi_i \cdot [w_{t,i}]_+
$$
where $[x]_+ = \max\{0,x\}$ is the positive part of $x$. Then, the LEA
algorithm predicts $p_t = (p_{t,1}, p_{t,2}, \dots, p_{t,N}) \in \Delta^N$,
$$
p_t = \frac{\widehat p_t}{\norm{\widehat p_t}_1} \; .
$$
If $\norm{\widehat p_t}_1 = 0$, the algorithm predicts the uniform distribution
$p_t = \frac{1}{N}(1,1,\dots,1)$. Then, the algorithm receives the loss vector
$\ell_t = (\ell_{t,1}, \ell_{t,2}, \dots, \ell_{t,N}) \in [0,1]^N$. Finally, it
feeds reward to each copy of $A$. The reward for $i$-th copy is $g_{t,i} \in
[-1,1]$ defined as
\begin{align}
\label{eq:gradients_experts_reduction}
g_{t,i} =
\begin{cases}
\langle \ell_t, p_t \rangle - \ell_{t,i} & \text{if } w_{t,i} > 0 \; , \\
\left[\langle \ell_t, p_t \rangle - \ell_{t,i} \right]_+ & \text{if } w_{t,i} \le 0 \; .
\end{cases}
\end{align}

Suppose that the algorithm $A$, for any sequence
$\{g_t\}_{t=1}^\infty$ such that $g_t \in [-1,1]$, satisfies
\begin{equation}
\label{equation:experts-one-dimensional-assumption}
\Wealth_t = 1 + \sum_{i=1}^t g_i w_i \ge F_t\left(\sum_{i=1}^t g_i\right) \; .
\end{equation}
where $\{F_t\}_{t=0}^\infty$ is a sequence of coin-betting potentials for
initial endowment $1$.\footnote{Any initial endowment $\epsilon > 0$ can be
rescaled to $1$. Instead of $F_t(x)$ we would use $F_t(x)/\epsilon$. The
prediction $w_t$ would become $w_t/\epsilon$. The key observation is that $p_t$
is invariant to scaling of $w_t$, hence the resulting LEA algorithm would be the
same regardless of $\epsilon$.}

We show how to convert \eqref{equation:experts-one-dimensional-assumption} into
a regret bound for the experts algorithm. The regret bound is expressed in terms
of the inverse of $f_t(x) = \ln(F_t(x))$. We define it as follows. We
restrict the function $f_t$ to the domain $I_t \cap [0, \infty)$. By the
definition of coin-betting potential, $f_t$ is convex, strictly increasing.
Image of $f_t$ contains $[0,\infty)$, because $F_t(0) \le 1$ and because $F_t$
satisfies \eqref{equation:potential-limit-assumption}. We define $f_t^{-1}$ as
the inverse of $f_t$. Note that $f_t$ is strictly increasing, concave, and its
domain contains $[0, \infty)$.

\begin{theorem}[Regret Bound for Experts]
\label{theorem:regret-bound-experts}
Let $\{F_t\}_{t=0}^\infty$ be a sequence of coin-betting potentials for initial
endowment $1$. Let $f_t^{-1}$ be in the inverse of $f_t(x) = \ln(F_t(x))$.
Let $A$ be an algorithm for OLO over one-dimensional Hilbert space
$\R$. Suppose that $A$ satisfies
\eqref{equation:experts-one-dimensional-assumption} for any sequence
$\{g_t\}_{t=1}^\infty$ such that $g_t \in [-1,1]$. Then, the regret of the
experts algorithm with prior $\pi \in \Delta_N$, based on $A$, satisfies
$$
\forall u \in \Delta_N \qquad \qquad
\Regret_t(u) \le f_t^{-1}\left( \KL{u}{\pi} \right) \; .
$$
\end{theorem}

\begin{proof}
We first prove that $\sum_{i=1}^N \pi_i g_{t,i} w_{t,i} \le 0$. Indeed,
\begin{align*}
\sum_{i=1}^N \pi_i g_{t,i} w_{t,i}
& = \sum_{i \, : \, \pi_i w_{t,i} > 0} \pi_i [w_{t,i}]_+ (\langle \ell_t, p_t \rangle - \ell_{t,i})  \ + \ \sum_{i \, : \, \pi_i w_{t,i} \le 0} \pi_i w_{t,i} [\langle p_t, \ell_t\rangle - \ell_{t,i}]_+ \\
& = \norm{\widehat p_t}_1 \sum_{i=1}^N p_{t,i} (\langle \ell_t, p_t \rangle - \ell_{t,i})  \ + \ \sum_{i \, : \, \pi_i w_{t,i} \le 0} \pi_i w_{t,i} [\langle p_t, \ell_t\rangle - \ell_{t,i}]_+ \\
& = 0 \ + \ \sum_{i \, : \, \pi_i w_{t,i} \le 0} \pi_i w_{t,i} [\langle p_t, \ell_t\rangle - \ell_{t,i}]_+
\ \le 0 \; .
\end{align*}
The first equality follows from definition of $g_{t,i}$. To see the second equality,
consider two cases: If $\pi_i w_{t,i} \le 0$ for all $i$ then $\norm{\widehat p_t}_1 = 0$ and $p_t$ is the uniform distribution;
therefore $\norm{\widehat p_t}_1 \sum_{i=1}^N p_{t,i} (\langle \ell_t, p_t \rangle - \ell_{t,i}) = 0$ and
$\sum_{i \, : \, \pi_i w_{t,i} > 0} \pi_i [w_{t,i}]_+ (\langle \ell_t, p_t \rangle - \ell_{t,i}) = 0$.
If $\norm{\widehat p_t}_1 > 0$ then $\pi_i [w_{t,i}]_+ = \widehat p_{t,i} = \norm{\widehat p_t}_1 p_{t,i}$ for all $i$.

Inequality $\sum_{i=1}^N \pi_i g_{t,i} w_{t,i} \le 0$ and the assumption
\eqref{equation:experts-one-dimensional-assumption} imply
\begin{equation}
\label{equation:bounded-potential}
\sum_{i=1}^N  \pi_i F_t \left(\sum_{t=1}^T g_{t,i} \right) \le 1 + \sum_{i=1}^N \pi_i \sum_{t=1}^T  g_{t,i} w_{t,i} \le 1 \; .
\end{equation}
Now, let $G_{t,i} =
\sum_{t=1}^T g_{t,i}$. For any competitor $u \in \Delta_N$,
\begingroup
\allowdisplaybreaks
\begin{align*}
\allowdisplaybreaks
\Regret_T(u)
& = \sum_{t=1}^T \langle \ell_t, p_t - u \rangle
= \sum_{t=1}^T \sum_{i=1}^N u_i \left( \langle \ell_t, p_t \rangle - \ell_{t,i} \right) \\
& \le \sum_{t=1}^T \sum_{i=1}^N u_i g_{t,i} & \text{(by definition of $g_{t,i}$)} \\
& \le \sum_{i=1}^N u_i |G_{T,i}| & \text{(since $u \ge 0$)}  \\
& = \sum_{i=1}^N u_i f_T^{-1}\left(\ln [F_T(G_{T,i})] \right)  & \text{(since $F_T(x) = \exp(f_T(x))$ is even)} \\
& \le f_T^{-1}\left(\sum_{i=1}^N u_i \ln \left[ F_T(G_{T,i}) \right]\right) & \text{(by concavity of $f_T$)} \\
& = f_T^{-1}\left(\sum_{i=1}^N u_i \left\{\ln \left[ \frac{\pi_i}{u_i} F_T(G_{T,i}) \right] + \ln \left[\frac{u_i}{\pi_i}\right] \right\} \right) \\
& = f_T^{-1}\left(\sum_{i=1}^N u_i \left\{\ln \left[\frac{\pi_i}{u_i} F_T(G_{T,i}) \right]\right\} + \KL{u}{\pi}\right) \\
& \le f_T^{-1}\left(\ln \left(\sum_{i=1}^N \pi_i F_T(G_{T,i}) \right) + \KL{u}{\pi}\right) & \text{(by concavity of $\ln(\cdot)$)} \\
& \le f_T^{-1}\left(\KL{u}{\pi}\right) & \text{(by \eqref{equation:bounded-potential})}.
\end{align*}
\endgroup
The chain of inequalities above is based on a simple modification of the change
of measure lemma used in the PAC-Bayes literature; see for
example~\citet{McAllester13}.
\end{proof}
