\section{Appendix}

\subsection{Optimal Betting Fraction}
\begin{theorem}
\label{theo:opt_beta}
Assume that the function $\phi(z)=F_t(x+z)$, is piece-wise logarithmically convex in $[-\infty,0)$ and $(0,\infty]$. Then, for any $z\in \R$, we have
\[
\argmin_{\beta} \max_{|g_t|\leq 1} \ \frac{F_t(x+g_t)}{1 + \beta g_t} =\frac{F_t(x+1) - F_t(x-1)}{F_t(x+1) + F_t(x-1)}~.
\]
\end{theorem}
\begin{proof}
First consider the case that $g_t=0$. In this case, $\frac{F_t(x+g_t)}{1 + \beta g_t}=F_t(x)$, independently of $\beta$. Hence, in the following we can exclude the case $g_t=0$.

Minimizing $\frac{\phi(z)}{1+\beta z}$ is equivalent to minimize the function $g(z,\beta)$ defined as
\[
g(z,\beta) := \ln \phi(z) - \log(1+\beta z)~.
\]
We have that $g(z,\beta)$ is piece-wise convex on $[-\infty,0)$ and $(0,\infty]$, w.r.t. the first argument.
Hence, the maximum of $g(z,\beta)$ is attained in $z=-1$ or $z=1$, given that the case $z=0$ is escluded by assumption.
Hence, we want to choose $\beta$ as
\[
\argmin_{\beta} \max_{z \in \{-1, 1\}} \ g(z,\beta),
\]
in other words we want to set $\beta$ such that $g(1)=g(-1)$, that is
\[
\beta^* = \frac{\exp\left(\ln \phi(1) - \ln \phi(-1)\right)-1}{\exp\left(\ln \phi(1) - \ln \phi(-1)\right)+1} 
= \frac{\phi(1) - \phi(-1)}{\phi(1) + \phi(-1)}~.
\]
\end{proof}

\subsection{Proof of Lemma~\ref{lemma:recursion_hilbert}}

First we state the following Lemma from~\cite{McMahanO14} and reported here with our notation for completeness.
\begin{lemma}[Extremes]
\label{lemma:extremes}
Let $h:(-a,a) \to \R$ be an even concave twice-differentiable function that
satisfies $x \cdot h''(x) \le h'(x)$ for all $x \in [0,a)$. Then, if vectors
$u,v \in \H$ satisfy $\|u\| + \|v\| < a$, then
\begin{equation}
\label{equation:lemma-extremes-1}
\langle u, v \rangle + h(\|u + v\|) \ge \min \left\{ \|u\| \cdot \|v\| + h(\|v\| + \|u\|), \ - \|u\| \cdot \|v\| + h(\|u\| - \|v\|) \right\} \; .
\end{equation}
Futhermore, for any $b \in [0,a]$ and any $u, v \in \H$ such that $\|v\| < a - b$ and $\|u\| \le b$,
\begin{equation}
\label{equation:lemma-extremes-2}
\langle u, v \rangle + h(\|u + v\|) \ge \min \left\{ b \cdot \|v\| + h(\|v\| + b), \ - b \cdot \|v\| + h(\|u\| - b) \right\} \; .
\end{equation}
\end{lemma}
%
\begin{proof}
If $u$ or $v$ is zero, the inequality \eqref{equation:lemma-extremes-1} clearly holds. From now on we assume that
$u,v$ are non-zero. Let $c$ be the cosine of the angle of between $u$ and $v$.
More formally,
$$
c = \frac{\langle u, v \rangle}{\|u\| \cdot \|v\|} \; .
$$
With this notation, the left-hand side is
$$
\langle u, v \rangle + h(\|u + v\|) = c \|u\| \cdot \|v\|  + h(\sqrt{\|u\|^2 + \|v\|^2 + 2 c \|u\| \cdot \|v\|}) \; .
$$
We consider the last expression as a function of $c$ and we call it $f(c)$. The
inequality \eqref{equation:lemma-extremes-1} is equivalent to
$$
\forall c \in [-1,1] \qquad \qquad f(c) \ge \min \left\{f(+1), f(-1)\right\} \; .
$$
The last inequality is clearly true if $f:[-1,1] \to \R$ is concave. We now
check that $f$ is indeed concave, which we prove by showing that the second
derivative is non-positive. The first derivate of $f$ is
$$
f'(c) = \|u\| \cdot \|v\| + \frac{h'(\sqrt{\|u\|^2 + \|v\|^2 + 2 c \|u\| \cdot \|v\|}) \cdot \|u\| \cdot \|v\|}{\sqrt{\|u\|^2 + \|v\|^2 + 2 c \|u\| \cdot \|v\|}} \; .
$$
The second derivative of $f$ is
\begin{align*}
&f''(c) = \|u\|^2 \cdot \|v\|^2 \times \\
&\qquad \frac{h''(\sqrt{\|u\|^2 + \|v\|^2 + 2 c \|u\| \cdot \|v\|})  - \frac{1}{\sqrt{\|u\|^2 + \|v\|^2 + 2c \|u\| \cdot \|v\|}} \cdot h'(\sqrt{\|u\|^2 + \|v\|^2 + 2 c \|u\| \cdot \|v\|})  }{\|u\|^2 + \|v\|^2 + 2 c \|u\| \cdot \|v\|} \; .
\end{align*}
If we consider $x=\sqrt{\|u\|^2 + \|v\|^2 + 2 c \|u\| \cdot \|v\|}$, the
assumption $x \cdot h''(x) \le h'(x)$ implies that $f''(c)$ is non-positive.
This finishes the proof of the inequality \eqref{equation:lemma-extremes-1}.

Inequality \eqref{equation:lemma-extremes-2}, can be derived from inequality
\eqref{equation:lemma-extremes-1} as follows
\begin{align*}
\langle u, v \rangle + h(\|u + v\|)
& \ge \min_{u \in \H : \|u\| \le b} \min \left\{ \|u\| \cdot \|v\| + h(\|v\| + b), \ - \|u\| \cdot \|v\| + h(\|u\| - \|v\|) \right\} \\
& = \min_{z \in [0,b]} \min \left\{ z \cdot \|v\| + h(\|v\| + z), \ - z \cdot \|v\| + h(\|u\| - z) \right\} \\
& = \min_{z \in [-b,b]} z \cdot \|v\| + h(\|v\| + z) \\
& = \min \left\{ b \cdot \|v\| + h(\|v\| + b), \ - b \cdot \|v\| + h(\|u\| - b) \right\} \; .
\end{align*}
The inequality in the chain is the inequality \eqref{equation:lemma-extremes-1}.
The last equality follows since $g(z) = z \cdot \|v\| + h(\|v\| + z)$
is concave.
\end{proof}

We can now prove Lemma~\ref{lemma:recursion_hilbert}.
\begin{proof}[Proof of Lemma~\ref{lemma:recursion_hilbert}]
\begin{align*}
&\left(1 + \beta_t \frac{\langle g_t, x \rangle}{\|x\|} \right) F_{t-1}(\norm{x}) - F_t(\norm{x + g_t}) \\
&\quad \geq \min_{r \in \{-1,1\}} \left(1 + \beta_t r \norm{g_t} \right) F_{t-1}(\norm{x}) - F_t(\norm{x} + r \norm{g_t}) \\
&\quad \geq 0
\end{align*}
where the first inequality comes from Lemma~\ref{lemma:extremes}, the second one from the property 3 of a coin-betting potential.
\end{proof}

