\section{Properties of Krichevsky-Trofimov Potential}
\label{section:properties-kt-potential}

\begin{lemma}[Analytic Properties of KT potential]
\label{lemma:kt-potential-analytic-properties}
Let $a > 0$. The function $F:(-a,a) \to \R_+$,
$$
F(x) = \Gamma(a+x) \Gamma(a-x)
$$
is even, logarithmically convex, strictly increasing on $[0,a)$, satisfies
$$
\lim_{x \nearrow a} F(x) = \lim_{x \searrow -a} F(x) = + \infty
$$
and
\begin{equation}
\label{equation:gamma-property-4}
\forall x \in (-a,a) \qquad x \cdot F''(x) \ge F'(x) \; .
\end{equation}
\end{lemma}

\begin{proof}
$F(x)$ is obviously even. $\Gamma(z) = \int_0^\infty t^{z-1} e^{-t} dt$ is
defined for any real number $z > 0$. Hence $F$ is defined on the interval
$(-a,a)$. According to Bohr-Mollerup theorem \cite[Theorem 2.1]{Artin-1964},
$\Gamma(x)$ is logarithmically convex on $(0,\infty)$. Hence $F(x)$ is also
logarithimically convex, since $\ln (F(x)) = \ln(\Gamma(a+x)) +
\ln(\Gamma(a-x))$ is a sum of convex functions.

It is well known that $\lim_{z \searrow 0} \Gamma(z) = +\infty$. Thus,
$$
\lim_{x \nearrow a} F(x)
= \lim_{x \nearrow a} \Gamma(a+x) \Gamma(a-x)
= \Gamma(2a) \lim_{x \nearrow a} \Gamma(a-x)
= \Gamma(2a) \lim_{z \searrow 0} \Gamma(z)
= + \infty \; ,
$$
since $\Gamma$ is continuous at $2a$. Because $F(x)$ is even, we also have
$\lim_{x \searrow -a} F(x) = +\infty$.

To show that $F(x)$ is increasing and that it satisfies
\eqref{equation:gamma-property-4}, we write $f(x) = \ln(F(x))$ as a Mclaurin series.
The derivatives of $\ln(\Gamma(z))$ are so called polygamma functions
$$
\psi^{(n)}(z) = \frac{d^{n+1}}{dz^{n+1}} \ln(\Gamma(z))
\qquad \qquad \text{for $z > 0$ and $n=0,1,2,\dots$.}
$$
Polygamma functions have well-known integral representation
$$
\psi^{(n)}(z) = (-1)^{n+1} \int_0^\infty \frac{t^n e^{-zt}}{1 - e^{-t}} dt
\qquad \qquad \text{for $z > 0$ and $n=1,2,\dots$.}
$$
Using polygama functions, we can write the Mclaurin series for $f(x) = \ln(F(x))$ as
$$
f(x) = \ln(F(x)) = \ln(\Gamma(a+x)) + \ln(\Gamma(a-x)) = 2 \ln(\Gamma(a)) + 2 \sum_{\substack{n \ge 2 \\ \text{$n$ even}}} \frac{\psi^{(n-1)}(a) x^n}{n!} \; .
$$
The series converges for $x \in (-a,a)$, since
for even $n \ge 2$, $\psi_{n-1}(a)$ is positive and can be upper bounded as
\begin{align*}
\psi^{(n-1)}(a)
& = \int_0^\infty \frac{t^n e^{-at}}{1 - e^{-t}} dt \\
& = \int_0^1 \frac{t^n e^{-at}}{1 - e^{-t}} dt + \int_1^\infty \frac{t^n e^{-zt}}{1 - e^{-t}} dt \\
& \le \int_0^1 \frac{t^n e^{-at}}{t(1 - 1/e)} dt + \int_1^\infty t^n e^{-at} dt \\
& \le \frac{1}{1 - 1/e} \int_0^\infty t^{n-1} e^{-at} dt + \int_0^\infty t^n e^{-at} dt \\
& = \frac{1}{1 - 1/e} a^{-n} \Gamma(n) + a^{-n-1} \Gamma(n+1) \\
& \le \frac{1}{1 - 1/e} a^{-n} n! (1 + 1/a) \; .
\end{align*}
From the Mclaurin expansion we see that $f(x)$ is increasing on $[0,a)$
since all coefficients are positive (except for zero order term).

Finally, to prove \eqref{equation:gamma-property-4}, note that
for any $x \in (-a,a)$,
$$
f(x) = c_0 + \sum_{n=2}^\infty c_n x^n
$$
where $c_2, c_3, \dots$ are non-negative coefficients. Thus
\begin{align*}
f'(x) & = \sum_{n=2}^\infty n c_n x^{n-1} &
& \text{and} &
f''(x) & = \sum_{n=2}^\infty n (n-1) c_n x^{n-2} \; .
\end{align*}
and hence $x \cdot f''(x) \ge f'(x)$ for $x \in [0,a)$. Since $F(x) = \exp(f(x))$,
\begin{align*}
F'(x) & = f'(x) \cdot F(x) &
& \text{and} &
F''(x) & = \left[f''(x) + (f'(x))^2 \right] \cdot F(x) \; .
\end{align*}
Therefore, for $x \in [0,a)$,
$$
x \cdot F''(x)
= x \left[ f''(x) + (f'(x))^2 \right] F(x)
\ge \left[ f'(x) + x (f'(x))^2 \right] F(x)
\ge f'(x) F(x) = F'(x) \; .
$$
This proves \eqref{equation:gamma-property-4}.
\end{proof}

\begin{theorem}[KT potential]
\label{theorem:kt-potential}
Let $\delta \ge 0$ and $\epsilon > 0$. The sequence of functions
$\{F_t\}_{t=0}^\infty$, $F_t:(-t-\delta-1, t+\delta+1) \to \R_+$ defined by
$$
F_t(x) = \epsilon \frac{2^t \cdot \Gamma(\delta + 1) \Gamma(\frac{t+\delta+1}{2} + \frac{x}{2}) \Gamma(\frac{t+\delta+1}{2} - \frac{x}{2})}{\Gamma(\frac{\delta+1}{2})^2 \Gamma(t+\delta+1)} \; .
$$
is a sequence of excellent coin-betting potentials for initial endowment $\epsilon$.
Furthermore, for any $x \in (-t-\delta-1, t+\delta+1)$,
\begin{equation}
\label{equation:kt-potential-beta}
\frac{F_t(x+1) - F_{t}(x-1)}{F_t(x+1) + F_{t}(x-1)} = \frac{x}{t+\delta} \; .
\end{equation}
\end{theorem}

\begin{proof}
Property (2) and (4) of the definition follow from
Lemma~\ref{lemma:kt-potential-analytic-properties}.
Property (1) follows by simple substitution for $t=0$ and $x=0$.

Before verifying property (3), we prove \eqref{equation:kt-potential-beta}. We
use an algebraic property of the gamma function that states that $\Gamma(1+z) =
z \Gamma(z)$ for any positive $z$. Equation \eqref{equation:kt-potential-beta}
follows from
\begin{align*}
\frac{F_t(x + 1) - F_t(x - 1)}{F_t(x + 1) + F_t(x - 1)}
& = \frac{\Gamma(\frac{t+\delta+2}{2} + \frac{x}{2}) \Gamma(\frac{t+\delta}{2} - \frac{x}{2}) - \Gamma(\frac{t+\delta}{2} + \frac{x}{2}) \Gamma(\frac{t+\delta+2}{2} - \frac{x}{2})}{\Gamma(\frac{t+\delta+2}{2} + \frac{x}{2}) \Gamma(\frac{t + \delta}{2} - \frac{x}{2}) + \Gamma(\frac{t + \delta}{2} + \frac{x}{2}) \Gamma(\frac{t+\delta+2}{2} - \frac{x}{2})} \\
& = \frac{(\frac{t+\delta}{2} + \frac{x}{2})\Gamma(\frac{t+\delta}{2} + \frac{x}{2}) \Gamma(\frac{t+\delta}{2} - \frac{x}{2}) - (\frac{t+\delta}{2} - \frac{x}{2})\Gamma(\frac{t+\delta}{2} + \frac{x}{2}) \Gamma(\frac{t+\delta}{2} - \frac{x}{2})}{(\frac{t+\delta}{2} + \frac{x}{2})\Gamma(\frac{t+\delta}{2} + \frac{x}{2}) \Gamma(\frac{t+\delta}{2} - \frac{x}{2}) + (\frac{t+\delta}{2} - \frac{x}{2})\Gamma(\frac{t+\delta}{2} + \frac{x}{2}) \Gamma(\frac{t+\delta}{2} - \frac{x}{2})} \\
& = \frac{(\frac{t+\delta}{2} + \frac{x}{2}) - (\frac{t+\delta}{2} - \frac{x}{2})}{(\frac{t+\delta}{2} + \frac{x}{2}) + (\frac{t+\delta}{2} - \frac{x}{2})} \\
& = \frac{x}{t+\delta} \; .
\end{align*}

Let $\phi(g) = \frac{F_t(x+g)}{F_{t-1}(x)}$. To verify property (3) of the definition,
we need to show that $\phi(g) \le 1 + g \frac{x}{t+\delta}$
for any $x \in [-t+1, t-1]$ and any $g \in [-1,1]$. We can write $\phi(g)$ as
\begin{align*}
\phi(g)
& = \frac{F_t(x+g)}{F_{t-1}(x)} \\
& = \frac{2 \Gamma(\frac{t+\delta+1}{2} + \frac{x+g}{2}) \Gamma(\frac{t+\delta+1}{2} - \frac{x+g}{2}) \Gamma(t + \delta)}{\Gamma(\frac{t+\delta}{2} + \frac{x}{2}) \Gamma(\frac{t+\delta}{2} - \frac{x}{2}) \Gamma(t+\delta+1)} \\
& = \frac{2}{t+\delta} \cdot \frac{\Gamma(\frac{t+\delta+1}{2} + \frac{x+g}{2}) \Gamma(\frac{t+\delta+1}{2} - \frac{x+g}{2})}{\Gamma(\frac{t+\delta}{2} + \frac{x}{2}) \Gamma(\frac{t+\delta}{2} - \frac{x}{2})} \; .
\end{align*}
For $g=+1$, using the formula $\Gamma(1+z) = z \Gamma(z)$, we have
$$
\phi(+1)
= \frac{2}{t+\delta} \cdot \frac{\Gamma(\frac{t+\delta}{2} + \frac{x}{2} + 1) \Gamma(\frac{t+\delta}{2} - \frac{x}{2})}{\Gamma(\frac{t+\delta}{2} + \frac{x}{2}) \Gamma(\frac{t+\delta}{2} - \frac{x}{2})}
= \frac{2}{t+\delta} \left(\frac{t+\delta}{2} + \frac{x}{2} \right)
= 1 + \frac{x}{t+\delta} \; .
$$
Similarly, for $g=-1$, using the formula $\Gamma(1+z) = z \Gamma(z)$, we have
$$
\phi(-1)
= \frac{2}{t+\delta} \cdot \frac{\Gamma(\frac{t+\delta}{2} + \frac{x}{2}) \Gamma(\frac{t+\delta}{2} - \frac{x}{2} + 1)}{\Gamma(\frac{t+\delta}{2} + \frac{x}{2}) \Gamma(\frac{t+\delta}{2} - \frac{x}{2})}
= \frac{2}{t+\delta} \left(\frac{t+\delta}{2} - \frac{x}{2} \right)
= 1 - \frac{x}{t+\delta} \; .
$$
We can write any $g \in [-1,1]$ as a convex combination of $-1$ and $+1$, i.e.,
$g = \lambda \cdot (-1) + (1-\lambda) \cdot (+1)$ for some $\lambda \in [0,1]$.
Since $\phi(g)$ is (logaritmically) convex,
\begin{align*}
\phi(g)
& = \phi(\lambda \cdot (-1) + (1-\lambda) \cdot (+1)) \\
& \le \lambda \phi(-1) + (1-\lambda) \phi(+1) \\
& = \lambda \left(1 + \frac{x}{t+\delta}\right) + (1-\lambda) \left(1 - \frac{x}{t+\delta}\right) \\
& = 1 + g \frac{x}{t+\delta} \; .
\end{align*}
\end{proof}
