\section{Introduction}
\label{section:introduction}

We consider the standard \ac{OLO}~\citep{Cesa-Bianchi-Lugosi-2006,
Shalev-Shwartz-2011} setting. In each round $t$, an algorithm chooses a point
$x_t$ from a convex \emph{decision set} $K$ and then receives a loss vector
$\ell_t$. Algorithm's goal is to keep its \emph{regret} small, defined as the
difference between its cumulative loss and the cumulative loss of a fixed
strategy $u \in K$, that is
$$
\Regret_T(u) = \sum_{t=1}^T \langle \ell_t, x_t \rangle - \sum_{t=1}^T \langle \ell_t, u \rangle \; .
$$

We focus on two particular sets, the $N$-dimensional probability simplex
$\Delta_N = \{ x \in \R^N ~:~ x \ge 0, \norm{x}_1 = 1\}$ and the Hilbert space
$\H$.  \ac{OLO} over $\Delta_N$ is referred to as the problem of \ac{LEA}.  We
assume bounds on the norms of the loss vectors: For \ac{OLO} over $\H$, we
assume that $\norm{\ell_t}_2 \le 1$, and for \ac{LEA} we assume that
$\ell_t \in [0,1]^N$.

\ac{OLO} is a basic building block of many machine learning problems. For
example, \ac{OCO}, the analogous problem where $\langle \ell_t, w_t \rangle$ is
generalized to $\ell_t(w_t)$ where $\ell_t$ is an arbitrary convex function, is
solved through a reduction to an \ac{OLO}~\citep{Shalev-Shwartz-2011}.
\ac{LEA}~\citep{Littlestone-Warmuth-1994, Vovk-1998,
Cesa-Bianchi-Freund-Haussler-Helmbold-Schapire-Warmuth-1997} provides a way of
combining classifiers and it at the heart of
boosting~\cite{Freund-Schapire-1997}. Batch and stochastic convex optimization
can be solved through a reduction to \ac{OLO}~\citep{Shalev-Shwartz-2011}.
Statistical learning with convex losses can also seen as stochastic convex
optimization and solved through \ac{OCO}~\citep{Munro-1951}.

However, as essential as it is to achieve sublinear regret, this is only half
of the problem in online learning. In fact, we are often interested in the
adaptation to the (often unknown) characteristics of the data. Most of the
time, online and batch learning algorithms fail on this side, requiring to set
hyperparameters (e.g., learning rates, step sizes, and regularization weights)
to oracle choices in order to achieve the best possible theoretical and
empirical performance. Recently, a new family of algorithms that adapt to the
data has been proposed, both for \ac{LEA}~\citep{Chaudhuri-Freund-Hsu-2009,
Chernov-Vovk-2010, Luo-Schapire-2014, Luo-Schapire-2015, Koolen-van-Erven-2015}
and for \ac{OLO}/\ac{OCO} over Hilbert spaces~\citep{Streeter-McMahan-2012,
Orabona-2013, McMahan-Abernethy-2013, McMahan-Orabona-2014, Orabona-2014}.
These algorithms adapt to the number of experts and to the norm of the optimal
predictor, respectively, without the need to tune parameters. Given the
connections between \ac{OLO}/\ac{LEA} and machine learning, these algorithms
allow to design parameter-free batch machine learning algorithms through
straightforward reductions~\citep{Orabona-2014,Luo-Schapire-2015}.
Surprisingly enough, these two families of algorithms are also very similar,
yet no attempt has been made to unify them.

Our contributions are as follows. We claim that a more fundamental notion
subsumes both \ac{OLO} and \ac{LEA} parameter-free algorithms. This notion is
linked to the ability of an algorithm to repeatedly bet on an outcome of a coin
flip (Section~\ref{section:coin-betting-potentials}). In fact, we show black
box reductions from the coin betting scenario to \ac{OLO} over Hilbert spaces
and to \ac{LEA} (Sections~\ref{section:reduction_hilbert}
and~\ref{section:reduction-experts}).  We prove that coin betting strategies
that assure an exponential growth of the wealth for biased coins allow to
obtain optimal worst-case parameter-free regret bounds in \ac{OLO} and in
\ac{LEA}.

\noindent\textbf{Notation and Definitions.} We will use the following notation
in the rest of this note. We denote by $\indicator$ the vector $(1,1,\dots,1)
\in \R^N$. Shannon entropy $H(u) = -\sum_{i=1}^N u_i \ln u_i$ is defined for
any $u \in \Delta_N$.  The Kullback-Leibler divergence $\KL{u}{v} =
\sum_{i=1}^N u_i \ln(u_i/v_i)$ is defined for any $u,v \in \Delta_N$. If $\H$
is a real Hilbert space, we denote by $\langle \cdot, \cdot \rangle$ its inner
product, and by $\norm{\cdot}$ the induced norm.
