\section{From Online Learning to Convex Optimization and Machine Learning}
\label{section:applications}

The results in the previous sections immediately implies new algorithms and
results in convex optimization and machine learning.

\textbf{Convex Optimization.} It is immediate to transform
Algorithm~\ref{algorithm:hilbert-space-olo} into a \ac{SGD} algorithm.  For
example, consider the popular setting of minimizing an empirical risk of the
form
\begin{equation}
\label{equation:objective-function}
\widehat{F}(w) = \frac{1}{N} \sum_{i=1}^N \ell_t(w),
\end{equation}
where $\ell$ is convex and $w \in \R^d$. Without loss of generality, we also
assume that the norm of the sub-gradient of $\ell$ is bounded by 1. Also,
define $\widehat{w}=  \argmin_w \widehat{F}(w)$.  The corresponding reduction to OLO
gives rise\footnote{For simplicity, we show the algorithm and its convergence
guarantee in $\R^d$, but it can be implemented with kernels as well.} to the
Algorithm~\ref{algorithm:kt-sgd}.

\begin{algorithm}[t]
\caption{SGD algorithm based on KT potential \label{algorithm:kt-sgd}}
\begin{algorithmic}
{
\STATE{Initialize $m_0 \leftarrow 1$}
\FOR{$t=1,2,\dots,T$}
\STATE{Set $w_t \leftarrow m_{t-1} \tfrac{\theta_{t-1}}{t} $}
\STATE{Select a $\ell_t$ at random and receive $g_t \in \partial \ell_t(w_{t-1})$}
\STATE{Update $\theta_t = \theta_{t-1}-g_t$}
\STATE{Update $m_t = m_{t-1}-g_t^\top w_t$}
\ENDFOR
\STATE{Output $\overline{w}_T=\tfrac{1}{T}\sum_{t=1}^T w_t$}
}
\end{algorithmic}
\end{algorithm}

Beside the simplicity of the algorithm, its important property is that it
\emph{does not have a learning rate to be tuned}, yet it achieves optimal
convergence rate w.r.t. the norm of the optimal solution.

\begin{theorem}
After $T$ iterations of Algorithm~\ref{algorithm:kt-sgd}, $\overline{w}_T$ is
an approximate minimizer of the function \eqref{equation:objective-function} in
the sense
$$
\widehat{F}(\overline{w}_T) - \widehat{F}(\widehat{w}) \leq \frac{\norm{\widehat{w}}}{\sqrt{T}} \sqrt{\log(1+4 T^2 \norm{\widehat{w}}^2)} +\tfrac{1}{T} \; .
$$
\end{theorem}
Note that in the above theorem, $T$ can be larger (multiple epochs) or smaller
than $N$ (not using all the samples).

\textbf{Machine Learning.} In a machine learning view, the minimization of a
function \eqref{equation:objective-function} is just a proxy to minimize the
\emph{true risk} over an unknown distribution. For example, $\ell_t(w)$ can be
a regression loss, e.g. the logistic loss, over a sample $(x_t, y_t)$. That is,
$\ell_t(w)=\ell(w,x_t,y_t)$. A very common approach in order to have a small
risk on the test set is to minimize a regularized objective function:
\begin{equation}
\label{eq:reg_logloss}
\widehat{F}^{\text{Reg}}(w) = \lambda \norm{w}^2 + \frac{1}{N} \sum_{i=1}^N \ell(w,x_t,y_t) \; .
\end{equation}
This problem is strongly convex, so there are very efficient methods to
minimize it, we can assume that we are able to get the minimizer of
$\widehat{F}^{\text{Reg}}$ with arbitrary good precision. Yet, this is not enough. In
fact, we are rarely interested in the value of the objective function, rather
we are interested in the \emph{true risk} of a solution $w$, that is
$\Exp[\ell(w,X,Y)]$ where $X,Y$ comes from an unknown distribution from which we
sample training and test points.  Hence, in order to get a good performance we
have to select a good regularization parameter. In particular, from the results
in \cite{???} immediately follows that we have
$$
\Exp[\ell(\widehat{w},X,Y)] - \Exp[\ell(w^*,X,Y)] \le O(\lambda \norm{w^*}^2 + \tfrac{1}{\lambda N}),
$$
where $w^*=\argmin_w \Exp[\ell(w,X,Y)]$.  From the above bound, it is clear
that the optimal value of $\lambda$ depends on the $\norm{w^*}$ that is
unknown. We would like to stress that this is not just a theoretical problem:
Any practictioner knows how painfull it is to find the right regularization for
the problem at hand.  Assuming we know $\norm{w^*}$ (which is in practice
unknown), we can set $\lambda = O(1/(\norm{w^*} \sqrt{N}))$ to achieve
\begin{equation}
\label{equation:optimal-rate}
\Exp[\ell(\widehat{w},X,Y)] - \Exp[\ell(w^*,X,Y)] \le O\left(\tfrac{\norm{w^*}}{\sqrt{N}}\right) \; .
\end{equation}

Using again Algorithm~\ref{algorithm:kt-sgd} with only one pass
over the dataset, we get almost the same guarantee \eqref{equation:optimal-rate}
\emph{without tuning any parameters}, learning rates, regularization parameters, etc.
\begin{theorem}
Shuffle the dataset and do only one pass over the training set, using
$\ell_t(w)=\ell(w,x_t,y_t)$.  Then, we have
$$
\Exp[\ell(\overline{w}_N,X,Y)] - \Exp[\ell(w^*,X,Y)] \le \frac{\norm{w^*}}{\sqrt{N}} \sqrt{\log(1+4 N^2 \norm{w^*}^2)} + \tfrac{1}{N} \; .
$$
\end{theorem}
Comparing this guarantee to the one in \eqref{equation:optimal-rate}, we see
that, just paying a very small logarithmic price, we obtain the optimal
convergence rate and we remove all the parameters, i.e. learning rate and/or
regularization parameters.
