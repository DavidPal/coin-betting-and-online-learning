\section{From Online Learning to Convex Optimization and Machine Learning}
\label{section:applications}

The result in Section~\ref{section:algorithms} immediately implies new
algorithms and results in convex optimization and machine learning. We will state some of them here, see
\cite{Orabona-2014} for more results.

\begin{algorithm}[t]
\caption{SGD algorithm based on KT estimator \label{algorithm:kt-sgd}}
\begin{algorithmic}[1]
{
\REQUIRE{Convex functions $f_1, f_2, \dots, f_N$ and desired number of iterations $T$}
\STATE{Initialize $\Wealth_0 \leftarrow 1$ and $\theta_0 \leftarrow 0$}
\FOR{$t=1,2,\dots,T$}
\STATE{Set $w_t \leftarrow \Wealth_{t-1} \tfrac{\theta_{t-1}}{t} $}
\STATE{Select an index $j$ at random from $\{1,2,\dots,N\}$ and compute $\ell_t = \grad f_j(w_{t-1})$}
\STATE{Update $\theta_t \leftarrow \theta_{t-1} - \ell_t$ and $\Wealth_t \leftarrow \Wealth_{t-1} - \langle \ell_t, w_t \rangle$}
\ENDFOR
\STATE{Output $\overline{w}_T = \tfrac{1}{T}\sum_{t=1}^T w_t$}
}
\end{algorithmic}
\end{algorithm}

\textbf{Convex Optimization.}
Consider an empirical risk minimization problem of the form
%
\begin{equation}
\label{equation:objective-function}
F(w) = \frac{1}{N} \sum_{i=1}^N f_i(w),
\end{equation}
%
where $f_i:\R^d \to \R$ is convex.\footnote{The algorithm can also be
implemented and analyzed with kernels~\citep{Orabona-2014}.} It is immediate to
transform Algorithm~\ref{algorithm:hilbert-space-olo} into a \ac{SGD} algorithm
for this problem, obtaining Algorithm~\ref{algorithm:kt-sgd}.
In Algorithm~\ref{algorithm:kt-sgd}, $\grad f_j(w)$
denotes a subgradient of $f_j$ at a point $w$.  We assume that the
norm of the subgradient of $f_j$ is bounded by $1$.

Beside the simplicity of the Algorithm~\ref{algorithm:kt-sgd}, it has the
important property is that it \emph{does not have a learning rate to be tuned},
yet it achieves the optimal convergence rate. In fact, denoting by $\widehat{w} = \argmin_w F(w)$
the optimal solution of~\eqref{equation:objective-function}, the following
theorem states the rate of convergence of Algorithm~\ref{algorithm:kt-sgd}.
%
\begin{theorem}
The average $\overline{w}_T$ produced by Algorithm~\ref{algorithm:kt-sgd} is
an approximate minimizer of the objective function \eqref{equation:objective-function}:
\[
\Exp\left[F(\overline{w}_T)\right] - F(\widehat{w}) \leq \tfrac{\norm{\widehat{w}}}{\sqrt{T}} \sqrt{\log(1+4 T^2 \norm{\widehat{w}}^2)} +\tfrac{1}{T} \; .
\]
\end{theorem}
%
Note that in the above theorem, $T$ can be larger (multiple epochs) or smaller
than $N$.

\textbf{Machine Learning.} In machine learning, the minimization of a function
\eqref{equation:objective-function} is just a proxy to minimize the \emph{true
risk} over an unknown distribution. For example, $f_i(w)$ can be of the form
$f_i(w) = f(w, X_i, Y_i)$ where $\{(X_i, Y_i)\}_{i=1}^N$ is a sequences of
labeled sampled generated i.i.d. from some \emph{unknown} distribution and
$f(w, X_i, Y_i)$ is the logistic loss of a weight vector $w$ on a sample $(X_i,
Y_i)$.  A common approach to have a small risk on the test set is to minimize a
regularized objective function over the training set:
%
\begin{equation}
\label{equation:reg_logloss}
F_\lambda^{\text{Reg}}(w) = \lambda \norm{w}^2 + \frac{1}{N} \sum_{i=1}^N f(w, X_i, Y_i) \; .
\end{equation}
%
This problem is strongly convex, so there are very efficient methods to
minimize it, hence we can assume to be able to get the minimizer of
$F_\lambda^{\text{Reg}}$ with arbitrary high precision and algorithms that do
not require to tune learning rates. Yet, this is not enough. In fact, we are
rarely interested in the value of the objective function
$F_\lambda^{\text{Reg}}$ or its minimizer, rather we are interested in the
\emph{true risk} of a solution $w$, that is $\Exp[f(w,X,Y)]$, where $(X,Y)$ is
an independent ``test'' sample from the same distribution from which the
training set $\{(X_i,Y_i)\}_{i=1}^N$ came from. Hence, in order to get a good
performance we have to select a good regularization parameter. In particular,
from \cite{Sridharan-Shalev-Shwartz-Srebro-2009} we get
\[
\Exp[f(\widehat{w}_\lambda,X,Y)] - \Exp[f(w^*,X,Y)] \le O(\lambda \norm{w^*}^2 + \tfrac{1}{\lambda N}) \; ,
\]
where $w^*=\argmin_w \Exp[f(w,X,Y)]$ and $\widehat{w}_\lambda = \argmin_w
F_\lambda^{\text{Reg}}(w)$.  From the above bound, it is clear that the optimal
value of $\lambda$ depends on the $\norm{w^*}$ that is unknown.  Yet another
possibility is to select the optimal learning rate and/or the number of epochs
of \ac{SGD} to directly minimize $\Exp[f(w^*,X,Y)]$. However, all these methods
are equivalent~\citep{Lin-Camoriano-Rosasco-2016} and they still require to
tune at least one parameter.
We would like to stress that this is not just a theoretical problem: Any
practitioner knows how painful it is to find the right regularization for the
problem at hand.

Assuming we would know $\norm{w^*}$, we could set $\lambda =
O(1/(\norm{w^*} \sqrt{N}))$ to achieve the worst-case optimal bound
%
\begin{equation}
\label{equation:optimal-rate}
\Exp[f(\widehat{w}_\lambda,X,Y)] - \Exp[f(w^*,X,Y)] \le O\left(\tfrac{\norm{w^*}}{\sqrt{N}}\right) \; .
\end{equation}
However, we can get the same guarantee without knowing $\norm{w^*}$ or the
optimal $\lambda$, by doing a single pass over the data set. More precisely, we
derive Algorithm~\ref{algorithm:averaged-olo} from
Algorithm~\ref{algorithm:hilbert-space-olo} by applying the standard
online-to-batch reduction~\citep{Shalev-Shwartz-2011}.  The algorithm makes
only a single pass over the dataset and it does not have any tuning parameters.
Yet, it has almost the same guarantee \eqref{equation:optimal-rate}
\emph{without knowing $\norm{w^*}$ or the optimal regularization parameter
$\lambda$ or the learning rate, or any other tuning parameter}.

\begin{algorithm}[t]
\caption{Averaging algorithm based on KT estimator \label{algorithm:averaged-olo}}
\begin{algorithmic}[1]
{
\REQUIRE{Sample $(X_1, Y_1), (X_2, Y_2), \dots, (X_N, Y_N)$}
\STATE{Initialize $\Wealth_0 \leftarrow 1$ and $\theta_0 \leftarrow 0$}
\FOR{$i=1,2,\dots,N$}
\STATE{Set $w_i \leftarrow \Wealth_{i-1} \tfrac{\theta_{i-1}}{i} $}
\STATE{Compute $\ell_i = \frac{\partial f(w, X_i, Y_i)}{\partial w}|_{w=w_i}$}
\STATE{Update $\theta_i \leftarrow \theta_{i-1} - \ell_i$ and $\Wealth_i \leftarrow \Wealth_{i-1} - \langle \ell_i, w_i \rangle$}
\ENDFOR
\STATE{Output $\overline{w}_N = \tfrac{1}{N} \sum_{i=1}^N w_i$}
}
\end{algorithmic}
\end{algorithm}
%
\begin{theorem}
Assume that $(X, Y), (X_1, Y_1), (X_2, Y_2), \dots, (X_N,Y_N)$ are i.i.d.  The
output $\overline{w}_N$ of Algorithm~\ref{algorithm:averaged-olo} satisfies
$$
\Exp[f(\overline{w}_N,X,Y)] - \Exp[f(w^*,X,Y)] \le \frac{\norm{w^*}}{\sqrt{N}} \sqrt{\log(1+4 N^2 \norm{w^*}^2)} + \tfrac{1}{N} \; .
$$
\end{theorem}
%
Comparing this guarantee to the one in \eqref{equation:optimal-rate}, we see
that, just paying a sub-logarithmic price, we obtain the optimal convergence
rate and we remove all the parameters.
