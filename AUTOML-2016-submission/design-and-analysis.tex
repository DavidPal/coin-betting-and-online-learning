\section{Design and Analysis of Algorithms}

We explain the intuition behind
algorithms~\ref{algorithm:hilbert-space-olo}~and~\ref{algorithm:experts}.  Both
are based on the adversarial coin-betting problem. We first explain the
adversarial coin betting game. In section~\ref{subsection:connection}, we
explain the connection to
algorithms~\ref{algorithm:hilbert-space-olo}~and~\ref{algorithm:experts}.

Consider a gambler making repeated bets on outcomes of adversarial coin flips.
The gambler starts with an initial endowment $1$ dollar. In each round $t$, he
bets on an outcome of a coin flip $g_t \in \{-1,1\}$ where $+1$ denotes heads
and $-1$ denotes tails. We do not make any assumption on how $g_t$ is
generated, that is, it can be chosen by an adversary.  The gambler can bet any
amount on either heads or tails. However, he is not allowed to borrow any
additional money. If he loses (i.e. he bets on the incorrect outcome), he loses
the betted amount. If he wins (i.e. he bets on the correct outcome), he gets
the betted amount back and, in addition to that, he gets the same amount as a
reward. Note that it does not make sense for the gambler to bet on both
outcomes, since the same winnings can be achieved by betting the difference of
the two amounts on one of the outcomes and betting zero on the other outcome.
We encode gambler's bet in round $t$ by a single number $\beta_t \in (-1,1)$.
The sign of $\beta_t$ encodes whether he is betting on heads or tails. The
absolute value encodes the betted amount as the fraction of his current wealth.

Let $\Wealth_t$ be gambler's wealth at the end of round $t$. It satisfies the
recurrence
\begin{align}
\label{equation:wealth-recurrence}
\Wealth_0 & = 1 &
& \text{and} &
\Wealth_t & = (1 + g_t \beta_t) \Wealth_{t-1} \qquad \text{for $t \ge 1$} \; .
\end{align}
Note that since $\beta_t \in (-1,1)$, gambler's wealth stays always non-negative.

The problem can be generalized slightly by allowing the outcome of the coin
flip $g_t$ to be any real number in the interval $[-1,1]$. The formula
\eqref{equation:wealth-recurrence} for wealth remains exactly the same.

\subsection{Kelly Betting and Krichevsky-Trofimov Estimator}

\citet{Kelly-1956} proposed a general strategy for sequential bets. For coin
betting, the strategy assumes that the coin flips $\{g_t\}_{t=1}^\infty$, $g_t
\in \{+1,-1\}$, are generated i.i.d. with known probability of heads. If $p \in
[0,1]$ is the probability of heads, the Kelly bet is
$$
\beta_t = 2p - 1 \; .
$$
He showed that this strategy maximizes $\Exp[\ln(\Wealth_t)]$.

For adversarial coins, Kelly betting does not make sense. However, we can
replace $p$ with an estimate. This problem was studied by
\citet{Krichevsky-Trofimov-1981}, who proposed that after seeing coin flips
$g_1, g_2, \dots, g_{t-1}$ an empirical estimate $k_t = \frac{1/2 +
\sum_{i=1}^{t-1} \indicator[g_i = +1]}{t}$ should be used instead of $p$. Their
estimate is commonly called \emph{KT estimator}.\footnote{Compared to the
standard maximum likelihood estimate $\frac{\sum_{i=1}^{t-1} \indicator[g_i =
+1]}{t-1}$, KT estimator ``shrinks'' slightly towards $\frac{1}{2}$.} KT
estimator results in betting strategy
\begin{equation}
\label{equation:kt-estimator-betting-strategy}
\beta_t = 2k_t - 1 = \frac{\sum_{i=1}^{t-1} g_i}{t}
\end{equation}
which we call \emph{adaptive Kelly betting based on KT estimator}.
\citeauthor{Krichevsky-Trofimov-1981} showed that
\begin{equation}
\label{equation:wealth-lower-bound}
\forall \beta \in [-1,1] \qquad \qquad \ln(\Wealth_t) \ge \ln(\Wealth_t(\beta)) \ - \ \frac{1}{2} \ln t \ - \ \ln(2) \; ,
\end{equation}
where $\Wealth_t(\beta)$ is the wealth of a strategy that bets the same
fraction $\beta$ in every round. This guarantee is optimal up to constant
additive factors~\citep{Cesa-Bianchi-Lugosi-2006}.

\subsection{Connection to Algorithms~\ref{algorithm:hilbert-space-olo}~and~\ref{algorithm:experts}}
\label{subsection:connection}

In Algorithm~\ref{algorithm:hilbert-space-olo}, the ``coin outcome'' is
$g_t = -\ell_t$ and algorithm's wealth is
$$
\Wealth_t = 1 - \sum_{i=1}^t \langle \ell_i, x_i \rangle \; .
$$
The algorithm explicitly keeps track of its wealth. It bets the ``vectorial fraction''
$$
\beta_t = - \frac{\sum_{i=1}^{t-1} \ell_i}{t} \; .
$$

In Algorithm~\ref{algorithm:experts}, there are $N$ independent coordinates.
With each coordinate $i$ there is an associated coin outcome $g_{t,i}$, wealth
$$
\Wealth_{t,i} = 1 + \sum_{j=1}^t g_{j,i} w_{j,i}
$$
and a betting fraction
$$
\beta_{t,i} = \frac{\sum_{j=1}^t g_{j,i}}{t + T/2} \; .
$$
The regret bounds for both algorithms are a consequence of the wealth lower
bound~\label{equation:wealth-lower-bound}.  For more details, see full version
of the paper~\citep{Orabona-Pal-2016-parameter-free}.
