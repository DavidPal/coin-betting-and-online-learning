\section{Parameter-Free Algorithm From Coin Betting}
\label{section:algorithms}

\begin{algorithm}[t]
\caption{Algorithm for OLO over Hilbert space $\H$ based on Krichevsky-Trofimov estimator
\label{algorithm:hilbert-space-olo}}
\begin{algorithmic}[1]
{
\FOR{$t=1,2,\dots$}
\STATE{Predict with $w_t \leftarrow - \tfrac{1}{t} \left(1 - \sum_{i=1}^{t-1} \langle \ell_i, w_i \rangle \right) \sum_{i=1}^{t-1} \ell_i$}
\STATE{Receive loss vector $\ell_t \in \H$ such that $\norm{\ell_t} \le 1$}
\ENDFOR
}
\end{algorithmic}
\end{algorithm}

Here we present our new parameter-free algorithm for \ac{OLO} over a Hilbert space $\H$, 
stated as Algorithm~\ref{algorithm:hilbert-space-olo}.  We would like to stress the extreme simplicity of the algorithm.
The theorem below upper
bounds its regret in the form of \eqref{equation:olo-parameter-free-2}, the proof can be found
in~\cite{Orabona-Pal-2016-parameter-free}.

\begin{theorem}[Regret Bound for Algorithm~\ref{algorithm:hilbert-space-olo}]
\label{theorem:hilbert-space-olo-regret}
Let $\{\ell_t\}_{t=1}^\infty$ be any sequence of loss vectors
in a Hilbert space $\H$ such that $\norm{\ell_t} \le 1$.
Algorithm~\ref{algorithm:hilbert-space-olo} satisfies
$$
\forall \, T \ge 0 \quad
\forall u \in \H \qquad \qquad
\Regret_T(u) \le \norm{u} \sqrt{T \ln\left(1 + 4T^2 \norm{u}^2 \right)} + 1 \;.
$$
\end{theorem}

We now explain how Algorithm~\ref{algorithm:hilbert-space-olo} is derived from the
Krichevsky-Trofimov solution to the adversarial coin-betting problem.

\textbf{Adversarial Coin Betting.}
Consider a gambler making repeated bets on the outcomes of adversarial coin
flips. The gambler starts with an initial endowment of $1$ dollar. In each
round $t$, he bets on the outcome of a coin flip $c_t \in \{-1,1\}$, where $+1$
denotes heads and $-1$ denotes tails.  The outcome $c_t$ is chosen by an
adversary.  The gambler can bet any amount on either heads or tails. However,
he cannot borrow any additional money. If he loses, he loses the betted amount;
if he wins, he gets the betted amount back and, in addition to that, he gets
the same amount as a reward.  We encode the gambler's bet in round $t$ by a
single number $\beta_t \in [-1,1]$. The sign of $\beta_t$ encodes whether he is
betting on heads or tails. The absolute value encodes the betted amount as the
fraction of his current wealth.  Let $\Wealth_t$ be gambler's wealth at the end
of round $t$. It satisfies
\begin{align}
\label{equation:wealth-recurrence}
\Wealth_0 & = 1 &
& \text{and} &
\Wealth_t & = (1 + c_t \beta_t) \Wealth_{t-1} \qquad \text{for $t \ge 1$} \; .
\end{align}
Note that since $\beta_t \in [-1,1]$, gambler's wealth stays always
non-negative.

\textbf{Kelly Betting and Krichevsky-Trofimov Estimator.}
For sequential betting on i.i.d. coin flips, the optimal strategy has been
proposed by \citet{Kelly-1956}.  The strategy assumes that the coin flips
$\{c_t\}_{t=1}^\infty$, $c_t \in \{+1,-1\}$, are generated i.i.d. with known
probability of heads. If $p \in [0,1]$ is the probability of heads, the Kelly
bet is $\beta_t = 2p - 1$. He showed that, in the long run, this strategy will
provide more wealth than betting any other fixed fraction~\citep{Kelly-1956}.

For adversarial coins, Kelly betting does not make sense.
\citet{Krichevsky-Trofimov-1981} proposed to replace $p$ with an estimate:
After seeing coin flips $c_1, c_2, \dots, c_{t-1}$, use the empirical
estimate $k_t = \frac{1/2 + \sum_{i=1}^{t-1} \indicator[c_i = +1]}{t}$. Their
estimate is commonly called \emph{KT estimator}\footnote{Compared to the
standard maximum likelihood estimate $\frac{\sum_{i=1}^{t-1} \indicator[c_i =
+1]}{t-1}$, KT estimator ``shrinks'' slightly towards $\frac{1}{2}$.} and it
results in the betting strategy $\beta_t = 2k_t - 1 = \tfrac{\sum_{i=1}^{t-1}
c_i}{t}$.  \citeauthor{Krichevsky-Trofimov-1981} showed that this strategy
guarantees almost the same wealth that one would obtain knowing in advance the
fraction of heads. Namely, 
if we denote by $\Wealth_t(\beta)$ the wealth of the strategy that bets the fraction
$\beta$ in every round, then the wealth of the Krichevsky-Trofimov betting strategy
satisfies
\begin{equation}
\label{equation:kt-wealth-lower-bound}
\forall \beta \in [-1,1] \qquad \Wealth_t \ge \frac{\Wealth_t(\beta)}{2\sqrt{t}} \; .
\end{equation}
Moreover, this guarantee is optimal up to constant multiplicative factors~\citep{Cesa-Bianchi-Lugosi-2006}.

\textbf{From betting to \ac{OLO}.}
In Algorithm~\ref{algorithm:hilbert-space-olo}, the ``coin outcome'' is the
vector $c_t \in \H$ where $c_t = -\ell_t$ and algorithm's wealth is $\Wealth_t
= 1 + \sum_{i=1}^t \langle c_i, w_i \rangle = 1 - \sum_{i=1}^t \langle \ell_i,
w_i \rangle$.  The algorithm explicitly keeps track of its wealth and it bets
``vectorial fraction'' $\beta_t = \tfrac{\sum_{i=1}^{t-1} c_i}{t} = -
\tfrac{\sum_{i=1}^{t-1}\ell_i}{t}$ of its current wealth. The regret bound
(Theorem~\ref{theorem:hilbert-space-olo-regret}) is a consequence of
Krichevsky-Trofimov lower bound~\eqref{equation:kt-wealth-lower-bound} on the
wealth and the duality between regret and wealth.  For more details,
see~\cite{Orabona-Pal-2016-parameter-free}.
