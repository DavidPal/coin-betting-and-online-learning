\section{How Do You Tune the Learning Rates?}
\label{section:learning-rates}

Consider \ac{OLO} a Hilbert Space $\H$. \ac{FTRL} with regularizer
$\tfrac{1}{2}\norm{u}^2$ and learning rate $\eta$ satisfies~\cite{???}
\begin{equation}
\label{equation:ftrl-vanila}
\forall u \in \H \qquad \Regret_T(u) \le \norm{u}^2/(2\eta) + (\eta T)/2 \; .
\end{equation}
It is obvious that the optimal tuning of the learning rate depends on the
unknown norm of $u$.

A simple choice $\eta = 1/\sqrt{T}$ leads to an algorithm that satisfies
\begin{equation}
\label{equation:ftrl-vanila-2}
\Regret_T(u) \le 0.5\left(1+\norm{u}^2\right)\sqrt{T} \; .
\end{equation}
However, in this bound the dependency on $\norm{u}$ is suboptimal. As we will
see shortly, the quadratic dependency can be replaced by an (almost) linear
dependency.

Starting from \eqref{equation:ftrl-vanila}, if we choose learning rate $\eta =
D/\sqrt{T}$, we get a family of algorithms parametrized by $D \in [0,\infty)$
that satisfy an analogue of the bound \eqref{equation:hedge-bound-3}:
\begin{equation}
\label{equation:ftrl-vanila-3}
\forall u \in \H : \norm{u} \le D \quad  \Longrightarrow \quad \Regret_T(u) \le D \sqrt{T} \; .
\end{equation}
Instead of family of algorithms parametrized by $D \in [0,\infty)$ satisfying
bound \eqref{equation:ftrl-vanila-3}, one \emph{would like
to have} a single algorithm (without any tuning parameters) satisfying
\begin{equation}
\label{equation:olo-parameter-free}
\forall u \in \H \qquad \Regret_T(u) \le \norm{u} \sqrt{T} \; .
\end{equation}
Notice that \eqref{equation:olo-parameter-free} is stronger than
\eqref{equation:ftrl-vanila-3} in the following sense: A single algorithm
satisfying \eqref{equation:olo-parameter-free} implies
\eqref{equation:ftrl-vanila-3} for all values of $D \in [0,\infty)$. However,
a family of algorithms $\{A_D : D \in [0,\infty)\}$ parametrized by $D$ where
$A_D$ satisfies \eqref{equation:ftrl-vanila-3}, does not yield a single
algorithm that satisfies \eqref{equation:olo-parameter-free}.  Finally, note
that \eqref{equation:olo-parameter-free} has better dependency on $\norm{u}$
than \eqref{equation:ftrl-vanila-2}.

There have been a lot of work on algorithms
\citep{Streeter-McMahan-2012, Orabona-2013, McMahan-Abernethy-2013,
McMahan-Orabona-2014} that satisfy a slightly weaker version of
\eqref{equation:olo-parameter-free}
\begin{equation}
\label{equation:olo-parameter-free-2}
\forall u \in \H \qquad \Regret_T(u) \le \big(O(1)+\polylog(1 + \norm{u})\norm{u} \big) \sqrt{T} \; ,
\end{equation}
where $\polylog(1 + \norm{u})$ represents a function that is upper bounded
by a polynomial in $\log(1+\norm{u})$.\footnote{It can be shown
that for \ac{OLO} over Hilbert space the extra poly-logarithmic factor is
necessary~\citep{McMahan-Abernethy-2013,Orabona-2013}.} Algorithms satisfying
\eqref{equation:olo-parameter-free-2} are called \emph{parameter-free}, since
they do not need to know $D$, yet they have an optimal dependency on $\norm{u}$.

The \emph{exact} same problem is present in the setting of \ac{LEA}.
Hedge is identical to \ac{FTRL} with regularizer $R(u) = -H(u)$, hence it is very easy to show that, with learning rate $\eta$, it satisfies
\begin{equation}
\label{equation:hedge-bound-2}
\Regret_T(u) \le (\ln N - H(u))/\eta + (\eta T)/2 \; .
%= (\KL{u}{\tfrac{1}{N}\indicator})/\eta + (\eta T)/2 \; .
\end{equation}
Dropping the term $-H(u)$ , we can set $\eta = \sqrt{\ln(N)/T}$ and recover the usual
Hedge bound~\citep{Freund-Schapire-1997}
\begin{equation}
\label{equation:hedge-bound}
\forall u \in \Delta_N \qquad \Regret_T(u) \le \sqrt{2 T \ln N}
\end{equation}
This bound is known to be optimal in the worst-case sense~\cite[Section
3.7]{Cesa-Bianchi-Lugosi-2006}. However the right-hand side of \eqref{equation:hedge-bound} is
independent of $u$, that is, the algorithm does not adapt to $u$.
Let $p \in [0, \ln N)$. If we choose $\eta = \sqrt{(\ln(N) - p)/T}$, we get that
\begin{equation}
\label{equation:hedge-bound-3}
\forall u \in \Delta_N: H(u) \ge p \quad \Longrightarrow \quad \Regret_T(u) \le \sqrt{2 T (\ln(N) - p)} \; .
\end{equation}
%The bound \eqref{equation:hedge-bound} corresponds to choice $p=0$ and $\eta=\sqrt{\tfrac{\ln N}{T}}$.
Instead of the family of algorithms parametrized by $p \in [0,\ln N)$ that
satisfy bound~\eqref{equation:hedge-bound-3}, one \emph{would like to have} a single
algorithm (without any tuning parameters) satisfying
\begin{equation}
\label{equation:parameter-free-bound-experts}
\forall u \in \Delta_N \qquad \Regret_T(u) \le \sqrt{2 T (\ln N - H(u))} = \sqrt{2T \cdot \KL{u}{\tfrac{1}{N} \indicator}} \; .
\end{equation}
In the same sense of the OLO case, \eqref{equation:parameter-free-bound-experts} is stronger than
\eqref{equation:hedge-bound-3}.
Bounds of the form~\eqref{equation:parameter-free-bound-experts} were not
considered until 2009. Since then, however, there have been a lot of work
\citep{Chaudhuri-Freund-Hsu-2009, Chernov-Vovk-2010, Koolen-van-Erven-2015,
Luo-Schapire-2014, Luo-Schapire-2015, Foster-Rakhlin-Sridharan-2015,
Orabona-Pal-2016-parameter-free} on algorithms that satisfy slightly
looser\footnote{Earlier papers have extra logarithmic factors.
\citet{Foster-Rakhlin-Sridharan-2015, Orabona-Pal-2016-parameter-free} have
a fixed multiplicative constant hidden in $\widetilde
O(\cdot)$} versions of~\eqref{equation:parameter-free-bound-experts}
\[
\forall u \in \Delta_N \qquad \Regret_T(u) \le \widetilde O(\sqrt{T (1 + \ln N - H(u))}) = \widetilde O\left(\sqrt{T \left(1 + \KL{u}{\tfrac{1}{N}\indicator} \right)} \right) \; .
\]
Algorithms of this type are called \emph{parameter-free} since, in contrast to
Hedge, they do not need to know $p$. The regret in
\eqref{equation:parameter-free-bound-experts} is also called a \emph{quantile bound}
because one can bound the regret with respect to $(\epsilon N)$-th best expert
for any $\epsilon \in (0,1)$ as the regret with respect to
a competitor $u$ that, up to permutation of coordinates, has the form
$
u = \left( 1/(\epsilon N), \dots, 1/(\epsilon N), 0, \dots, 0 \right)
$.
Such competitor satisfies $H(u) = \ln (\epsilon N)$ and the regret with respect
to any such $u$ is $\widetilde O(\sqrt{T (1 + \ln(1/\epsilon)})$.
%In particular, the last bound does not depend on the number of experts $N$, only on the quantile $\epsilon$.
