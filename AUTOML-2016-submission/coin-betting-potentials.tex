\section{Coin-Betting Potentials}
\label{section:coin-betting-potentials}

We introduce a class of potential that gives rise to coin-betting strategies. These potentials will be the key tools in our analysis.

We consider a gambler making repeated bets on outcomes of adversarial coin
flips. The gambler starts with an initial endowment $\epsilon > 0$. In each
round $t$, he bets on an outcome of a coin flip $g_t \in \{-1,1\}$ where $+1$
denotes heads and $-1$ denotes tails. We do not make any assumption on how $g_t$
is generated, that is, it can be chosen by an adversary.

The gambler can bet any amount on either heads or tails. However, he is not
allowed to borrow any additional money. If he loses (i.e. he bets on the
incorrect outcome), he loses the betted amount. If he wins (i.e. he bets on the
correct outcome), he gets the betted amount back and, in addition to that, he
gets the same amount as a reward.
%Note that it does not make sense for the
%gambler to bet on both outcomes, since the same winnings can be achieved by
%betting the difference of the two amounts on one of the outcomes and betting
%zero on the other outcome.
We encode gambler's bet in round $t$ by a single
number $\beta_t \in [-1,1]$. The sign of $\beta_t$ encodes whether he is betting
on heads or tails. The absolute value encodes the betted amount as the fraction
of his current wealth.

Let $\Wealth_t$ be gambler's wealth at the end of round $t$. It satisfies the
recurrence
\begin{align}
\label{equation:wealth-recurrence}
\Wealth_0 & = \epsilon &
& \text{and} &
\Wealth_t & = (1 + g_t \beta_t) \Wealth_{t-1} \qquad \text{for $t \ge 1$} \; .
\end{align}
Note that since $\beta_t \in [-1,1]$, gambler's wealth stays always non-negative.
Gambler's net reward (difference of wealth and initial endowment) after $t$
rounds is
\begin{align}
\label{equation:reward-wealth}
\Reward_t = \Wealth_t - \ \epsilon \; .
\end{align}

We generalize the problem slightly by allowing the outcome of the coin flip
$g_t$ to be any real number in the interval $[-1,1]$. In this case we talk about
\emph{betting on a continuous coin}. The formulas
\eqref{equation:wealth-recurrence} and \eqref{equation:reward-wealth} defining
wealth and reward remain exactly the same.

\begin{definition}[Coin-Betting Potential]
\label{definition:potential}
Let $\epsilon > 0$. Let $\{F_t\}_{t=0}^\infty$ be a sequence of functions
$F_t:R_+  \to \R_+$.  The sequence $\{F_t\}_{t=0}^\infty$ is called a \textbf{sequence of coin-betting potentials
for initial endowment $\epsilon$}, if it satisfies the following three
conditions:
\begin{enumerate}[(1)]
\item $F_0(0) = \epsilon$.

\item For every $t \ge 0$, $F_t(x)$ is even, logarithmically convex, strictly
increasing on $[0,a_t)$, and
$\lim_{x \nearrow a_t} F_t(x) = \lim_{x \searrow -a_t} F_t(x) = +\infty$.

\item For every $t \ge 1$, every $x \in [-(t-1), (t-1)]$ and every $g \in [-1,1]$, and $\beta_t:=\tfrac{F_t(x + 1) - F_t(x - 1)}{F_t(x + 1) + F_t(x - 1)}$, we have
$$
\left(1 + g \beta_t \right) F_{t-1}(x) \ge F_t(x+g) \; .
$$
\end{enumerate}
The sequence $\{F_t\}_{t=0}^\infty$ is called a
\textbf{sequence of excellent coin-betting potentials for initial
endowment $\epsilon$} if it satisfies conditions (1)--(3) and the condition (4)
below.
\begin{enumerate}[(1)]
\setcounter{enumi}{3}
\item For every $t \ge 0$, $F_t$ is twice-differentiable and
satisfies $x \cdot F_t''(x) \ge F_t'(x)$ for every $x \in I_t$.
\end{enumerate}
\end{definition}

It is a routine exercise to show by induction on $t$ that conditions 2 and 3 of
the definition imply that the \emph{potential based strategy}
\eqref{equation:potential-based-strategy} gives a lower bound on the wealth. The base case $t=0$ is trivial,
since both sides of \eqref{equation:wealth-lower-bound-generic} are equal to
$\epsilon$. For $t \ge 1$, if we let $x = \sum_{i=1}^{t-1} g_i$, then
$$
\Wealth_t
= (1+g_t \beta_t) \Wealth_{t-1}
\ge (1 + g_t \beta_t) F_{t-1}(x)
\ge F_t(x + g_t)
= F_t \left( \sum_{i=1}^t g_i \right) \; .
$$

%The formula for the potential-based
% strategy~\eqref{equation:potential-based-strategy} might seem strange. However,
% it can be derived by considering the inequality
% $$
% (1+g_t\beta_t) F_{t-1}(x) \ge F_{t-1}(x + g_t)
% $$
% used in the induction proof above. Dividing through $1+g_t\beta_t$, the left-hand
% side becomes independent of $g_t$ and $\beta_t$ and the right-hand side becomes
% a function of $\beta_t$ and $g_t$,
% $$
% h(g_t, \beta_t) = \tfrac{F_{t-1}(x + g_t)}{1 + g_t\beta_t}
% $$
% Since $g_t \in [-1,1]$ is chosen by an adversary, our best hope is to find
% $\beta_t$ that minimizes $\max_{g_t \in [-1,1]} h(g_t,\beta_t)$. Solution of
% this minimization problem is $\beta_t$ given by
% \eqref{equation:potential-based-strategy}; see
% Theorem~\ref{theorem:optimal-betting-fraction} in
% Appendix~\ref{section:optimal-betting-fraction}.

We show concrete examples of sequences of coin-betting potentials in
Section~\ref{section:kt-estimator}.
