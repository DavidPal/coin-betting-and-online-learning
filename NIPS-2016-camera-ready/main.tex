\documentclass{article}

\PassOptionsToPackage{numbers, compress}{natbib}

\usepackage[final]{nips_2016}
% \usepackage{nips_2016}
% For anonymous submission use:  \usepackage{nips_2016}
% For camera-ready version use:  \usepackage[final]{nips_2016}

\usepackage{amssymb,amsmath,amsthm}
\usepackage[nolist]{acronym}
\usepackage{algorithm,algorithmic}
\usepackage{enumerate}

\usepackage[colorlinks,linkcolor=blue,citecolor=blue]{hyperref}
\usepackage{booktabs}       % professional-quality tables
\usepackage{nicefrac}       % compact symbols for 1/2, etc.
\usepackage{graphicx}
\usepackage{url}
\usepackage{subfigure}

\graphicspath{{figs/}}

\newtheorem{definition}{Definition}
\newtheorem{lemma}[definition]{Lemma}
\newtheorem{theorem}[definition]{Theorem}
\newtheorem{corollary}[definition]{Corollary}
\newtheorem{proposition}[definition]{Proposition}

\DeclareMathOperator*{\Exp}{\mathbf{E}}
\DeclareMathOperator{\Regret}{Regret}
\DeclareMathOperator{\Wealth}{Wealth}
\DeclareMathOperator{\Reward}{Reward}
\DeclareMathOperator*{\argmin}{arg\,min}
\DeclareMathOperator*{\argmax}{arg\,max}

\newcommand{\R}{\mathbb{R}}     % real numbers
\renewcommand{\H}{\mathcal{H}}  % Hilbert space
\newcommand{\KL}[2]{\operatorname{D}\left({#1}\middle\|{#2}\right)}  % KL divergence
\newcommand{\norm}[1]{\left\|{#1}\right\|}
\newcommand{\indicator}{\mathbf{1}}
\newcommand{\scO}{\mathcal{O}}  % big-Oh
\DeclareMathOperator*{\dom}{dom}    % KL divergence

\begin{acronym}
\acro{EG}{Exponentiated Gradient}
\acro{DFEG}{Dimension-Free Exponentiated Gradient}
\acro{OMD}{Online Mirror Descent}
\acro{ASGD}{Averaged Stochastic Gradient Descent}
\acro{SGD}{Stochastic Gradient Descent}
\acro{PiSTOL}{Parameter-free STOchastic Learning}
\acro{OCO}{Online Convex Optimization}
\acro{OLO}{Online Linear Optimization}
\acro{RKHS}{Reproducing Kernel Hilbert Space}
\acro{IID}{Independent and Identically Distributed}
\acro{SVM}{Support Vector Machine}
\acro{ERM}{Empirical Risk Minimization}
\acro{COCOB}{Continous Coin Betting}
\acro{MBA}{Master Betting Algorithm}
\acro{KT}{Krichevsky-Trofimov}
\acro{LEA}{Learning with Expert Advice}
\acro{OGD}{Online Gradient Descent}
\acro{KL}{Kullback-Leibler}
\end{acronym}

\author{
  Francesco Orabona\\
  Stony Brook University, Stony Brook, NY \\
  \texttt{francesco@orabona.com}
  \And
  D\'avid P\'al\\
  Yahoo Research, New York\\
  \texttt{dpal@yahoo-inc.com}
}

\title{Coin Betting and Parameter-Free Online Learning}

\begin{document}

\maketitle

\begin{abstract}
In the recent years, a number of parameter-free algorithms have been developed
for online linear optimization over Hilbert spaces and for learning with expert
advice.  These algorithms achieve optimal regret bounds that depend on the
unknown competitors, without having to tune the learning rates with oracle
choices.

We present a new intuitive framework to design parameter-free algorithms based
on a reduction to betting on outcomes of an adversarial coin. We instantiate it
using a betting algorithm based on the Krichevsky-Trofimov estimator.  The
resulting algorithms are simple, with no parameters to be tuned, and they
improve or match previous results in terms of regret guarantee and per-round
complexity.
\end{abstract}

\vspace{-0.2cm}

\section{Introduction}
\label{section:introduction}

We consider the \ac{OLO}~\cite{Cesa-Bianchi-Lugosi-2006, Shalev-Shwartz-2011}
setting. In each round $t$, an algorithm chooses a point $w_t$ from a convex
\emph{decision set} $K$ and then receives a reward vector $g_t$. Algorithm's
goal is to keep its \emph{regret} small, defined as the difference between its
cumulative reward and the cumulative reward of a fixed strategy $u \in K$, that
is
\[
\Regret_T(u) = \sum_{t=1}^T \langle g_t, u \rangle - \sum_{t=1}^T \langle g_t, w_t \rangle \; .
\]

We focus on two particular sets, the $N$-dimensional probability simplex
$\Delta_N = \{ x \in \R^N ~:~ x \ge 0, \norm{x}_1 = 1\}$ and the Hilbert space
$\H$.  \ac{OLO} over $\Delta_N$ is referred to as the problem of \ac{LEA}.  We
assume bounds on the norms of the reward vectors: For \ac{OLO} over $\H$, we
assume that $\norm{g_t} \le 1$, and for \ac{LEA} we assume that $g_t \in
[0,1]^N$.

\ac{OLO} is a basic building block of many machine learning problems. For
example, \ac{OCO}, the problem analogous to \ac{OLO} where $-\langle g_t, w_t
\rangle$ is generalized to $\ell_t(w_t)$ where $\ell_t$ is an arbitrary convex
function, is solved through a reduction to an
\ac{OLO}~\cite{Shalev-Shwartz-2011}.  \ac{LEA}~\cite{Littlestone-Warmuth-1994,
Vovk-1998, Cesa-Bianchi-Freund-Haussler-Helmbold-Schapire-Warmuth-1997}
provides a way of combining classifiers and it is at the heart of
boosting~\cite{Freund-Schapire-1997}. Batch and stochastic convex optimization
can be solved through a reduction to \ac{OLO}~\cite{Shalev-Shwartz-2011}.
Statistical learning with convex losses can also seen as stochastic convex
optimization and solved through \ac{OCO}~\cite{Munro-1951}, that in turns can
be reduced to \ac{OLO}.

However, to achieve small regret, most of the algorithms still require to set
hyperparameters (e.g., learning rates, regularization weights) in order to
achieve the best possible theoretical and empirical performance.  Recently, a
new family of parameter-free algorithms that has been proposed, both for
\ac{LEA}~\cite{Chaudhuri-Freund-Hsu-2009, Chernov-Vovk-2010, Luo-Schapire-2014,
Luo-Schapire-2015, Koolen-van-Erven-2015} and for \ac{OLO}/\ac{OCO} over
Hilbert spaces~\cite{Streeter-McMahan-2012, Orabona-2013,
McMahan-Abernethy-2013, McMahan-Orabona-2014, Orabona-2014}.  These algorithms
adapt to the number of experts and to the norm of the optimal predictor,
respectively, without the need to tune parameters.  Surprisingly, these two
families of algorithms are very similar as shown in the non-constructive
framework of~\cite{Foster-Rakhlin-Sridharan-2015}.  Given the connections
between \ac{OLO}/\ac{LEA} and machine learning, these algorithms allow to
design parameter-free batch machine learning algorithms through straightforward
reductions~\cite{Orabona-2014, Luo-Schapire-2015}.  Yet, all of existing
algorithms for LEA either have sub-optimal regret bound (e.g. extra $\scO(\log
\log T)$ factor) or sub-optimal running time (e.g. requiring solving a
numerical problem in every round, or with extra factors); see
Table~\ref{table:bounds}.

\textbf{Contributions.} We show that a more fundamental notion subsumes \emph{both}
\ac{OLO} and \ac{LEA} parameter-free algorithms. We show that the ability to
maximize the wealth in bets on the outcomes of coin flips \emph{implies}
\ac{OLO} and \ac{LEA} parameter-free algorithms.  Hence, we present a novel
framework to characterize betting algorithms through potential functions and we
instantiate it with the Krichevsky-Trofimov estimator.  The resulting
algorithms are new, simple, and natural. They also have optimal guarantees on
time and regret, see Table~\ref{table:bounds}.

\begin{table}
\begin{center}
\resizebox{\linewidth}{!}{%
\begin{tabular}{l c c c c}
\toprule
Algorithm & Worst-case regret guarantee& \begin{tabular}{@{}c@{}}Per-Round Time\\Complexity\end{tabular} & Adaptive & \begin{tabular}{@{}c@{}}Unified \\  design\end{tabular}\\
\midrule
\cite{Abernethy-Bartlett-Rakhlin-Tewari-2008} & \begin{tabular}{@{}c@{}}$U \sqrt{T}$ for any $u \in \H$ s.t. $\norm{u} \le U$\\
$\scO(T)$ otherwise\end{tabular} & $\scO(1)$ &  \\
OGD, $\eta=\tfrac{1}{\sqrt{T}}$ \cite{Shalev-Shwartz-2011} & $\scO((1 + \norm{u}^2)\sqrt{T})$, $\forall u \in \H$ & $\scO(1)$ &  \\
\cite{McMahan-Orabona-2014} & $\scO(\norm{u}\sqrt{T \ln(1+\norm{u}T)})$, $\forall u \in \H$ & $\scO(1)$ & \checkmark \\
This paper, Sec.~\ref{section:kt-olo} & $\scO(\norm{u}\sqrt{T \ln(1+\norm{u}T)})$, $\forall u \in \H$ & $\scO(1)$ & \checkmark & \checkmark\\
\midrule
Hegde~\cite{Freund-Schapire-1997}, $\eta=\sqrt{\tfrac{\ln N - U}{T}}$ & $\scO\left(\sqrt{T (\ln N -U)}\right)$ for any $u \in \Delta_N$ s.t. $H(u) \geq U$ & $\scO(N)$ &  \\
\cite{Chaudhuri-Freund-Hsu-2009}  & $\scO\left(\sqrt{T \KL{u}{\tfrac{1}{N}\boldsymbol{1}}}+\ln^2 N\right)$, $\forall u \in \Delta_N$ & $\scO(N\,K)$\footnotemark[1]& \checkmark \\
\cite{Chernov-Vovk-2010} & $\scO\left(\sqrt{T \left(1+\KL{u}{\tfrac{1}{N}\boldsymbol{1}}\right)}\right)$, $\forall u \in \Delta_N$ & $\scO(N\,K)$\footnotemark[1] & \checkmark \\
\cite{Chernov-Vovk-2010, Luo-Schapire-2015,Koolen-van-Erven-2015} & $\scO\left(\sqrt{T \left(\ln \ln T+\KL{u}{\tfrac{1}{N}\boldsymbol{1}}\right)}\right)$, $\forall u \in \Delta_N$ & $\scO(N)$ & \checkmark \\
\cite{Foster-Rakhlin-Sridharan-2015} & $\scO\left(\sqrt{T \left(1+\KL{u}{\tfrac{1}{N}\boldsymbol{1}}\right)}\right)$, $\forall u \in \Delta_N$ & $\scO(N \max_{u \in \Delta_N} \ln \KL{u}{\pi})$\footnotemark[2] & \checkmark & \checkmark \\
This paper, Sec.~\ref{section:kt-lea} & $\scO\left(\sqrt{T \left(1+\KL{u}{\tfrac{1}{N}\boldsymbol{1}}\right)}\right)$, $\forall u \in \Delta_N$ & $\scO(N)$ & \checkmark & \checkmark\\
\bottomrule
\end{tabular}}
\caption{Algorithms for \ac{OLO} over Hilbert space and \ac{LEA}}
\label{table:bounds}
\end{center}
\end{table}
\footnotetext[1]{These algorithms require to solve a numerical problem at each step. The number $K$ is the number of steps needed to reach the required precision. Neither the required precision
nor the number of steps to reach it is calculated in these papers. In general, $K$ is a function of the number rounds $T$ and the number of experts $N$.}
\footnotetext[2]{A slighly modified version of algorithm in \cite{Foster-Rakhlin-Sridharan-2015} can be implemented with the stated time complexity~\cite{Foster:private}.}

%\vspace{-0.1cm}

\section{Preliminaries}
\label{section:preliminaries}

We begin by providing some definitions.
The \ac{KL} divergence between two discrete distributions $p$ and $q$ is
$\KL{p}{q} = \sum_{i} p_i \ln\left( p_i/q_i \right)$. If $p,q$ are real numbers
in $[0,1]$, we denote by $\KL{p}{q} = p \ln \left(p/q \right) + (1-p) \ln
\left((1-p)/(1-q) \right)$ the \ac{KL} divergence between two Bernoulli
distributions with parameters $p$ and $q$.  We denote by $\H$ a Hilbert space
with inner product $\langle \cdot, \cdot\rangle$ and by $\norm{\cdot}$ the
induced norm.  We denote by $\norm{\cdot}_1$ the $1$-norm in $\R^N$.  A
function $F:I \to \R_+$ is called \emph{logarithmically convex} iff $f(x) =
\ln(F(x))$ is convex.  Let $f:V \to \R \cup \{\pm\infty\}$, the Fenchel
conjugate of $f$ is $f^*:V^* \to \R \cup \{\pm \infty\}$ defined on the dual
vector space $V^*$ by $f^*(\theta) = \sup_{x \in V} \ \langle \theta, x \rangle
- f(x)$.  A function $f:V \to \R \cup \{+ \infty\}$ is said to be proper if
$\dom f \neq \emptyset$.  If $f$ is a proper lower semi-continuous convex
function then $f^*$ is also proper lower semi-continuous convex and
$f^{**}=f$.

\textbf{Coin Betting.} We consider a gambler making
repeated bets on the outcomes of adversarial coin flips. The gambler starts with an
initial endowment $\epsilon > 0$. In each round $t$, he bets on the outcome of a
coin flip $g_t \in \{-1,1\}$, where $+1$ denotes heads and $-1$ denotes tails.
We do not make any assumption on how $g_t$ is generated, that is, it can be
chosen by an adversary.

The gambler can bet any amount on either heads or tails. However, he is not
allowed to borrow any additional money. If he loses, he loses the betted
amount; if he wins, he gets the betted amount back and, in addition to that, he
gets the same amount as a reward.  We encode the gambler's bet in round $t$ by a
single number $\beta_t \in [-1,1]$. The sign of $\beta_t$ encodes whether he is
betting on heads or tails. The absolute value encodes the betted amount as the
fraction of his current wealth.

Defining $\Wealth_t$ to be the gambler's wealth at the end of round $t$, it satisfies the
recurrence
\begin{align}
\label{equation:wealth-recurrence}
\Wealth_0 & = \epsilon &
& \text{and} &
\Wealth_t & = (1 + g_t \beta_t) \Wealth_{t-1} \qquad \text{for $t \ge 1$} \; .
\end{align}
Note that since $\beta_t \in [-1,1]$, the gambler's wealth stays always non-negative.
The gambler's net reward (difference of wealth and initial endowment) after $t$
rounds is
\begin{align}
\label{equation:reward-wealth}
\Reward_t = \Wealth_t - \ \epsilon \; .
\end{align}
We generalize the problem slightly by allowing the outcome of the coin flip
$g_t$ to be any real number in the interval $[-1,1]$.  Equations
\eqref{equation:wealth-recurrence} and \eqref{equation:reward-wealth} defining
wealth and reward remain exactly the same.

\section{Warm-Up: One-Dimensional Hilbert Space}
\label{section:one-dimensional-hilbert-space-olo}

There are many frameworks for the design and analysis of online learning
algorithms, e.g. the potential function view~\cite{Cesa-Bianchi-Lugosi-2006},
the regularizer view~\cite{Shalev-Shwartz-2011}, relax and
randomize~\cite{Rakhlin-Shamir-Sridharan-2012}. However, they do not
provide much help on how to craft the potential, regularizer, or relaxation.
Here, we show how the betting view gives a very natural intuition and it also provides a clear path to the tools needed.

As a warm-up, let us consider an algorithm for OLO over one-dimensional Hilbert
space $\R$.  Let $\{w_t\}_{t=1}^\infty$ be its sequence of predictions on a
sequence of rewards $\{g_t\}_{t=1}^\infty$, $g_t \in [-1,1]$. The total reward
of the algorithm after $t$ rounds is $\Reward_t = \sum_{i=1}^t g_i w_i$.

Let us define ``wealth'' of the OLO algorithm as $\Wealth_t = \epsilon +
\Reward_t$, in accordance with \eqref{equation:reward-wealth}.
Intuitively, we want the reward to be big, on any sequence of $g_t$, so that
the regret will be small. We now restrict our attention to algorithms whose
predictions are of the form of a bet:
\vspace{-.1cm}
\begin{equation}
\label{equation:one-dimensional-olo}
w_t = \beta_t \Wealth_{t-1}
= \beta_t \left(\epsilon+ \sum_{i=1}^{t-1} g_i w_i \right),
\end{equation}
where $\beta_t \in [-1,1]$. In this way the recurrence
\eqref{equation:wealth-recurrence} holds. We will see that this restriction
does not prevent us to obtain parameter-free algorithms with optimal bounds.
If we have a coin-betting algorithm that, on a sequence of coin flips
$\{g_t\}_{t=1}^\infty$, $g_t \in [-1,1]$, bets fractions $\beta_t \in [-1,1]$,
we can use it to construct an OLO algorithm in a one-dimensional Hilbert space
$\R$ according to \eqref{equation:one-dimensional-olo}.

Assume now that the betting algorithm at hand guarantees that its wealth is at
least $F(\sum_{t=1}^T g_t)$ starting from an endowment $\epsilon$, then
\vspace{-.1cm}
\begin{equation}
\label{equation:one-dimensional-olo-reward-lower-bound}
\Reward_T
= \sum_{t=1}^T g_t w_t
= \Wealth_T \ - \ \epsilon \ge F\left(\sum_{t=1}^T g_t \right) \ - \ \epsilon \; .
\end{equation}

We are almost done, we just need to convert a lower bound on the reward to an
upper bound on the regret. This can be done using the following lemma
from~\cite{McMahan-Orabona-2014}.
\begin{lemma}[Reward-Regret relationship~\cite{McMahan-Orabona-2014}]
\label{lemma:reward-regret}
Let $V,V^*$ be a pair of dual vector spaces. Let $F:V \to \R \cup \{+\infty\}$
be a proper convex lower semi-continuous function and let $F^*:V^* \to \R \cup
\{+\infty\}$ be its Fenchel conjugate. Let $w_1, w_2, \dots, w_T \in V$ and
$g_1, g_2, \dots, g_T \in V^*$.  Then,
\[
\underbrace{\sum_{t=1}^T \langle g_t, w_t \rangle}_{\Reward_T} \ge F\left( \sum_{t=1}^T g_t \right) -\epsilon
\qquad \text{is equivalent to} \qquad
\forall u \in V^*, \quad
\underbrace{\sum_{t=1}^T \langle g_t, u - w_t\rangle}_{\Regret_T(u)} \le F^*(u) + \epsilon\; .
\]
\end{lemma}
\vspace{-.1cm}
Applying the lemma, we get a regret upper bound
\[
\forall u \in \H \qquad \qquad
\Regret_T(u) \le F^*(u) \ + \ \epsilon \; .
\]

Hence, we have showed that if we have a betting algorithm that guarantees a
minimum wealth, $F(\sum_{t=1}^T g_t)$, this can be used to design and analyze
the one-dimensional \ac{OLO} case. Moreover, the faster is the growth of the wealth, the
smaller the regret will be.  
Also, the most important point is that \emph{optimal methods for the betting sccenario are already known}, see, e.g., \cite[Section 9]{Cesa-Bianchi-Lugosi-2006}. We can re-use them to get online algorithms with \emph{optimal} regret guarantees.

In the next sections, we will provide the tools to analyze the betting case and the reductions to extend the 1-d case to the generic \ac{OLO} case and to \ac{LEA} as well.

\section{Coin-Betting Potentials}
\label{section:coin-betting-potentials}

In this section, we will provide a framework to analyze betting algorithms.
We will also presents the reductions to effortlessly extend the one dimensional case in the previous section to the
generic \ac{OLO} case and to \ac{LEA} as well.

For sequential betting on i.i.d. coin flips, the optimal strategy has been proposed by \citet{Kelly-1956}.
The strategy assumes that the coin flips $\{g_t\}_{t=1}^\infty$, $g_t
\in \{+1,-1\}$, are generated i.i.d. with known probability of heads. If $p \in
[0,1]$ is the probability of heads, the Kelly bet is $\beta_t = 2p - 1$. He
showed that, in the long run, this strategy will provide more wealth than
betting any other fixed fraction~\cite{Kelly-1956}.

For adversarial coins, Kelly betting does not make sense. With perfect knowledge
of the future, the gambler could always bet everything on the right outcome.
Hence, after $T$ rounds from an initial endowment $\epsilon$, the maximum
wealth he could get is $\epsilon 2^T$.  Instead, assume he bets the same
fraction $\beta$ at each round and let $\Wealth_t(\beta)$ the wealth of such strategy
after $t$ rounds. As observed in
\cite{McMahan-Abernethy-2013}, the optimal fixed fraction to bet is
$\beta^*=(\sum_{t=1}^T g_t)/T$ and it gives the wealth
\begin{equation}
\label{eq:opt_wealth}
% \max_{\beta \in [-1,1]} \Wealth_t(\beta)
%=
\Wealth_t(\beta^*)
= \epsilon \exp\left(T \KL{\tfrac{1}{2}+\tfrac{\sum_{t=1}^T g_t}{2T}}{\tfrac{1}{2}}\right)
\geq \epsilon \exp\left(\tfrac{(\sum_{t=1}^T g_t)^2}{2 T}\right) \; .
\end{equation}
%In this sense,
%\emph{good} betting strategies should guarantee an exponential growth of the
%wealth, when the outcomes of the coin are predictable.

However, even without knowledge of the future, it is possible to go very close to the wealth in \eqref{eq:opt_wealth}.
This problem was studied by \citet{Krichevsky-Trofimov-1981},
who proposed that after seeing coin flips $g_1, g_2, \dots, g_{t-1}$ the
empirical estimate $k_t = \frac{1/2 + \sum_{i=1}^{t-1} \indicator[g_i = +1]}{t}$
should be used instead of $p$. Their estimate is commonly called \emph{KT
estimator}.\footnote{Compared to the maximum likelihood estimate
$\frac{\sum_{i=1}^{t-1} \indicator[g_i = +1]}{t-1}$, KT estimator shrinks
slightly towards $\nicefrac{1}{2}$.} The KT estimator results in the betting strategy
\begin{equation}
\label{equation:kt-estimator-betting-strategy}
\beta_t = 2k_t - 1 = \tfrac{\sum_{i=1}^{t-1} g_i}{t}
\end{equation}
which we call \emph{adaptive Kelly betting based on KT estimator}. It looks like an online and slightly biased version of the oracle choice of $\beta^*$.
\citeauthor{Krichevsky-Trofimov-1981} showed that this strategy guarantees
\[
\Wealth_t \ge \frac{\Wealth_t(\beta^*)}{2\sqrt{t}}
= \tfrac{\epsilon}{2\sqrt{t}} \exp\left(T \KL{\tfrac{1}{2}+\tfrac{\sum_{t=1}^T g_t}{2T}}{\tfrac{1}{2}}\right)\; .
\]
This guarantee is optimal up to constant factors~\citep{Cesa-Bianchi-Lugosi-2006} and mirrors the guarantee of
the Kelly bet.


Here, we propose a new set of definitions that allows to generalize the
strategy of adaptive Kelly betting based on KT estimator. For these strategies
it will be possible to prove that, for any $g_1, g_2, \dots, g_t \in [-1,1]$,
\begin{equation}
\label{equation:wealth-lower-bound-generic}
\Wealth_t \ge F_t \left( \sum_{i=1}^t g_i \right) \; ,
\end{equation}
where $F_t(x)$ is a certain function. We call such functions \emph{potentials}.
The betting strategy will be determined uniquely by the potential (see (c) in
the Definition~\ref{definition:potential}), and we restrict our attention to
potentials for which \eqref{equation:wealth-lower-bound-generic} holds. These
constraints are specified in the definition below.
\begin{definition}[Coin-Betting Potential]
\label{definition:potential}
Let $\epsilon > 0$. Let $\{F_t\}_{t=0}^\infty$ be a sequence of functions
$F_t:(-a_t, a_t)  \to \R_+$ where $a_t > t$.  The sequence
$\{F_t\}_{t=0}^\infty$ is called a \textbf{sequence of coin-betting potentials
for initial endowment $\epsilon$}, if it satisfies the following three
conditions:
\begin{enumerate}[(a)]
\item $F_0(0) = \epsilon$.

\item For every $t \ge 0$, $F_t(x)$ is even, logarithmically convex, strictly
increasing on $[0,a_t)$, and
$\lim_{x \to a_t} F_t(x) = +\infty$.
\item Given $t \ge 1$ and $x \in [-(t-1), (t-1)]$, let
\begin{equation}
\label{equation:potential-based-strategy}
\beta_t=\tfrac{F_t(x + 1) - F_t(x - 1)}{F_t(x + 1) + F_t(x - 1)} \;.
\end{equation}
For every $t \ge 1$, every $x \in [-(t-1), (t-1)]$ and every $g \in [-1,1]$, $\left(1 + g \beta_t \right) F_{t-1}(x) \ge F_t(x+g)$.
\end{enumerate}
The sequence $\{F_t\}_{t=0}^\infty$ is called a
\textbf{sequence of excellent coin-betting potentials for initial
endowment $\epsilon$} if it satisfies conditions (a)--(c) and the condition (d)
below.
\begin{enumerate}[(a)]
\setcounter{enumi}{3}
\item For every $t \ge 0$, $F_t$ is twice-differentiable and
satisfies $x \cdot F_t''(x) \ge F_t'(x)$ for every $x \in [0,a_t)$.
\end{enumerate}
\end{definition}

Let's give some intuition on this definition.  First, let's show by induction
on $t$ that (b) and (c) of the definition together with
\eqref{equation:wealth-recurrence} give a betting strategy that satisfies
\eqref{equation:wealth-lower-bound-generic}. The base case $t=0$ is trivial. At
time $t \ge 1$, let $x = \sum_{i=1}^{t-1} g_i$ and bet $\beta_t \Wealth_{t-1}$
where $\beta_t$ is defined in \eqref{equation:potential-based-strategy}, then
\vspace{-.2cm}
\[
\Wealth_t
= (1+g_t \beta_t) \Wealth_{t-1}
\ge (1 + g_t \beta_t) F_{t-1}(x)
\ge F_t(x + g_t)
= F_t \left( \sum_{i=1}^t g_i \right) \; .
\]

The formula for the potential-based
strategy~\eqref{equation:potential-based-strategy} might seem strange. However,
it is derived---see Theorem~\ref{theorem:optimal-betting-fraction} in\footnote{For lack of space, all the appendices are in the supplementary material.}
Appendix~\ref{section:optimal-betting-fraction}---by minimizing the worst-case
value of the right-hand side of the inequality used w.r.t. to $g_t$ in the
induction proof above: $F_{t-1}(x) \ge \tfrac{F_{t}(x + g_t)}{1+g_t\beta_t}$.

The last point, (d), is a technical condition that allows us to
seamlessly reduce the infinite dimensional case to the one dimensional one, characterizing the worst case direction for the reward vectors.

Regarding the design of coin betting potentials, we expect any potential that approximates the best possible wealth in \eqref{eq:opt_wealth} to be a good candidate.
In fact, $F_t(x)=\epsilon
\exp \left(x^2/(2t)\right)/\sqrt{t}$, essentially the potential used in
the parameter-free algorithms in \cite{McMahan-Orabona-2014, Orabona-2014} for
\ac{OLO} and in \cite{Chaudhuri-Freund-Hsu-2009, Luo-Schapire-2014,
Luo-Schapire-2015} for \ac{LEA}, approximates \eqref{eq:opt_wealth} and it is an excellent coin betting potential---see Theorem~\ref{thm:exp_x2} in Appendix~\ref{section:optimal-betting-fraction}. Hence, our framework provides intuition to previous constructions and in Section~\ref{section:kt-estimator} we show new examples of coin-betting potentials. In the next two sections, we reduce \ac{OLO} and \ac{LEA} to coin-betting.

\section{From Coin Betting to OLO over Hilbert Space}
\label{section:reduction_hilbert}

We show how to use a sequence of coin-betting potentials $\{F_t\}_{t=0}^\infty$
to construct an algorithm for \ac{OLO} over a Hilbert space
and how to prove regret bound for it. The basic idea is to realize that
continuous coin-betting corresponds to a certain type of algorithms
for \ac{OLO} over one-dimensional Hilbert space $\R$.
This idea can be then generalized to an arbitrary Hilbert space $\H$.

\subsection{A Warm-Up: One-Dimensional Hilbert Space}
\label{section:one-dimensional-hilbert-space-olo}

Let us consider an algorithm for OLO over one-dimensional Hilbert space $\R$.
Let $\{w_t\}_{t=1}^\infty$ be its sequence of predictions on a sequence of
rewards $\{g_t\}_{t=1}^\infty$, $g_t \in [-1,1]$. The total reward of the
algorithm after $t$ rounds is
$$
\Reward_t = \sum_{i=1}^t g_i w_i \; .
$$
Let us define ``wealth'' of the OLO algorithm as $\Wealth_t = \epsilon +
\Reward_t$ in accordance with equation
\eqref{equation:reward-wealth}. Now, suppose we want to satisfy the recurrence
\eqref{equation:wealth-recurrence}. Clearly, the recurrence is not necessarily
satisfied for an arbitrary OLO algorithm. However, if we assume that its
predictions are of the form \begin{equation}
\label{equation:one-dimensional-olo}
w_t = \beta_t \Wealth_{t-1}
= \beta_t \left(\epsilon+ \sum_{i=1}^{t-1} g_i w_i \right),
\end{equation}
where $\beta_t \in [-1,1]$, we see that the recurrence
\eqref{equation:wealth-recurrence} holds. Indeed,
$$
\Wealth_t
= g_t w_t + \Wealth_{t-1}
= \beta_t \Wealth_{t-1} + \Wealth_{t-1}
= (1+\beta_t) \Wealth_{t-1} \; .
$$
This of course works in reverse: If we have a coin-betting algorithm that on a
sequence of coin flips $\{g_t\}_{t=1}^\infty$, $g_t \in [-1,1]$, bets fractions
$\beta_t \in [-1,1]$, we can use it to construct an OLO algorithm in a
one-dimensional Hilbert space $\R$ according to equation
\eqref{equation:one-dimensional-olo}.

If the betting algorithm is based on a sequence of coin-betting potentials
$\{F_t\}_{t=1}^\infty$ then using \eqref{equation:wealth-lower-bound-generic},
\begin{equation}
\label{equation:one-dimensional-olo-reward-lower-bound}
\Reward_T
= \sum_{t=1}^T g_t w_t
= \Wealth_T \ - \ \epsilon \ge F_T\left(\sum_{t=1}^T g_t \right) \ - \ \epsilon \; .
\end{equation}
It is straightforward to convert a lower bound on the reward to an upper bound
on the regret. This can be done using the following lemma, as observed
by~\cite{McMahanO14}.
\begin{lemma}[Reward-Regret relationship]
\label{lemma:reward-regret}
Let $V,V^*$ be a pair dual vector spaces. Let $F:V \to \R \cup \{+\infty\}$ be a
proper convex lower semi-continuous function and let $F^*:V^* \to \R \cup
\{+\infty\}$ be its Fenchel conjugate. Let $w_1, w_2, \dots, w_T \in V$ and
$g_1, g_2, \dots, g_T \in V^*$ be two sequences of vectors.  Then,
$$
\underbrace{\sum_{t=1}^T \langle g_t, w_t \rangle}_{\Reward_T} \ge F\left( \sum_{t=1}^T g_t \right)
\qquad \text{is equivalent to} \qquad
\forall u \in V^*, \quad
\underbrace{\sum_{t=1}^T \langle g_t, u - w_t\rangle}_{\Regret_T(u)} \le F^*(u) \; .
$$
\end{lemma}

Applying the lemma to the function $F(x) = F_T(x) - \epsilon$, we get a regret
upper bound
$$
\forall u \in \H \qquad \qquad
\Regret_T(u) \le F_T^*(u) \ + \ \epsilon \; .
$$

\subsection{Arbitrary Hilbert Space}

The one-dimensional construction for OLO can be generalized to an arbitrary
Hilbert space $\H$. Reward and wealth are defined analogously
to the one-dimensional case:
\begin{align*}
\Reward_t &= \sum_{i=1}^t \langle g_i, w_i \rangle &
& \text{and} &
\Wealth_t &= \epsilon + \Reward_t \; .
\end{align*}
Given a sequence of coin-betting potentials $\{F_t\}_{t=0}^\infty$,
we define fraction
\begin{equation}
\label{equation:potential-based-strategy-hilbert-space}
\beta_t = \frac{F_t \left(\norm{\sum_{i=1}^{t-1} g_i} + 1\right) - F_t\left(\norm{\sum_{i=1}^{t-1} g_i} - 1 \right)}{F_t\left(\norm{\sum_{i=1}^{t-1} g_i} + 1 \right) + F_t\left(\norm{\sum_{i=1}^{t-1} g_i} - 1 \right)} \; .
\end{equation}
This definition of $\beta_t$ is a generalization of equation
\eqref{equation:potential-based-strategy}.  Analogously to equation
\eqref{equation:one-dimensional-olo}, the prediction of the OLO algorithm
defined by this potentials is
\begin{equation}
\label{equation:hilbert-space-olo}
w_t = \beta_t \Wealth_{t-1} \frac{\sum_{i=1}^{t-1} g_i}{\norm{\sum_{i=1}^{t-1} g_i}}
= \beta_t \frac{\sum_{i=1}^{t-1} g_i}{\norm{\sum_{i=1}^{t-1} g_i}} \left(\epsilon+ \sum_{i=1}^{t-1} \langle g_i, w_i\rangle \right) \; .
\end{equation}
The only difference between \eqref{equation:hilbert-space-olo} and
\eqref{equation:one-dimensional-olo} is the multiplication by the unit vector
$\frac{\sum_{i=1}^{t-1} g_i}{\norm{\sum_{i=1}^{t-1} g_i}}$. If $\sum_{i=1}^{t-1}
g_i$ is the zero vector, we define $w_t$ be the zero vector as well.
For this prediction strategy we can prove the following regret guarantee.
%
\begin{theorem}[Regret Bound for OLO in Hilbert Spaces]
\label{theorem:hilbert-space-olo-regret-bound}
Let $\{F_t\}_{t=0}^\infty$ be a sequence of excellent coin-betting potentials.
Let $\{g_t\}_{t=1}^\infty$ be any sequence of reward vectors in a Hilbert space
$\H$ such that $\norm{g_t} \le 1$ for all $t$. Then, the algorithm that makes
prediction $w_t$ defined by \eqref{equation:hilbert-space-olo} and
\eqref{equation:potential-based-strategy-hilbert-space} satisfies
$$
\forall u \in \H \qquad \qquad
\Regret_T(u) \le F_T^*\left(\norm{u} \right) \ + \ \epsilon \; .
$$
\end{theorem}

\begin{proof}
Compared to the one dimensional case, the only hard part is to show an analogue of \eqref{equation:wealth-lower-bound-generic},
\begin{equation}
\label{equation:wealth-lower-bound-hilbert-space}
\Wealth_t \ge F_t\left(\norm{\sum_{t=1}^T g_t} \right) \; .
\end{equation}
To prove \eqref{equation:wealth-lower-bound-hilbert-space}, we imitate the induction
proof from Section~\ref{section:coin-betting-potentials}. The base case
$t=0$ is trivial, since both sides of the inequality are equal to $\epsilon$.
For $t \ge 1$, if we let $x = \sum_{i=1}^{t-1} g_i$, we have
\begin{align*}
\Wealth_t
&= \langle g_t, w_t \rangle + \Wealth_{t-1}
= \left(1 + \beta_t \frac{\langle g_t, x \rangle}{\norm{x}} \right) \Wealth_{t-1} \\
&\ge \left(1 + \beta_t \frac{\langle g_t, x \rangle}{\norm{x}} \right) F_{t-1}(\norm{x})
\stackrel{\text{\textbf{(?)}}}{\ge} F_t(\norm{x + g_t})
= F_t\left(\norm{\sum_{i=1}^t g_i} \right) \; .
\end{align*}
The only non-trivial inequality is marked with a question mark. The inequality
is the content of the Lemma~\ref{lemma:recursion_hilbert}, whose proof is in
Appendix~\ref{section:hilbert-space-reduction}. The proof relies mainly on
property (4) of Definition~\ref{definition:potential}.

\begin{lemma}
\label{lemma:recursion_hilbert}
Let $\{F_t\}_{t=0}^\infty$ be a sequence of excellent coin-betting potentials.
Let $g_1, g_2, \dots, g_t$ be vectors in a Hilbert space $\H$ such that
$\norm{g_1}, \norm{g_2}, \dots, \norm{g_t} \le 1$. Let $\beta_t$
be defined by \eqref{equation:potential-based-strategy-hilbert-space}
and let $x = \sum_{i=1}^{t-1} g_i$. Then,
$$
\left(1 + \beta_t \frac{\langle g_t, x \rangle}{\norm{x}} \right) F_{t-1}(\norm{x})
\ge F_t(\norm{x + g_t}) \; .
$$
\end{lemma}

This establishes \eqref{equation:wealth-lower-bound-hilbert-space},
from which we immediately have a reward lower bound
\begin{equation}
\label{equation:hilbert-space-olo-reward-lower-bound}
\Reward_T
= \sum_{t=1}^T \langle g_t, w_t \rangle
= \Wealth_T \ - \ \epsilon
\ge F_T\left(\norm{\sum_{t=1}^T g_t} \right) \ - \ \epsilon \; .
\end{equation}
We apply Lemma~\ref{lemma:reward-regret} to the function $F(x) = F_T(\norm{x}) -
\epsilon$ and we are almost done. The only remaining property we need is that if
$F$ is an even function then Fenchel conjugate of $F(\norm{\cdot})$ is
$F^*(\norm{\cdot})$; see \citet[Example 13.7]{BauschkeC2011}.
\end{proof}

\subsection{From Coin Betting to Learning with Expert Advice}
\label{section:reduction-experts}

In this section, we show how to use the algorithm for OLO over one-dimensional
Hilbert space $\R$ from
Section~\ref{section:one-dimensional-hilbert-space-olo}--which is itself based
on a coin-betting strategy---to construct an algorithm for \ac{LEA}.

Let $N \ge 2$ be the number of experts and $\Delta_N$ be the $N$-dimensional
probability simplex. Let $\pi = (\pi_1, \pi_2, \dots, \pi_N) \in \Delta_N$ be
any \emph{prior} distribution. Let $A$ be an algorithm for OLO over
one-dimensional Hilbert space $\R$, based on a sequence of coin-betting
potentials $\{F_t\}_{t=0}^\infty$ with initial endowment\footnote{Any initial
endowment $\epsilon > 0$ can be rescaled to $1$. Instead of $F_t(x)$ we would
use $F_t(x)/\epsilon$. The $w_t$ would become $w_t/\epsilon$, but $p_t$ is
invariant to scaling of $w_t$. Hence, the LEA algorithm is the same regardless
of $\epsilon$.} $1$. We instantiate $N$ copies of $A$.

Consider any round $t$. Let $w_{t,i} \in \R$ be the prediction of $i$-th copy of
$A$. The LEA algorithm computes $\widehat p_t = (\widehat p_{t,1}, \widehat
p_{t,2}, \dots, \widehat p_{t,N}) \in \R_{0,+}^N$,
\begin{equation}
\label{eq:phat}
\widehat p_{t,i} = \pi_i \cdot [w_{t,i}]_+
\end{equation}
where $[x]_+ = \max\{0,x\}$ is the positive part of $x$. Then, the LEA
algorithm predicts $p_t = (p_{t,1}, p_{t,2}, \dots, p_{t,N}) \in \Delta^N$,
\begin{equation}
\label{eq:preds_experts}
p_t = \tfrac{\widehat p_t}{\norm{\widehat p_t}_1} \; .
\end{equation}
If $\norm{\widehat p_t}_1 = 0$, the algorithm predicts the prior $\pi$.
Then, the algorithm receives the reward vector
$g_t = (g_{t,1}, g_{t,2}, \dots, g_{t,N}) \in [0,1]^N$. Finally, it
feeds the reward to each copy of $A$. The reward for $i$-th copy is $\widetilde g_{t,i} \in
[-1,1]$ defined as
\begin{align}
\label{eq:gradients_experts_reduction}
\widetilde g_{t,i} =
\begin{cases}
g_{t,i} - \langle g_t, p_t \rangle & \text{if } w_{t,i} > 0 \; , \\
\left[g_{t,i} - \langle g_t, p_t \rangle \right]_+ & \text{if } w_{t,i} \le 0 \; .
\end{cases}
\end{align}

The construction above defines a \ac{LEA} algorithm defined by the predictions
$p_t$, based on the algorithm $A$.  We can prove the following regret bound for
it.
%
\begin{theorem}[Regret Bound for Experts]
\label{theorem:regret-bound-experts}
Let $A$ be an algorithm for \ac{OLO} over
one-dimensional Hilbert space $\R$, based on coin-betting potential
$\{F_t\}_{t=0}^\infty$ for initial
endowment $1$. Let $f_t^{-1}$ be in the inverse of $f_t(x) = \ln(F_t(x))$
restricted to $[0,\infty)$.  Then, the regret of the \ac{LEA}
algorithm with prior $\pi \in \Delta_N$ that predicts at each round with $p_t$ in \eqref{eq:preds_experts} satisfies
\[
\forall T \ge 0 \quad \forall u \in \Delta_N \qquad \qquad
\Regret_T(u) \le f_T^{-1}\left( \KL{u}{\pi} \right) \; .
\]
\end{theorem}
The proof, in Appendix~\ref{section:appendix-expert-reduction}, is based on the
fact that \eqref{eq:phat}--\eqref{eq:gradients_experts_reduction} guarantee
that $\sum_{i=1}^N \pi_i \widetilde g_{t,i} w_{t,i} \le 0$ and on a variation
of the change of measure lemma used in the PAC-Bayes literature,
e.g.~\cite{McAllester-2013}.

\section{Applications of Krichevsky-Trofimov Estimator}
\label{section:kt-estimator}

In the previous two sections we have shown that a coin-betting potential with a
guaranteed rapid growth of the wealth will give good regret guarantees for
\ac{OLO} and \ac{LEA}. In this section we show that the optimal
Krichevsky-Trofimov (KT) estimator has associated a sequence of excellent
coin-betting potentials, which we call \emph{KT potentials}. We then prove
corollaries of the regret bounds for \ac{OLO} over Hilbert space and \ac{LEA}
that we have proved in previous sections, obtaining optimal regret bounds.

The potential corresponding to adaptive Kelly betting strategy
$\beta_t$ defined by \eqref{equation:kt-estimator-betting-strategy}
based on the KT estimator is
\begin{equation}
\label{equation:kt-estimator-potential}
F_t(x) = \epsilon \tfrac{2^t \cdot \Gamma \left( \tfrac{t+1}{2} + \frac{x}{2} \right) \cdot \Gamma \left( \tfrac{t+1}{2} - \frac{x}{2} \right)}{\pi \cdot t!}
\qquad \qquad \text{$t \ge 0$, \quad $x \in \left(-t-1, t+1\right)$,}
\end{equation}
where $\Gamma(x) = \int_0^\infty t^{-x} e^{-t} dt$ is Euler's gamma function.
The potential was introduced by~\citet{Krichevsky-Trofimov-1981} who used it
for proving regret bound for online prediction with log-loss; see also
\cite[Section 9.7]{Cesa-Bianchi-Lugosi-2006}.
Theorem~\ref{theorem:kt-potential} stated in
Appendix~\ref{section:properties-kt-potential} shows that
\eqref{equation:kt-estimator-potential} is a sequence of excellent coin-betting
potentials for initial endowment $\epsilon$. Theorem~\ref{theorem:kt-potential}
also shows that the KT betting strategy $\beta_t$ as defined by
\eqref{equation:kt-estimator-betting-strategy}, is a potential-based strategy,
i.e., it satisfies \eqref{equation:potential-based-strategy}.

\subsection{OLO in Hilbert Space}
\label{section:kt-olo}

\begin{algorithm}[t]
\caption{Algorithm for OLO over Hilbert space $\H$ based on KT potential
\label{algorithm:kt-hilbert-space-olo}}
\begin{algorithmic}[1]
{
\REQUIRE{Initial endowment $\epsilon > 0$}
\FOR{$t=1,2,\dots$}
\STATE{Predict with $w_t \leftarrow \tfrac{1}{t} \left(\epsilon + \sum_{i=1}^{t-1} \langle g_i, w_i \rangle \right) \sum_{i=1}^{t-1} g_i$}
\STATE{Receive reward vector $g_t \in \H$ such that $\norm{g_t} \le 1$}
\ENDFOR
}
\end{algorithmic}
\end{algorithm}

We apply the KT potential for the construction of an OLO algorithm over a Hilbert
space $\H$. According to \eqref{equation:hilbert-space-olo}, the resulting algorithm predicts
in round $t$,
\[
w_t = \beta_t \Wealth_{t-1} \tfrac{\sum_{i=1}^{t-1} g_i}{\norm{\sum_{i=1}^{t-1} g_i}}
\]
where $\beta_t$ is defined by
\eqref{equation:potential-based-strategy-hilbert-space}. According to
Theorem~\ref{theorem:kt-potential} in Appendix
\ref{section:properties-kt-potential}, the formula for $\beta_t$ simplifies to
$\beta_t = \frac{\norm{\sum_{i=1}^{t-1} g_i}}{t}$. Hence, the prediction can be
simply written as
\[
w_t
= \tfrac{1}{t} \Wealth_{t-1} \sum_{i=1}^{t-1} g_i
= \tfrac{1}{t} \left(\epsilon + \sum_{i=1}^{t-1} \langle g_i, w_i \rangle \right) \sum_{i=1}^{t-1} g_i \; .
\]
The algorithm is stated as Algorithm~\ref{algorithm:kt-hilbert-space-olo}.  We
derive a regret bound for it as a very simple corollary of
Theorem~\ref{theorem:hilbert-space-olo-regret-bound} to the KT potential
\eqref{equation:kt-estimator-potential}.  \begin{corollary}[Regret Bound for
Algorithm~\ref{algorithm:kt-hilbert-space-olo}]
\label{corollary:kt-hilbert-space-olo-regret} Let $\epsilon > 0$. Let
$\{g_t\}_{t=1}^\infty$ be any sequence of reward vectors in a Hilbert space
$\H$ such that $\norm{g_t} \le 1$.
Algorithm~\ref{algorithm:kt-hilbert-space-olo} satisfies
\[
\forall \, T \ge 0 \quad
\forall u \in \H \qquad \qquad
\Regret_T(u) \le \norm{u} \sqrt{T \ln\left(1 + \tfrac{4T^2 \norm{u}^2}{\epsilon^2} \right)} + \epsilon \left(1 - \tfrac{1}{2\sqrt{T}} \right) \;.
\]
\end{corollary}
The proof can be found in the Appendix~\ref{section:corollaries_reductions}. The only technical
part of the proof is an upper bound on Fenchel conjugate $F_t^*$ since it cannot
be expressed as an elementary function.

\subsection{Learning with Expert Advice}
\label{section:kt-lea}

\begin{algorithm}[t]
\begin{algorithmic}[1]
\caption{Algorithm for Learning with Expert Advice based on shifted KT potential
\label{algorithm:kt-experts}}
{
\REQUIRE{Number of experts $N$, number of rounds $T$, prior distribution $\pi \in \Delta_N$}
\FOR{$t=1,2,\dots,T$}
\STATE{For each $i \in [N]$, set $w_{t,i} \leftarrow \tfrac{\sum_{j=1}^{t-1} \widetilde g_{j,i}}{t+T/2} \left(1 + \sum_{j=1}^{t-1} \widetilde g_{j,i} w_{j,i} \right)$}
\STATE{For each $i \in [N]$, set $\widehat{p}_{t,i} \leftarrow \pi_i [w_{t,i}]_+$}
\STATE{Predict with $p_t \leftarrow
\begin{cases}
\widehat{p}_t/\norm{\widehat{p_t}}_1 & \text{if $\norm{\widehat p_t}_1 > 0$} \\
\pi & \text{if $\norm{\widehat p_t}_1 = 0$}
\end{cases}$}
\STATE{Receive loss vector $\ell_t \in [0,1]^N$}
\STATE{For each $i \in [N]$, set $\widetilde g_{t,i} \leftarrow \begin{cases}
g_{t,i} - \langle g_t, p_t \rangle & \text{if $w_{t,i} > 0$} \\
[g_{t,i} - \langle g_t, p_t \rangle]_+ & \text{if $w_{t,i} \le 0$}
\end{cases}$}
\ENDFOR
}
\end{algorithmic}
\end{algorithm}

We will now construct an algorithm for \ac{LEA} based on \emph{shifted KT
potential}. The shifted potential and the resulting algorithm requires to know
the number of rounds $T$ in advance. The shifted KT potential is defined as
\[
F_t(x) = \tfrac{2^t \cdot \Gamma\left(T/2 + 1 \right) \cdot \Gamma\left(\tfrac{t+T/2+1}{2} + \frac{x}{2} \right) \cdot \Gamma\left(\tfrac{t+T/2+1}{2} - \frac{x}{2} \right)}{\Gamma\left(\tfrac{T/2+1}{2} \right)^2 \cdot \Gamma \left(t+T/2+1\right)} \; .
\]
The reason for its name is that, up to a multiplicative constant, $F_t$ is
equal to the KT potential shifted in time by $T/2$, i.e., $t$ is replaced by
$T/2+t$.  According to Theorem~\ref{theorem:kt-potential} in Appendix
\ref{section:properties-kt-potential}, the shifted KT potentials also form a
sequence of coin-betting potentials for initial endowment $1$. Furthermore, the
corresponding betting fraction is
\[
\beta_t = \tfrac{\sum_{j=1}^{t-1} \widetilde g_j}{T/2+t} \; .
\]
Recall that for construction of the final algorithm, we need, as an
intermediate step, an OLO algorithm for one-dimensional Hilbert space $\R$.
This algorithm predicts for any sequence $\{\widetilde g_t\}_{t=1}^\infty$ of reward
vectors,
\[
w_t
= \beta_t \Wealth_{t-1}
= \beta_t \left(1 + \sum_{j=1}^{t-1} \widetilde g_j w_j \right)
= \frac{\sum_{i=1}^{t-1} \widetilde g_i}{T/2+t} \left(1 + \sum_{j=1}^{t-1} \widetilde g_j w_j \right) \; .
\]
Following the construction in Section~\ref{section:reduction-experts}, we
arrive at the final algorithm, Algorithm~\ref{algorithm:kt-experts}.

We can derive a regret bound for Algorithm~\ref{algorithm:kt-experts} by
applying Theorem~\ref{theorem:regret-bound-experts} to the shifted KT
potential.  \begin{corollary}[Regret Bound for
Algorithm~\ref{algorithm:kt-experts}] \label{corollary:kt-experts-regret} Let
$N \ge 2$ and $T \ge 0$ be integers. Let $\pi \in \Delta_N$ be a prior.  For
any sequence $\ell_1, \ell_2, \dots, \ell_T \in [0,1]^N$ of loss vectors,
Algorithm~\ref{algorithm:kt-experts} with input $N,T,\pi$ satisfies
\[
\forall u \in \Delta_N \qquad \qquad \Regret_T(u) \le \sqrt{3T (4 + \KL{u}{\pi})} \; .
\]
\end{corollary}
The proof of the corollary is in the
Appendix~\ref{section:corollaries_reductions}.  The technical part of the proof
is an upper bound on $f_t^{-1}(x)$, which we conveniently do by lower bounding
$F_t(x)$.

The reason for using the shifted potential comes from the analysis of
$f_t^{-1}(x)$. The unshifted algorithm would have a $O(\sqrt{T (\log T +
\KL{u}{\pi}})$ regret bound; the shifting improves the bound to $O(\sqrt{T (1 +
\KL{u}{\pi}})$.  By changing $T/2$ in Algorithm~\ref{algorithm:kt-experts} to
another constant fraction of $T$, it is possible to trade-off between the two
constants $3$ and $4$ present in the square root.

The requirement of knowing the number of rounds $T$ in advance can be lifted by
the standard doubling trick~\citep[Section 2.3.1]{Shalev-Shwartz-2011}. We
obtain an anytime algorithm at the expense of slightly worse regret bound,
\[
\forall \, T \ge 0 \quad \forall u \in \Delta_N \qquad \qquad
\Regret_T(u) \le \tfrac{\sqrt{2}}{\sqrt{2} - 1} \sqrt{3T (4 + \KL{u}{\pi})} \; .
\]

Also, as observed by \citet{Chernov-Vovk-2010}, bounds in terms of the KL
divergence are superior to the $\epsilon$-quantile bounds.

\section{Discussion of the Results and Related Work}
\label{sec:discussion}

The interpretation of parameter-free algorithms as coin-betting algorithms is
new. This interpretation, far from being just a mathematical gimmick, reveals
the common hidden structure of previous parameter-free algorithms. For example,
it is clear now that the characteristic of parameter-freeness is just a
consequence of having an algorithm that guarantees the maximum reward possible.
In this sense, all the online learning algorithms requiring parameter tuning
are just guaranteeing a suboptimal wealth growth. The concept of
``learning rate'' becomes questionable in the light of the presented results.
At the same time, the previous ad-hoc choices of the potentials were just
approximations of the optimal obtainable wealth.

In particular, in previous \ac{LEA} papers the potential used for an expert at
time $t$ is $\exp \left(\frac{([x]_+)^2}{t} \right)$; see
\citep{ChaudhuriYH09,LuoE14,LuoS15}. It is now clear that that potential is
just an approximation of the optimal wealth achievable by a coin-betting
algorithm.  Moreover, that potential and the ones
by~\citet{ChernovV10} and \cite{KoolenE15} are not even, which introduces a lot of
technical difficulties in the analysis. Instead, our reduction moves the
truncation outside of the potential (see equation
\eqref{eq:gradients_experts_reduction}) making the analysis straightforward.

In the \ac{OLO} setting, \citet{StreeterM12} used a one-dimensional potential
of the form $\exp \left(\frac{|x|}{\sqrt{t}}\right)$, the same base potential
as in \ac{EG}~\citep{KivinenW97}, obtaining a suboptimal regret
bound.  \citet{Orabona13} extended it to the infinite dimensional case,
unveiling the connection with \ac{EG}.  Notice that the same suboptimality is
present in \ac{EG}: the learning rate has to be tuned with the knowledge of the
number of experts.  The optimal bound has been obtained in \citet{McMahanO14}
with potential of the right form $\exp \left(\frac{x^2}{t}\right)$. Indeed, it
is very easy to show that this potential is an excellent coin betting
potential.

The obtained bounds, in Corollaries~\ref{corollary:kt-hilbert-space-olo-regret}
and~\ref{corollary:kt-experts-regret}, improve the previous ones and/or
correspond to simpler algorithms.  In particular, the only known bound for
\ac{LEA} without a $\ln(\ln(T))$ is due to \citet{ChernovV10} for a prediction
strategy that does not have a closed form.  Moreover, the reductions make
possible to transfer any advancement on the problem of coin-betting to \ac{OLO}
and \ac{LEA}. Notice that since the adaptive Kelly strategy based on \ac{KT}
estimator is very close to optimal, the only possible improvement is to have a
data-dependent bound, for example like the ones in~\cite{KoolenE15}. Indeed, it
is very easy to extend the proof of our reductions to hold in the
data-dependent case as well. For example, it is an easy exercise to show that a
potential of the form $\exp \left(\frac{x^2}{1+\sum_{i=1}^{t-1}
\|g_{i}\|}\right)$ is an excellent coin betting potential. Through our
reductions, such potential would recover at the same time the bounds in
\citet{LuoS15} and \citet{Orabona14}. It is an open problem how to design coin
betting potentials of the form $\exp \left(\frac{x^2}{1+\sum_{i=1}^{t-1}
\|g_{i}\|^2}\right)$, that would allow to recover immediately the bounds
in~\citet{KoolenE15} and the optimistic bounds for smooth losses in
\ac{OCO}~\citep{SrebroST10} with a parameter-free algorithm (see the discussion
in \citet{Orabona14}).

Moreover, as already proved in previous papers, the existence of parameter-free
algorithms have broad consequences, beyond online learning. For example,
\citet{LuoS15} prove that a parameter-free expert algorithm can be used to
design an efficient algorithm that predicts as the best pruning tree. In the
context of risk minimization over Lipschitz convex losses in an infinite
dimensional Hilbert space, \citet{Orabona14} proved that the parameter-free
\ac{OLO} can be used to obtain risk bound guarantees that compete with the
regularized \acl{ERM} solution with oracle tuning of the regularizer. In
simpler words, the parameter-free
Algorithm~\ref{algorithm:kt-hilbert-space-olo} can be used to obtain, for
example, the same risk guarantee of a kernel \acl{SVM} with optimal (unknown)
tuning of the regularizer.

Regarding the tightness of the reductions, it is easy to see that they are in a
certain sense optimal. In fact, the obtained
Algorithms~\ref{algorithm:kt-hilbert-space-olo} and~\ref{algorithm:kt-experts}
achieves the optimal worst case upper bound on regret, see \citet{Orabona13} and
\citet{Cesa-BianchiL06} respectively. Also, it is easy to use the  reduction
from Section~\ref{section:reduction-experts} to derive an $\Omega(\norm{u}
\sqrt{T\log(1+\norm{u})})$ lower bound for \ac{OLO} over (one-dimensional)
Hilbert space from the well known $\Omega(\sqrt{T \log N})$ lower bound for
\ac{LEA}.


% Acknowledgments: Comment out in anonymous submission.
% Uncomment for camera-ready version.

%\begin{small}
\textbf{Acknowledgments.} The authors thank Jacob Abernethy, Nicol\`{o}
Cesa-Bianchi, Satyen Kale, Chansoo Lee, and  Giuseppe Molteni for useful
discussions on this work.
%\end{small}

\vspace{-0.38cm}

\begin{small}
\setlength{\bibsep}{2pt}
\bibliographystyle{plainnat}
\bibliography{learning}
\end{small}

\appendix
\section{From Log Loss to Wealth}
\label{section:logloss-to-wealth}

Guarantees for betting or sequential investement algorithm are often expressed
as upper bounds on the regret with respect to the log loss.  Here, for the sake
of completeness, we show how to convert such a guarantee to a lower bound on
the wealth of the corresponding betting algorithm.

We consider the problem of predicting a binary outcome.  The algorithm predicts
at each round probability $p_t \in [0,1]$. The adversary generates a sequences
of outcomes $x_t \in \{0,1\}$ and the algorithm's loss is
\[
\ell(p_t,x_t) = -x_t \ln p_t -(1-x_t) \ln (1-p_t) \; .
\]
We define the regret with respect to a fixed probability vector $\beta$ as
\[
\Regret^{\mathrm{logloss}}_T = \sum_{t=1}^T \ell(p_t,x_t) - \min_{\beta \in [0,1]} \sum_{t=1}^T \ell(\beta,x_t) \; .
\]

\begin{lemma}
Assume that an algorithm that predicts $p_t$ guarantees
$\Regret^{\mathrm{logloss}}_T \leq R_T$.  Then, the coin betting strategy with
endowement $\epsilon$ and $\beta_t = 2 p_{t}-1$ guarantees
\[
\Wealth_T \ge \epsilon \exp\left(T \cdot \KL{\frac{1}{2} + \frac{\sum_{t=1}^T g_t}{2 T}}{\frac{1}{2}} - R_T \right)
\]
against any sequence of outcomes $g_t \in [-1,+1]$.
\end{lemma}

\begin{proof}
Define $x_t=\tfrac{1+g_t}{2}$. We have
\begin{align*}
\ln \Wealth_T
& = \ln (\Wealth_{t-1} + w_t g_t) \\
& = \ln (\Wealth_{t-1}(1 + g_t \beta_t))\\
& = \ln \epsilon \prod_{t=1}^T (1 + g_t\beta_t) \\
& = \ln \epsilon + \sum_{t=1}^T \ln (1 + g_t\beta_t)\\
& \ge \ln \epsilon +  \sum_{t=1}^T \left( \frac{1+g_t}{2} \right) \ln \left(1 + \beta_t\right) + \left( \frac{1-g_t}{2} \right) \ln \left(1 - \beta_t \right) \\
& =  \ln \epsilon + \sum_{t=1}^T \left( \frac{1+g_t}{2} \right) \ln \left(2p_t \right) + \left( \frac{1-g_t}{2} \right) \ln \left(2 (1 - p_t) \right) \\
& =  \ln \epsilon + T \ln(2) + \sum_{t=1}^T \left( \frac{1+g_t}{2} \right) \ln (p_t) + \left( \frac{1-g_t}{2} \right) \ln (1 - p_t) \\
& =  \ln \epsilon + T \ln(2) - \sum_{t=1}^T \ell(p_t, x_t) \\
& =  \ln \epsilon + T \ln(2) - \Regret^{\mathrm{logloss}}_T - \min_{\beta \in [0,1]} \sum_{t=1}^T \ell(\beta,x_t) \\
& \ge  \ln \epsilon + T \ln(2) - R_T - \min_{\beta \in [0,1]} \sum_{t=1}^T \ell(\beta,x_t) \; ,
\end{align*}
where the first inequality is due to the concavity of $\ln$ and the second one
is due to the assumption of the regret.

It is easy to see that the $\beta^*=\argmin_{\beta \in [0,1]} \sum_{t=1}^T
\ell(\beta,x_t)=\tfrac{\sum_{t=1}^T x_t}{T}$. Hence, we have
\[
\min_{\beta \in [0,1]} \sum_{t=1}^T \ell(\beta,x_t) = T \left( - \beta^* \ln \beta^* - (1-\beta^*) \ln (1-\beta^*)\right) \; .
\]
Also, we have that for any $\beta \in [0,1]$
\[
- \beta \ln \beta - (1-\beta) \ln (1-\beta) = - \KL{\beta}{\frac{1}{2}} + \ln 2 \; .
\]

Putting all together, we have the stated lemma.
\end{proof}

The lower bound on the wealth of the adaptive Kelly betting based on the KT
estimator is obtained simply by the stated Lemma and reminding that the log
loss regret of the KT estimator is upper bounded by $\frac{1}{2}\ln T + \ln 2$.

\section{Optimal Betting Fraction}
\label{section:optimal-betting-fraction}

\begin{theorem}[Optimal Betting Fraction]
\label{theorem:optimal-betting-fraction}
Let $x \in \R$. Let $F:[x-1,x+1]$ be a logaritmically convex function. Then,
\[
\argmin_{\beta \in (-1,1)} \max_{g \in [-1,1]} \ \frac{F(x+g)}{1 + \beta g}
= \frac{F(x+1) - F(x-1)}{F(x+1) + F(x-1)} \; .
\]
\end{theorem}

\begin{proof}
We define the functions $h,f:[-1,1] \times (-1,1) \to \R$ as
\begin{align*}
h(g, \beta) & = \frac{F(x+g)}{1 + \beta g} &
& \text{and} &
f(g, \beta) & = \ln (h(g,\beta)) = \ln(F(x+g)) - \ln(1 + \beta g) \; .
\end{align*}
Clearly, $\argmin_{\beta \in (-1,1)} \max_{g \in [-1,1]} h(g,\beta) = \argmin_{\beta \in (-1,1)} \max_{g \in [-1,1]} f(g,\beta)$
and we can work with $f$ instead of $h$. Function $h$ is logarithmically convex
in $g$ and thus $f$ is convex in $g$. Therefore,
\[
\forall \beta \in (-1,1) \qquad \qquad
\max_{g \in [-1,1]} f(g,\beta) = \max \left\{ f(+1,\beta), f(-1,\beta) \right\} \; .
\]
Let $\phi(\beta) = \max \left\{ f(+1,\beta), f(-1,\beta) \right\}$. We seek to
find $\argmin_{\beta \in (-1,1)} \phi(\beta)$. Since $f(+1,\beta)$ is decreasing
in $\beta$ and $f(-1,\beta)$ is increasing in $\beta$, the minimum of
$\phi(\beta)$ is at a point $\beta^*$ such that $f(+1,\beta^*) = f(-1,\beta^*)$.
In other words, $\beta^*$ satisfies
\[
\ln(F(x+1)) - \ln(1 + \beta^*) =  \ln(F(x-1)) - \ln(1 - \beta^*) \; .
\]
The only solution of this equation is
\[
\beta^* = \frac{F(x+1) - F(x-1)}{F(x+1) + F(x-1)} \; .
\]
\end{proof}

\section{Proof of Lemma~\ref{lemma:recursion_hilbert}}
\label{section:hilbert-space-reduction}

First we state the following Lemma from~\cite{McMahanO14} and reported here with our notation for completeness.
\begin{lemma}[Extremes]
\label{lemma:extremes}
Let $h:(-a,a) \to \R$ be an even concave twice-differentiable function that
satisfies $x \cdot h''(x) \le h'(x)$ for all $x \in [0,a)$. Then, if vectors
$u,v \in \H$ satisfy $\|u\| + \|v\| < a$, then
\begin{equation}
\label{equation:lemma-extremes-1}
\langle u, v \rangle + h(\|u + v\|) \ge \min \left\{ \|u\| \cdot \|v\| + h(\|v\| + \|u\|), \ - \|u\| \cdot \|v\| + h(\|u\| - \|v\|) \right\} \; .
\end{equation}
Futhermore, for any $b \in [0,a]$ and any $u, v \in \H$ such that $\|v\| < a - b$ and $\|u\| \le b$,
\begin{equation}
\label{equation:lemma-extremes-2}
\langle u, v \rangle + h(\|u + v\|) \ge \min \left\{ b \cdot \|v\| + h(\|v\| + b), \ - b \cdot \|v\| + h(\|u\| - b) \right\} \; .
\end{equation}
\end{lemma}
%
\begin{proof}
If $u$ or $v$ is zero, the inequality \eqref{equation:lemma-extremes-1} clearly holds. From now on we assume that
$u,v$ are non-zero. Let $c$ be the cosine of the angle of between $u$ and $v$.
More formally,
$$
c = \frac{\langle u, v \rangle}{\|u\| \cdot \|v\|} \; .
$$
With this notation, the left-hand side is
$$
\langle u, v \rangle + h(\|u + v\|) = c \|u\| \cdot \|v\|  + h(\sqrt{\|u\|^2 + \|v\|^2 + 2 c \|u\| \cdot \|v\|}) \; .
$$
We consider the last expression as a function of $c$ and we call it $f(c)$. The
inequality \eqref{equation:lemma-extremes-1} is equivalent to
$$
\forall c \in [-1,1] \qquad \qquad f(c) \ge \min \left\{f(+1), f(-1)\right\} \; .
$$
The last inequality is clearly true if $f:[-1,1] \to \R$ is concave. We now
check that $f$ is indeed concave, which we prove by showing that the second
derivative is non-positive. The first derivate of $f$ is
$$
f'(c) = \|u\| \cdot \|v\| + \frac{h'(\sqrt{\|u\|^2 + \|v\|^2 + 2 c \|u\| \cdot \|v\|}) \cdot \|u\| \cdot \|v\|}{\sqrt{\|u\|^2 + \|v\|^2 + 2 c \|u\| \cdot \|v\|}} \; .
$$
The second derivative of $f$ is
\begin{align*}
&f''(c) = \|u\|^2 \cdot \|v\|^2 \times \\
&\qquad \frac{h''(\sqrt{\|u\|^2 + \|v\|^2 + 2 c \|u\| \cdot \|v\|})  - \frac{1}{\sqrt{\|u\|^2 + \|v\|^2 + 2c \|u\| \cdot \|v\|}} \cdot h'(\sqrt{\|u\|^2 + \|v\|^2 + 2 c \|u\| \cdot \|v\|})  }{\|u\|^2 + \|v\|^2 + 2 c \|u\| \cdot \|v\|} \; .
\end{align*}
If we consider $x=\sqrt{\|u\|^2 + \|v\|^2 + 2 c \|u\| \cdot \|v\|}$, the
assumption $x \cdot h''(x) \le h'(x)$ implies that $f''(c)$ is non-positive.
This finishes the proof of the inequality \eqref{equation:lemma-extremes-1}.

Inequality \eqref{equation:lemma-extremes-2}, can be derived from inequality
\eqref{equation:lemma-extremes-1} as follows
\begin{align*}
\langle u, v \rangle + h(\|u + v\|)
& \ge \min_{u \in \H : \|u\| \le b} \min \left\{ \|u\| \cdot \|v\| + h(\|v\| + b), \ - \|u\| \cdot \|v\| + h(\|u\| - \|v\|) \right\} \\
& = \min_{z \in [0,b]} \min \left\{ z \cdot \|v\| + h(\|v\| + z), \ - z \cdot \|v\| + h(\|u\| - z) \right\} \\
& = \min_{z \in [-b,b]} z \cdot \|v\| + h(\|v\| + z) \\
& = \min \left\{ b \cdot \|v\| + h(\|v\| + b), \ - b \cdot \|v\| + h(\|u\| - b) \right\} \; .
\end{align*}
The inequality in the chain is the inequality \eqref{equation:lemma-extremes-1}.
The last equality follows since $g(z) = z \cdot \|v\| + h(\|v\| + z)$
is concave.
\end{proof}

We can now prove Lemma~\ref{lemma:recursion_hilbert}.
\begin{proof}[Proof of Lemma~\ref{lemma:recursion_hilbert}]
\begin{align*}
&\left(1 + \beta_t \frac{\langle g_t, x \rangle}{\|x\|} \right) F_{t-1}(\norm{x}) - F_t(\norm{x + g_t}) \\
&\quad \geq \min_{r \in \{-1,1\}} \left(1 + \beta_t r \norm{g_t} \right) F_{t-1}(\norm{x}) - F_t(\norm{x} + r \norm{g_t}) \\
&\quad \geq 0
\end{align*}
where the first inequality comes from Lemma~\ref{lemma:extremes}, the second one from the property 3 of a coin-betting potential.
\end{proof}

\section{Proof of Theorem~\ref{theorem:regret-bound-experts}}
\label{section:appendix-expert-reduction}

\begin{proof}
We first prove that $\sum_{i=1}^N \pi_i \widetilde g_{t,i} w_{t,i} \le 0$. Indeed,
\begin{align*}
\sum_{i=1}^N \pi_i \widetilde g_{t,i} w_{t,i}
& = \sum_{i \, : \, \pi_i w_{t,i} > 0} \pi_i [w_{t,i}]_+ (g_{t,i} - \langle g_t, p_t \rangle)  \ + \ \sum_{i \, : \, \pi_i w_{t,i} \le 0} \pi_i w_{t,i} [g_{t,i} - \langle g_t, p_t \rangle ]_+ \\
& = \norm{\widehat p_t}_1 \sum_{i=1}^N p_{t,i} (g_{t,i} - \langle g_t, p_t \rangle)  \ + \ \sum_{i \, : \, \pi_i w_{t,i} \le 0} \pi_i w_{t,i} [g_{t,i} - \langle g_t, p_t\rangle]_+ \\
& = 0 \ + \ \sum_{i \, : \, \pi_i w_{t,i} \le 0} \pi_i w_{t,i} [g_{t,i} - \langle g_t, p_t\rangle]_+
\ \le 0 \; .
\end{align*}
The first equality follows from definition of $g_{t,i}$. To see the second
equality, consider two cases: If $\pi_i w_{t,i} \le 0$ for all $i$ then
$\norm{\widehat p_t}_1 = 0$ and therefore both
$\norm{\widehat p_t}_1 \sum_{i=1}^N p_{t,i} (g_{t,i} - \langle g_t, p_t \rangle)$
and $\sum_{i \, : \, \pi_i w_{t,i} > 0} \pi_i [w_{t,i}]_+
(g_{t,i} - \langle g_t, p_t \rangle)$ are trivially zero.  If $\norm{\widehat p_t}_1 >
0$ then $\pi_i [w_{t,i}]_+ = \widehat p_{t,i} = \norm{\widehat p_t}_1 p_{t,i}$
for all $i$.

From the assumption on $A$, we have, for any sequence
$\{\widetilde g_t\}_{t=1}^\infty$ such that $\widetilde g_t \in [-1,1]$, satisfies
\begin{equation}
\label{equation:experts-one-dimensional-assumption}
\Wealth_t = 1 + \sum_{i=1}^t \widetilde g_i w_i \ge F_t\left(\sum_{i=1}^t \widetilde g_i\right) \; .
\end{equation}
Inequality $\sum_{i=1}^N \pi_i \widetilde g_{t,i} w_{t,i} \le 0$ and \eqref{equation:experts-one-dimensional-assumption} imply
\begin{equation}
\label{equation:bounded-potential}
\sum_{i=1}^N  \pi_i F_T \left(\sum_{t=1}^T \widetilde g_{t,i} \right)
\le 1 + \sum_{i=1}^N \pi_i \sum_{t=1}^T  \widetilde g_{t,i} w_{t,i} \le 1 \; .
\end{equation}
Now, let $\widetilde G_{t,i} =
\sum_{t=1}^T \widetilde g_{t,i}$. For any competitor $u \in \Delta_N$,
\begingroup
\allowdisplaybreaks
\begin{align*}
\allowdisplaybreaks
&\Regret_T(u)
= \sum_{t=1}^T \langle g_t, u - p_t \rangle
= \sum_{t=1}^T \sum_{i=1}^N u_i \left(g_{t,i} - \langle g_t, p_t \rangle \right) \\
& \le \sum_{t=1}^T \sum_{i=1}^N u_i \widetilde g_{t,i} \qquad \text{(by definition of $\widetilde g_{t,i}$)} \\
& \le \sum_{i=1}^N u_i \left|\widetilde G_{T,i}\right| \qquad \text{(since $u_i \ge 0, i=1,\ldots, N$)}  \\
& = \sum_{i=1}^N u_i f_T^{-1}\left(\ln [F_T(\widetilde G_{T,i})] \right)  \qquad \text{(since $F_T(x) = \exp(f_T(x))$ is even)} \\
& \le f_T^{-1}\left(\sum_{i=1}^N u_i \ln \left[ F_T(\widetilde G_{T,i}) \right]\right) \qquad \text{(by concavity of $f_T^{-1}$)} \\
& = f_T^{-1}\left(\sum_{i=1}^N u_i \left\{\ln \left[\frac{u_i}{\pi_i}\right] +\ln \left[ \frac{\pi_i}{u_i} F_T(\widetilde G_{T,i}) \right] \right\} \right)
= f_T^{-1}\left(\KL{u}{\pi}+\sum_{i=1}^N u_i\ln \left[\frac{\pi_i}{u_i} F_T(\widetilde G_{T,i}) \right]\right) \\
& \le f_T^{-1}\left(\KL{u}{\pi}+\ln \left(\sum_{i=1}^N \pi_i F_T(\widetilde G_{T,i}) \right)\right) \qquad \text{(by concavity of $\ln(\cdot)$)} \\
& \le f_T^{-1}\left(\KL{u}{\pi}\right) \qquad \text{(by \eqref{equation:bounded-potential})}. \qedhere
\end{align*}
\endgroup
\end{proof}

\section{Properties of Krichevsky-Trofimov Potential}
\label{section:properties-kt-potential}

\begin{lemma}[Analytic Properties of KT potential]
\label{lemma:kt-potential-analytic-properties}
Let $a > 0$. The function $F:(-a,a) \to \R_+$,
$$
F(x) = \Gamma(a+x) \Gamma(a-x)
$$
is even, logarithmically convex, strictly increasing on $[0,a)$, satisfies
$$
\lim_{x \nearrow a} F(x) = \lim_{x \searrow -a} F(x) = + \infty
$$
and
\begin{equation}
\label{equation:gamma-property-4}
\forall x \in (-a,a) \qquad x \cdot F''(x) \ge F'(x) \; .
\end{equation}
\end{lemma}

\begin{proof}
$F(x)$ is obviously even. $\Gamma(z) = \int_0^\infty t^{z-1} e^{-t} dt$ is
defined for any real number $z > 0$. Hence $F$ is defined on the interval
$(-a,a)$. According to Bohr-Mollerup theorem \cite[Theorem 2.1]{Artin64},
$\Gamma(x)$ is logarithmically convex on $(0,\infty)$. Hence $F(x)$ is also
logarithimically convex, since $\ln (F(x)) = \ln(\Gamma(a+x)) +
\ln(\Gamma(a-x))$ is a sum of convex functions.

It is well known that $\lim_{z \searrow 0} \Gamma(z) = +\infty$. Thus,
$$
\lim_{x \nearrow a} F(x)
= \lim_{x \nearrow a} \Gamma(a+x) \Gamma(a-x)
= \Gamma(2a) \lim_{x \nearrow a} \Gamma(a-x)
= \Gamma(2a) \lim_{z \searrow 0} \Gamma(z)
= + \infty \; ,
$$
since $\Gamma$ is continuous at $2a$. Because $F(x)$ is even, we also have
$\lim_{x \searrow -a} F(x) = +\infty$.

To show that $F(x)$ is increasing and that it satisfies
\eqref{equation:gamma-property-4}, we write $f(x) = \ln(F(x))$ as a Mclaurin series.
The derivatives of $\ln(\Gamma(z))$ are so called polygamma functions
$$
\psi^{(n)}(z) = \frac{d^{n+1}}{dz^{n+1}} \ln(\Gamma(z))
\qquad \qquad \text{for $z > 0$ and $n=0,1,2,\dots$.}
$$
Polygamma functions have well-known integral representation
$$
\psi^{(n)}(z) = (-1)^{n+1} \int_0^\infty \frac{t^n e^{-zt}}{1 - e^{-t}} dt
\qquad \qquad \text{for $z > 0$ and $n=1,2,\dots$.}
$$
Using polygama functions, we can write the Mclaurin series for $f(x) = \ln(F(x))$ as
$$
f(x) = \ln(F(x)) = \ln(\Gamma(a+x)) + \ln(\Gamma(a-x)) = 2 \ln(\Gamma(a)) + 2 \sum_{\substack{n \ge 2 \\ \text{$n$ even}}} \frac{\psi^{(n-1)}(a) x^n}{n!} \; .
$$
The series converges for $x \in (-a,a)$, since
for even $n \ge 2$, $\psi_{n-1}(a)$ is positive and can be upper bounded as
\begin{align*}
\psi^{(n-1)}(a)
& = \int_0^\infty \frac{t^n e^{-at}}{1 - e^{-t}} dt \\
& = \int_0^1 \frac{t^n e^{-at}}{1 - e^{-t}} dt + \int_1^\infty \frac{t^n e^{-zt}}{1 - e^{-t}} dt \\
& \le \int_0^1 \frac{t^n e^{-at}}{t(1 - 1/e)} dt + \int_1^\infty t^n e^{-at} dt \\
& \le \frac{1}{1 - 1/e} \int_0^\infty t^{n-1} e^{-at} dt + \int_0^\infty t^n e^{-at} dt \\
& = \frac{1}{1 - 1/e} a^{-n} \Gamma(n) + a^{-n-1} \Gamma(n+1) \\
& \le \frac{1}{1 - 1/e} a^{-n} n! (1 + 1/a) \; .
\end{align*}
From the Mclaurin expansion we see that $f(x)$ is increasing on $[0,a)$
since all coefficients are positive (except for zero order term).

Finally, to prove \eqref{equation:gamma-property-4}, note that
for any $x \in (-a,a)$,
$$
f(x) = c_0 + \sum_{n=2}^\infty c_n x^n
$$
where $c_2, c_3, \dots$ are non-negative coefficients. Thus
\begin{align*}
f'(x) & = \sum_{n=2}^\infty n c_n x^{n-1} &
& \text{and} &
f''(x) & = \sum_{n=2}^\infty n (n-1) c_n x^{n-2} \; .
\end{align*}
and hence $x \cdot f''(x) \ge f'(x)$ for $x \in [0,a)$. Since $F(x) = \exp(f(x))$,
\begin{align*}
F'(x) & = f'(x) \cdot F(x) &
& \text{and} &
F''(x) & = \left[f''(x) + (f'(x))^2 \right] \cdot F(x) \; .
\end{align*}
Therefore, for $x \in [0,a)$,
$$
x \cdot F''(x)
= x \left[ f''(x) + (f'(x))^2 \right] F(x)
\ge \left[ f'(x) + x (f'(x))^2 \right] F(x)
\ge f'(x) F(x) = F'(x) \; .
$$
This proves \eqref{equation:gamma-property-4}.
\end{proof}

\begin{theorem}[KT potential]
\label{theorem:kt-potential}
Let $\delta \ge 0$ and $\epsilon > 0$. The sequence of functions
$\{F_t\}_{t=0}^\infty$, $F_t:(-t-\delta-1, t+\delta+1) \to \R_+$ defined by
$$
F_t(x) = \epsilon \frac{2^t \cdot \Gamma(\delta + 1) \Gamma(\frac{t+\delta+1}{2} + \frac{x}{2}) \Gamma(\frac{t+\delta+1}{2} - \frac{x}{2})}{\Gamma(\frac{\delta+1}{2})^2 \Gamma(t+\delta+1)} \; .
$$
is a sequence of excellent coin-betting potentials for initial endowment $\epsilon$.
Furthermore, for any $x \in (-t-\delta-1, t+\delta+1)$,
\begin{equation}
\label{equation:kt-potential-beta}
\frac{F_t(x+1) - F_{t}(x-1)}{F_t(x+1) + F_{t}(x-1)} = \frac{x}{t+\delta} \; .
\end{equation}
\end{theorem}

\begin{proof}
Property (2) and (4) of the definition follow from
Lemma~\ref{lemma:kt-potential-analytic-properties}.
Property (1) follows by simple substitution for $t=0$ and $x=0$.

Before verifying property (3), we prove \eqref{equation:kt-potential-beta}. We
use an algebraic property of the gamma function that states that $\Gamma(1+z) =
z \Gamma(z)$ for any positive $z$. Equation \eqref{equation:kt-potential-beta}
follows from
\begin{align*}
\frac{F_t(x + 1) - F_t(x - 1)}{F_t(x + 1) + F_t(x - 1)}
& = \frac{\Gamma(\frac{t+\delta+2}{2} + \frac{x}{2}) \Gamma(\frac{t+\delta}{2} - \frac{x}{2}) - \Gamma(\frac{t+\delta}{2} + \frac{x}{2}) \Gamma(\frac{t+\delta+2}{2} - \frac{x}{2})}{\Gamma(\frac{t+\delta+2}{2} + \frac{x}{2}) \Gamma(\frac{t + \delta}{2} - \frac{x}{2}) + \Gamma(\frac{t + \delta}{2} + \frac{x}{2}) \Gamma(\frac{t+\delta+2}{2} - \frac{x}{2})} \\
& = \frac{(\frac{t+\delta}{2} + \frac{x}{2})\Gamma(\frac{t+\delta}{2} + \frac{x}{2}) \Gamma(\frac{t+\delta}{2} - \frac{x}{2}) - (\frac{t+\delta}{2} - \frac{x}{2})\Gamma(\frac{t+\delta}{2} + \frac{x}{2}) \Gamma(\frac{t+\delta}{2} - \frac{x}{2})}{(\frac{t+\delta}{2} + \frac{x}{2})\Gamma(\frac{t+\delta}{2} + \frac{x}{2}) \Gamma(\frac{t+\delta}{2} - \frac{x}{2}) + (\frac{t+\delta}{2} - \frac{x}{2})\Gamma(\frac{t+\delta}{2} + \frac{x}{2}) \Gamma(\frac{t+\delta}{2} - \frac{x}{2})} \\
& = \frac{(\frac{t+\delta}{2} + \frac{x}{2}) - (\frac{t+\delta}{2} - \frac{x}{2})}{(\frac{t+\delta}{2} + \frac{x}{2}) + (\frac{t+\delta}{2} - \frac{x}{2})} \\
& = \frac{x}{t+\delta} \; .
\end{align*}

Let $\phi(g) = \frac{F_t(x+g)}{F_{t-1}(x)}$. To verify property (3) of the definition,
we need to show that $\phi(g) \le 1 + g \frac{x}{t+\delta}$
for any $x \in [-t+1, t-1]$ and any $g \in [-1,1]$. We can write $\phi(g)$ as
\begin{align*}
\phi(g)
& = \frac{F_t(x+g)}{F_{t-1}(x)} \\
& = \frac{2 \Gamma(\frac{t+\delta+1}{2} + \frac{x+g}{2}) \Gamma(\frac{t+\delta+1}{2} - \frac{x+g}{2}) \Gamma(t + \delta)}{\Gamma(\frac{t+\delta}{2} + \frac{x}{2}) \Gamma(\frac{t+\delta}{2} - \frac{x}{2}) \Gamma(t+\delta+1)} \\
& = \frac{2}{t+\delta} \cdot \frac{\Gamma(\frac{t+\delta+1}{2} + \frac{x+g}{2}) \Gamma(\frac{t+\delta+1}{2} - \frac{x+g}{2})}{\Gamma(\frac{t+\delta}{2} + \frac{x}{2}) \Gamma(\frac{t+\delta}{2} - \frac{x}{2})} \; .
\end{align*}
For $g=+1$, using the formula $\Gamma(1+z) = z \Gamma(z)$, we have
$$
\phi(+1)
= \frac{2}{t+\delta} \cdot \frac{\Gamma(\frac{t+\delta}{2} + \frac{x}{2} + 1) \Gamma(\frac{t+\delta}{2} - \frac{x}{2})}{\Gamma(\frac{t+\delta}{2} + \frac{x}{2}) \Gamma(\frac{t+\delta}{2} - \frac{x}{2})}
= \frac{2}{t+\delta} \left(\frac{t+\delta}{2} + \frac{x}{2} \right)
= 1 + \frac{x}{t+\delta} \; .
$$
Similarly, for $g=-1$, using the formula $\Gamma(1+z) = z \Gamma(z)$, we have
$$
\phi(-1)
= \frac{2}{t+\delta} \cdot \frac{\Gamma(\frac{t+\delta}{2} + \frac{x}{2}) \Gamma(\frac{t+\delta}{2} - \frac{x}{2} + 1)}{\Gamma(\frac{t+\delta}{2} + \frac{x}{2}) \Gamma(\frac{t+\delta}{2} - \frac{x}{2})}
= \frac{2}{t+\delta} \left(\frac{t+\delta}{2} - \frac{x}{2} \right)
= 1 - \frac{x}{t+\delta} \; .
$$
We can write any $g \in [-1,1]$ as a convex combination of $-1$ and $+1$, i.e.,
$g = \lambda \cdot (-1) + (1-\lambda) \cdot (+1)$ for some $\lambda \in [0,1]$.
Since $\phi(g)$ is (logaritmically) convex,
\begin{align*}
\phi(g)
& = \phi(\lambda \cdot (-1) + (1-\lambda) \cdot (+1)) \\
& \le \lambda \phi(-1) + (1-\lambda) \phi(+1) \\
& = \lambda \left(1 + \frac{x}{t+\delta}\right) + (1-\lambda) \left(1 - \frac{x}{t+\delta}\right) \\
& = 1 + g \frac{x}{t+\delta} \; .
\end{align*}
\end{proof}

\section{Proofs of Corollaries~\ref{corollary:kt-hilbert-space-olo-regret} and~\ref{corollary:kt-experts-regret}}
\label{section:corollaries_reductions}

First we state some technical lemmas that will be used in the following proofs.

We first prove a lower bound on the potential function of the \ac{KT} estimator. This is a generalization to real numbers of the lower bound proved for integers in \citet{WillemsST95}.
\begin{lemma}[Lower Bound on KT Potential]
\label{lemma:approx_gamma_real}
If $c \ge 1$ and $a,b$ are non-negative reals such that $a + b = c$ then
$$
\ln \left(\frac{\Gamma(a + 1/2) \cdot \Gamma(b + 1/2)}{\pi \cdot \Gamma(c+1)} \right)
\ge - \ln(e \pi \sqrt{2}) -\frac{1}{2} \ln(c) +\ln \left(\left( \frac{a}{c} \right)^a \left( \frac{b}{c} \right)^b\right) \; .
$$
\end{lemma}
%
\begin{proof}
From \cite{WhittakerW27}[p. 263 Ex. 45], we have
\[
\frac{\Gamma(a+1/2)\Gamma(b+1/2)}{\Gamma(a+b+1)} \ge \sqrt{2\pi} \frac{(a+1/2)^a (b+1/2)^b}{(a+b+1)^{a+b+1/2}} \; .
\]
It remains to show that
$$
\sqrt{2\pi} \frac{(a+1/2)^a (b+1/2)^b}{(a+b+1)^{a+b+1/2}} > \frac{\sqrt{\pi}}{e} \frac{1}{\sqrt{a+b}} \left( \frac{a}{a+b} \right)^a \left( \frac{b}{a+b} \right)^b \; ,
$$
which is equivalent to
$$
\frac{(1+\frac{1}{2a})^a (1+\frac{1}{2b})^b}{(1+\frac{1}{a+b})^{a+b+1/2}} > \frac{1}{e\sqrt{2}} \; .
$$
From the inequality $1 \le (1+1/x)^x < e$ valid for any $x \ge 0$, it follows
that $1 \le (1+\frac{1}{2a})^a < \sqrt{e}$ and $1 \le (1+\frac{1}{2b})^b < \sqrt{e}$
and $1 \le (1+1/(a+b))^{a+b} < e$. Hence,
$$
\frac{(1+\frac{1}{2a})^a (1+\frac{1}{2b})^b}{(1+\frac{1}{a+b})^{a+b+1/2}}
> \frac{1}{e \sqrt{1 + \frac{1}{a+b}}}
\ge \frac{1}{e \sqrt{2}} \; .
$$
\end{proof}


\begin{lemma}
\label{lemma:ratio_gamma}
Let $\delta \geq 0$. Then
\[
\frac{\Gamma(\delta+1)}{2^\delta \Gamma(\frac{\delta+1}{2})^2}
\geq \frac{\sqrt{\delta+1}}{\pi}~.
\]
\end{lemma}
%
\begin{proof}
We will prove the equivalent statement that
\[
\ln \frac{\Gamma(\delta+1) \pi}{2^\delta \Gamma(\frac{\delta+1}{2})^2 \sqrt{\delta+1}} \geq 0~.
\]
The inequality holds with equality in $\delta=0$, so it is enough to prove that
the derivative of the right-hand side is positive for $\delta > 0$. The
derivative of the right-hand side is equal to
$$
\Psi(\delta+1) - \frac{1}{2(\delta+1)} -\ln(2) - \Psi\left(\frac{\delta+1}{2} \right) \; ,
$$
where $\Psi(x)$ is the digamma function.

We will use the following upper~\citep{Chen05} and lower bound~\citep{Batir08} to the digamma function, for any $x>0$
\begin{align*}
\Psi(x) &< \ln(x) -\frac{1}{2x} -\frac{1}{12 x^2} +\frac{1}{120 x^4}\\
\Psi(x+1) &> \ln \left(x+\frac{1}{2} \right) \; .
\end{align*}
Hence, we obtain
\begin{align*}
&\Psi(\delta+1) - \frac{1}{2(\delta+1)} - \ln(2) - \Psi \left(\frac{\delta+1}{2} \right) \\
&\quad \ge \ln \left(\delta+\frac{1}{2} \right) - \frac{1}{2(\delta+1)} -\ln(2) -\ln \left(\frac{\delta+1}{2}\right) + \frac{1}{\delta+1} +\frac{1}{3 (\delta+1)^2} -\frac{2}{15 (\delta+1)^4}\\
&\quad = \ln \left(1+\frac{\delta}{\delta+1} \right)+ \frac{1}{2(\delta+1)} -\ln(2)+\frac{1}{3 (\delta+1)^2} -\frac{2}{15 (\delta+1)^4}\\
&\quad \ge \frac{\delta}{\delta+1} - (1-\ln(2))\frac{\delta^2}{(\delta+1)^2}+ \frac{1}{2(\delta+1)} -\ln(2)+\frac{1}{3 (\delta+1)^2}-\frac{2}{15 (\delta+1)^4}\\
&\quad = \frac{\delta^2 \ln(2)+\frac{3}{2}\delta +\frac{5}{6}}{(\delta+1)^2} - \ln(2)-\frac{2}{15 (\delta+1)^4}\\
&\quad = \frac{(\frac{3}{2}-\ln(4)) \delta +\frac{5}{6}-\ln(2)}{(\delta+1)^2} -\frac{2}{15 (\delta+1)^4} \\
&\quad = \frac{15 (\delta+1)^2 \left((\frac{3}{2}-\ln(4)) \delta +\frac{5}{6}-\ln(2)\right)-2}{15(\delta+1)^4} \\
&\quad \geq \frac{15 \left(\frac{5}{6}-\ln(2)\right)-2}{15(\delta+1)^4} >0,
\end{align*}
where in the second inequality we used the elementary inequality $\ln(x+1) \leq x - (1-\ln(2))x^2$ valid for $x \in [0,1]$.
\end{proof}

\begin{lemma}[Lower Bound on Shifted KT Potential]
\label{lemma:lower_bound_gamma}
Let $T \ge 1$, $\delta \ge 0$, and $x \in [-T,T]$. Then
\begin{align*}
\frac{2^T \cdot \Gamma(\delta+1) \Gamma \left(\frac{T+\delta+1}{2} + \frac{x}{2} \right) \cdot \Gamma \left(\frac{T+\delta+1}{2} - \frac{x}{2} \right)}{ \Gamma(\frac{\delta+1}{2})^2 \Gamma(T+\delta+1)}
&\geq\exp\left(\frac{x^2}{2(T+\delta)} +\frac{1}{2} \ln \left(\frac{1+\delta}{T+\delta}\right) - 3\right) \;.
\end{align*}
\end{lemma}
\begin{proof}
Using Lemma~\ref{lemma:approx_gamma_real}, we have
\begin{align*}
&\ln \frac{2^T \cdot \Gamma(\delta+1) \Gamma \left(\frac{T+\delta+1}{2} + \frac{x}{2} \right) \cdot \Gamma \left(\frac{T+\delta+1}{2} - \frac{x}{2} \right)}{ \Gamma(\frac{\delta+1}{2})^2 \Gamma(T+\delta+1)} \\
&\quad \geq \ln \frac{2^{T+\delta} \sqrt{\delta+1} \cdot \Gamma \left(\frac{T+\delta+1}{2} + \frac{x}{2} \right) \cdot \Gamma \left(\frac{T+\delta+1}{2} - \frac{x}{2} \right)}{ \pi \Gamma(T+\delta+1)} \\
&\quad \geq -\ln(e \pi \sqrt{2}) +\frac{1}{2} \ln\left(\frac{1+\delta}{T+\delta}\right) +\ln \left(\left( 1+\frac{x}{T+\delta} \right)^\frac{T+\delta+x}{2} \left( 1+\frac{x}{T+\delta} \right)^\frac{T+\delta-x}{2}\right) \\
&\quad = -\ln(e \pi \sqrt{2}) +\frac{1}{2} \ln\left(\frac{1+\delta}{T+\delta}\right)+ (T+\delta) \, \KL{\frac{1}{2}+\frac{x}{2(T+\delta)}}{\frac{1}{2}} \\
&\quad \geq -\ln(e \pi \sqrt{2}) +\frac{1}{2} \ln\left(\frac{1+\delta}{T+\delta}\right)+ \frac{x^2}{2(T+\delta)} ,
\end{align*}
where in the first inequality we used Lemma~\ref{lemma:ratio_gamma}, in the second one Lemma~\ref{lemma:approx_gamma_real}, and in third one the known lower bound to the divergence $\KL{\frac{1}{2}+\frac{x}{2}}{\frac{1}{2}} \geq \frac{x^2}{2}$. Exponentiating and overapproximating, we get the stated bound.
\end{proof}


\subsection{Proof of Corollary~\ref{corollary:kt-hilbert-space-olo-regret}}

Define the Lambert function $W(x):\R\rightarrow\R$ as the one that satisfies
the equality\footnote{For $x<0$ the Lambert function is multivalued. Hence, to
avoid complication and because we only need non-negative arguments, we will
define it only for positive values of $x$.}
\begin{equation}
\label{eq:lambert}
x=W(x) \exp \left(W(x)\right) \qquad \qquad \text{for $\ge 0$}.
\end{equation}
Lmabert function has the following properties.
%
\begin{lemma}
\label{lemma:lambert}
The Lambert function satisfies $0.6321 \log(x+1) \leq W(x) \leq \log(x+1)$ for $x \ge 0$.
\end{lemma}
%
\begin{proof}
The inequalities are satisfied for $x=0$, hence we in the following we assume $x>0$.
We first prove the lower bound. From \eqref{eq:lambert} we have
\begin{equation}
W(x) = \log\left(\frac{x}{W(x)}\right)~. \label{eq:lm_lambert_1}
%&= \log\left(\frac{x}{\log(x/W(x))}\right). \label{eq:lm_lambert_1b}
\end{equation}
From the first equality, using the elementary inequality $\ln (x) \leq \frac{a}{e} x^\frac{1}{a}$ for any $a>0$, we get
\[
W(x) \leq \frac{1}{a\, e}\left(\frac{x}{W(x)}\right)^a  \ \ \forall a>0,
\]
that is
\begin{equation}
\label{eq:lm_lambert_2}
W(x) \leq \left(\frac{1}{a\, e}\right)^\frac{1}{1+a} x^\frac{a}{1+a} \ \ \forall a>0.
\end{equation}
Using \eqref{eq:lm_lambert_2} in \eqref{eq:lm_lambert_1}, we have
\begin{align*}
W(x)
\geq \log\left(\frac{x}{\left(\frac{1}{a\, e}\right)^\frac{1}{1+a} x^\frac{a}{1+a}}\right)
= \frac{1}{1+a}\log\left(a \, e\, x\right) \ \ \forall a>0~.
\end{align*}
Consider now the function $g(x)=\frac{x}{x+1} - \frac{b}{\log(1+b) (b+1)}
\log(x+1), x\geq b$. This function has a maximum in $x^*=(1+\frac{1}{b})
\log(1+b)-1$, the derivative is positive in $[0,x^*]$ and negative in
$[x^*,b]$. Hence the minimum is in $x=0$ and in $x=b$, where it is equal to
$0$.  Using the property just proved on $g$, setting $a=\frac{1}{x}$, we have
\begin{align*}
W(x)
\geq \frac{x}{x+1} \geq \frac{b}{\log(1+b) (b+1)} \log(x+1) \ \  \forall x\leq b~.
\end{align*}
For $x>b$, setting $a=\frac{x+1}{e x}$, we have
\begin{align}
W(x)
&\geq \frac{e\,x}{(e+1) x + 1} \log(x+1) \geq \frac{e\,b}{(e+1) b + 1} \log(x+1)
\end{align}
Hence, we set $b$ such that
\[
\frac{e\, b}{(e+1)b + 1} = \frac{b}{\log(1+b) (b+1)}
\]
Numerically, $b=1.71825...$, so
\[
W(x) \geq 0.6321 \log(x+1)~.
\]

For the upper bound, we use Theorem~2.3 in \cite{hoorfar2008inequalities}, that says that
\[
W(x) \leq \log\frac{x+C}{1+\log(C)}, \quad \forall x> -\frac{1}{e}, \ C>\frac{1}{e}.
\]
Setting $C=1$, we obtain the stated bound.
\end{proof}

\begin{lemma}
\label{lemma:dual_exp_square}
Define $f(\theta)= \beta \exp\frac{x^2}{2 \alpha}$, for $\alpha,\beta>0$, $x\geq0$. Then
\[
f^*(y)=y \sqrt{\alpha W\left(\frac{\alpha y^2}{\beta^2}\right)} - \beta \exp\left(\frac{W\left(\frac{\alpha y^2}{\beta^2}\right)}{2}\right).
\]
Moreover
\[
f^*(y) \leq y \sqrt{\alpha \log \left(\frac{\alpha y^2}{\beta^2} +1 \right)} - \beta.
\]
\end{lemma}
\begin{proof}
From the definition of Fenchel dual, we have
\begin{align*}
f^*(y)= \max_{x} \  x\, y - f(x) = \max_{x} \  x\, y - \beta \exp\frac{x^2}{2 \alpha} \leq x^*\,y -\beta
\end{align*}
where $x^*= \argmax_{x} x\, y - f(x)$. We now use the fact that $x^*$ satisfies $y = f'(x^*)$, to have
\begin{align*}
x^*=\sqrt{\alpha W\left(\frac{\alpha y^2}{\beta^2}\right)},
\end{align*}
where the $W(\cdot)$ is the Lambert function.
Using Lemma~\ref{lemma:lambert}, we obtain the stated bound.
\end{proof}


\begin{proof}[Proof of Corollary~\ref{corollary:kt-hilbert-space-olo-regret}]
Notice that the KT potential can written as
$$
F_t(x) = \epsilon \cdot \frac{2^t \cdot \Gamma(1) \Gamma \left(\frac{t+1}{2} + \frac{x}{2} \right) \cdot \Gamma \left(\frac{t+1}{2} - \frac{x}{2} \right)}{ \Gamma(\frac{1}{2})^2 \Gamma(t+1)} \; .
$$
Using Lemma~\ref{lemma:lower_bound_gamma} with $\delta = 0$ we can lower bound $F_t(x)$ with
$$
G_t(x) = \epsilon \cdot \exp\left(\frac{x^2}{2t} + \frac{1}{2} \ln \left(\frac{1}{t} \right) -3 \right) \; .
$$
Since $G_t(x) \le F_t(x)$, we have $F^*_t(x) \le G_t^*(x)$. Using Lemma~\ref{lemma:dual_exp_square}, we have
\[
\forall u \in \H \qquad \qquad
F^*_T\left(\norm{u}\right)
\le G^*_T\left(\norm{u}\right)
\le \sqrt{T \log \left(\frac{4 T^2 \norm{u}^2}{\epsilon^2} +1 \right)}+\epsilon\left(1-\frac{1}{2\sqrt{T}}\right).
\]
An application of Theorem~\ref{theorem:hilbert-space-olo-regret-bound} completes the proof.
\end{proof}

\subsection{Proof of Corollary~\ref{corollary:kt-experts-regret}}

\begin{proof}
Let
\begin{align*}
F_t(x) & = \frac{2^t \cdot \Gamma(\delta + 1) \Gamma(\frac{t+\delta+1}{2} + \frac{x}{2}) \Gamma(\frac{t+\delta+1}{2} - \frac{x}{2})}{\Gamma(\frac{\delta+1}{2})^2 \Gamma(t+\delta+1)} \; , \\
G_t(x) & = \exp\left(\frac{x^2}{2(t+\delta)} +\frac{1}{2} \ln \left(\frac{1+\delta}{t+\delta}\right) - 3\right) \; .
\end{align*}
Let $f_t(x) = \ln(F_t(x))$ and $g_t(x)=\ln(G_t(x))$. By Lemma~\ref{lemma:lower_bound_gamma}, $G_t(x) \le F_t(x)$
and therefore $f^{-1}_t(x) \leq g^{-1}_t(x)$ for all $x \ge 0$.
Theorem~\ref{theorem:regret-bound-experts} implies that
$$
\forall u \in \Delta_t \qquad \qquad
\Regret_t(u) \le f_t^{-1}(\KL{u}{\pi}) \le g_t^{-1}(\KL{u}{\pi}) \; .
$$
Setting $t=T$ and $\delta = T/2$, and overapproximating $g_t^{-1}(\KL{u}{\pi})$ we get the stated bound.
\end{proof}


\end{document}
