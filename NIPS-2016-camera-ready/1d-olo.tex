\section{Warm-Up: One-Dimensional Hilbert Space}
\label{section:one-dimensional-hilbert-space-olo}

There are many frameworks for the design and analysis of online learning
algorithms, e.g. the potential function view~\cite{Cesa-Bianchi-Lugosi-2006},
the regularizer view~\cite{Shalev-Shwartz-2011}, relax and
randomize~\cite{Rakhlin-Shamir-Sridharan-2012}. However, they do not provide
much help on how to craft the potential, regularizer, or relaxation.  Here, we
show how the betting view gives a very natural intuition and it also provides a
clear path to the tools needed.

As a warm-up, let us consider an algorithm for OLO over one-dimensional Hilbert
space $\R$.  Let $\{w_t\}_{t=1}^\infty$ be its sequence of predictions on a
sequence of rewards $\{g_t\}_{t=1}^\infty$, $g_t \in [-1,1]$. The total reward
of the algorithm after $t$ rounds is $\Reward_t = \sum_{i=1}^t g_i w_i$.  Let
us also define ``wealth'' of the OLO algorithm as $\Wealth_t = \epsilon +
\Reward_t$, in accordance with \eqref{equation:def_wealth_reward}.
Intuitively, we want the reward to be big on any sequence of $g_t$, so that the
regret will be small.

We now restrict our attention to algorithms whose predictions $w_t$ are of the
form of a bet, that is $w_t = \beta_t \Wealth_{t-1}$, where $\beta_t \in
[-1,1]$.  We will see that the restriction on $\beta_t$ does not prevent us
from obtaining parameter-free algorithms with optimal bounds.

Given the above, it is immediate to see that any coin-betting algorithm that,
on a sequence of coin flips $\{g_t\}_{t=1}^\infty$, $g_t \in [-1,1]$, bets the
amounts $w_t$ can be used as an OLO algorithm in a one-dimensional Hilbert
space $\R$. But, what would be the regret of such OLO algorithms?

Assume that the betting algorithm at hand guarantees that its wealth is at
least $F(\sum_{t=1}^T g_t)$ starting from an endowment $\epsilon$, then
\vspace{-.1cm}
\begin{equation}
\label{equation:one-dimensional-olo-reward-lower-bound}
\Reward_T
= \sum_{t=1}^T g_t w_t
= \Wealth_T \ - \ \epsilon \ge F\left(\sum_{t=1}^T g_t \right) \ - \ \epsilon \; .
\end{equation}
We are almost done, since we just need to convert the lower bound on the reward
to an upper bound on the regret. This can be done using the following lemma
from~\cite{McMahan-Orabona-2014}.
\begin{lemma}[Reward-Regret relationship~\cite{McMahan-Orabona-2014}]
\label{lemma:reward-regret}
Let $V,V^*$ be a pair of dual vector spaces. Let $F:V \to \R \cup \{+\infty\}$
be a proper convex lower semi-continuous function and let $F^*:V^* \to \R \cup
\{+\infty\}$ be its Fenchel conjugate. Let $w_1, w_2, \dots, w_T \in V$ and
$g_1, g_2, \dots, g_T \in V^*$. Let $\epsilon \in \R$. Then,
\[
\underbrace{\sum_{t=1}^T \langle g_t, w_t \rangle}_{\Reward_T} \ge F\left( \sum_{t=1}^T g_t \right) -\epsilon
\qquad \text{if and only if} \qquad
\forall u \in V^*, \quad
\underbrace{\sum_{t=1}^T \langle g_t, u - w_t\rangle}_{\Regret_T(u)} \le F^*(u) + \epsilon\; .
\]
\end{lemma}
\vspace{-.1cm}
Applying the lemma, we get a regret upper bound:
$\Regret_T(u) \le F^*(u) \ + \ \epsilon, \ \forall u \in \H$.

This shows that if we have a betting algorithm that guarantees a minimum wealth
of $F(\sum_{t=1}^T g_t)$, it can be used to design and analyze a
one-dimensional \ac{OLO} algorithm. The faster the growth of the wealth, the
smaller the regret will be.  Moreover, the lemma also shows that trying to
design an algorithm that is adaptive to $u$ is \emph{equivalent} to designing
an algorithm that is adaptive to $\sum_{t=1}^T g_t$.  Also, most importantly,
\emph{methods that guarantee optimal wealth for the betting scenario are
already known}, see, e.g., \cite[Chapter 9]{Cesa-Bianchi-Lugosi-2006}. We can
just re-use them to get optimal online algorithms!
